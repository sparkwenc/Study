\documentclass[article, a4paper, twoside]{universal}

\setshowlvl{1}
\begin{document}
\confighead{}{}{}
\printhead{}{}{1}


\sct{Higher Gross-Zagier}
Ref:~\cite{YZShtukas1}.

\begin{stp}
	Let $k=\bF_{q}$ be a finite field of characteristic $p$, $\ell\neq p$ prime, and $X,X'$ smooth projective geometrically connected curves over $k$ with a finite {\'e}tale double cover $v:X'\ar X$. Denote by $\sgm\in\Gal(X'/X)$ the nontrivial involution, $\eta:F\xT\bs\bA\xT/\cO\xT\ar\{\pm1\}$ the quadratic character.

	Let $G=\PGL_{2}$, the unramified Hecke algebra $\cH_{G}$ is generated by $h_{D}$ with $D$ effective divisors on $X$, and locally $\cH_{G,x}=\bQ[h_{x}]$. Let $A=\bG_{m}$ be the split torus of $G$, its Hecke algebra is $\cH_{A}=\bQ[\bA\xT/\cO\xT]$, and locally $\cH_{A,x}=\bQ[t_{x},t_{x}\xI]$. The Satake transform $\mathrm{Sat}:\cH_{G}\ar\cH_{A}$ is locally given by $h_{x}\amt t_{x}+q_{x}t_{x}\xI$, and identifies $\cH_{G,x}$ with $\bQ[t_{x}+t_{x}\xI]\sbs\cH_{A,x}$. The Eisenstein quotient $a_{\mathrm{Eis}}:\cH_{G}\atr\cH_{\mathrm{Eis}}:=\bQ[\Pic_{X}(k)]^{\ita_{\mathrm{Pic}}}$ is surjective, denote by $\cI_{\mathrm{Eis}}:=\Ker(a_{\mathrm{Eis}})\sbs\cH_{G}$ the Eisenstein ideal, and $Z_{\mathrm{Eis}}:=\Spc{\cH_{\mathrm{Eis}}}$.

\end{stp}

\ssc{Intersection and cup products}

\begin{thm}
	(\cite[Appendix~A]{YZShtukas1}) Let $\cX$ be a Deligne-Mumford stack locally of finite type over $k$ and dimension $n$. The rational Chow groups $\CH{c}{\blt}{(\cX)}_{\bQ}$ of cycles proper over $k$ admits an intersection product. The {\'e}tale cohomology groups $\HH{\text{{\'e}t},c}{\blt}{(\cX_{\oL{k}},\bQ_{\ell})}$ with compact support admits a cup product. The cycle class map $\cl_{X}:\CH{c}{i}{(\cX)}_{\bQ}\ar\HH{\text{{\'e}t},c}{2i}{(\cX_{\oL{k}},\bQ_{\ell})(i)}$ is $\bQ$-linear and intertwines intersection product with cup product. For $i+j=n$ one has
	\[
		\begin{tikzcd}
			\sA{-,-}_{\cX}: & \CH{c}{i}{(\cX)}_{\bQ}\ar[r, phantom, "\tms"]\ar[d, "\cl"] & \CH{c}{j}{(\cX)}_{\bQ}\ar[rr, "(-)\cdot_{X}(-)"]\ar[d, "\cl"] && \CH{c}{n}{(\cX)}_{\bQ}\ar[d, "\cl"]\ar[r, "\Deg"] & \bQ\ar[d, hook]\\
			(-,-): & \HH{\text{{\'e}t},c}{2i}{(\cX_{\oL{k}},\bQ_{\ell})(i)}\ar[r, phantom, "\tms"] & \HH{\text{{\'e}t},c}{2j}{(\cX_{\oL{k}},\bQ_{\ell})(j)}\ar[rr, "(-)\cup(-)"] && \HH{\text{{\'e}t},c}{2n}{(\cX_{\oL{k}},\bQ_{\ell})(n)}\ar[r]& \bQ_{\ell}
		\end{tikzcd}
	\]
\end{thm}

\begin{stp}
	The $G$-Shtukas $\cX=\Sht_{G}^{r}$ over $X$ is a Deligne-Mumford stack locally of finite type with a separated and smooth morphism $\pi_{G}:\Sht_{G}^{r}\ar X^{r}$ of relative dimension $r$. Each effective divisor $D$ gives a self-correspondence $\Sht_{G}^{r}(h_{D})$ of $\Sht_{G}^{r}$ over $X^{r}$, the maps $\ola{p},\ora{p}:\Sht_{G}^{r}(h_{D})\ar\Sht_{G}^{r}$ and $(\ola{p},\ora{p}):\Sht_{G}^{r}(h_{D})\ar\Sht_{G}^{r}\tms\Sht_{G}^{r}$ are representable and proper, this gives a map $H:\cH_{G}\ar{}_{c}\CH{}{2r}(\Sht_{G}^{r}\tms\Sht_{G}^{r})_{\bQ}$ via $h_{D}\amt{(\ola{p}\tms\ora{p})}\zS[\Sht_{G}^{r}(h_{D})]$.

	$\pi_{G}:\Sht_{G}^{r}\ar X^{r}$ admit a stability truncation by open substacks $\pi_{G}^{\lqs d}:\Sht_{G}^{r,\lqs d}\ar X^{r}$, where $d\in\cD$ are functions $\bZ/r\bZ\ar\bZ$ satisfying $d(i)-d(i-1)=\pm1$. The {\'e}tale cohomology of compact support of $\Sht_{G}^{r}$ is
	\[
		\HH{\text{{\'e}t},c}{\blt}{(\Sht_{G,\oL{k}}^{r},\bQ_{\ell})}:=\ilim_{d\in\cD}\HH{\text{{\'e}t},c}{\blt}{(\Sht_{G,\oL{k}}^{r,\lqs d},\bQ_{\ell})}=\ilim_{d\in\cD}\HH{\text{{\'e}t}}{\blt}{(X_{\oL{k}}^{r},R\pi_{G!}^{\lqs d}\uL{\bQ_{\ell}})},
	\]
	note that $\Sht_{G}^{r,\lqs d}$ satisfies $(\ora{p}\cc\ola{p}\xI)(\Sht_{G}^{r,\lqs d})\sbs\Sht_{G}^{r,\lqs d+\Deg{D}}$, this gives a self-correspondence of $\Sht_{G}^{r}$ and thus induces a map $C(h_{D})_{d,d'}:R\pi_{G!}^{\lqs d}\uL{\bQ_{\ell}}\ar R\pi_{G!}^{\lqs d'}\uL{\bQ_{\ell}}$ for any $d'\gqs d+\Deg{D}$, this defines a map $C:\cH_{G}\ar\End{\bQ_{\ell}}{}{(\HH{\text{{\'e}t},c}{\blt}{(\Sht_{G,\oL{k}}^{r},\bQ_{\ell})})}$ via $h_{D}\amt \ilim_{d,d'}C(h_{D})_{d,d'}$.
\end{stp}

\begin{thm}[{\cite[Proposition~5.10, Proposition~7.1, Lemma~5.12, Lemma~7.2, Lemma~7.3]{YZShtukas1}}]
	The maps $H$ and $C$ define an action of $\cH_{G}$ on $\CH{c}{\blt}{(\Sht_{G}^{r})}_{\bQ}$ and $\HH{\text{{\'e}t},c}{\blt}{(\Sht_{G,\oL{k}}^{r},\bQ_{\ell})}$ such that they are self-adjoint, the cycle class map $\cl_{X}:\CH{c}{i}{(\Sht_{G})}_{\bQ}\ar\HH{\text{{\'e}t},c}{2i}{(\Sht_{G,\oL{k}},\bQ_{\ell})(i)}$ is $\cH_{G}$-equivariant for $0\lqs i\lqs 2r$.
\end{thm}

\ssc{Two spectral decompositions}


\spg{Automorphic spectral decompositions}
\begin{thm}[{\cite[Proposition~2.1,Proposition~2.3]{YZShtukas1}}]
	The automorphic kernel function $\bK_{f}$ and orbital integral $\bJ(f,s)$ defined below are convergent,
	\begin{align*}
	  \bJ(f,s):=\sum_{n_{1},n_{2}\in\bZ}\bJ_{n_{1},n_{2}}(f,s),\quad \bK_{f}(g_{1},g_{2}):=\sum_{\gma\in G(F)}f(g_{1}\xI\gma g_{2}), \\ \bJ_{n_{1},n_{2}}(f,s):=\int_{[A]_{n_{1}}\tms[A]_{n_{2}}}\bK_{f}(h_{1},h_{2})\sP{h_{1}h_{2}}^{s}\eta(h_{2})\dif{h_{1}}\dif{h_{2}}.
	\end{align*}
	The map $\mathrm{inv}:G(F)\ar\bP^{1}(F)-\{1\}$ by $\gma\amt bc/ad$ decomposes $G(F)=\coprod_{u\in\bP^{1}(F)-\{1\}}G(F)_{u}$, one has
	\[
		\bJ(f,s)=\sum_{u\in\bP^{1}(F)-\{1\}}\bJ(u,f,s),\quad \bJ(u,f,s)=\sum_{\gma\in A(F)\bs G(F)_{u}/A(F)}\bJ(\gma,f,s).
	\]
\end{thm}

\begin{thm}[{\cite[Theorem~4.3, Lemma~4.4, Proposition~4.5]{YZShtukas1}}]
	For any $f\in C_{c}^{\ift}(G(\bA))$, $\bK_{f}$ and $\bJ(f,s)$ have automorphic spectral decompositions,
	\[
		\bK_{f}=\bK_{f, \mathrm{cusp}}+\bK_{f, \mathrm{sp}}+\bK_{f, \mathrm{Eis}},\quad \bJ(f,s)=\bJ_{\mathrm{cusp}}(f,s)+\bJ_{\mathrm{sp}}(f,s)+\bJ_{\mathrm{Eis}}(f,s).
	\]
	Moreover if $f\in\cI_{\mathrm{Eis}}\sbs\cH_{G}$, then $\bK_{\mathrm{Eis}}=0$ and $\bJ_{\mathrm{sp}}(f,s)=\bJ_{\mathrm{Eis}}(f,s)=0$, $\bJ_{\mathrm{cusp}}(f,s)=\sum_{\pi}\bJ_{\pi}(f,s)$ and
	\[
		\bJ_{\pi}(f,s)=\frac{1}{2}\sP{\oga_{X}}\lda_{\pi}(f)L(\pi_{F'},s+1/2).
	\]
\end{thm}


\spg{Cohomological spectral decompositions}

\begin{thm}[{\cite[Theorem~7.14, Lemma~7.15]{YZShtukas1}}]
	Denote by $V_{\bQ_{\ell}}^{r}:=\HH{\text{{\'e}t},c}{2r}{(\Sht_{G,\oL{k}}^{r},\bQ_{\ell})(r)}$, one has
	\begin{itm}
		\item If $r=0$, then $V_{\bQ_{\ell}}^{r}=\cA_{G}\ot_{\bQ}\bQ_{\ell}$, the decomposition is over $\bQ$:
		\[
			\begin{tikzcd}
				\cH_{G}\ar[r]\ar[d, two heads] & \cH_{\mathrm{Eis}}\tms\End{\bQ}{}{(\cA_{G})} \\
				\cH_{\mathrm{aut}}\ar[r, equal] & \cH_{\mathrm{Eis}}\tms\cH_{\mathrm{cusp}}\ar[u,hook]
			\end{tikzcd},
		\]
		where the underlying space of $Z_{\mathrm{cusp}}:=\Spc{\cH_{\mathrm{cusp}}}$ is the finite set of closed points $\km_{\pi}$ exactly representing everywhere unramified cuspidal automorphic representations $\pi$ of $G$, and $\cH_{\mathrm{cusp}}=\prod_{\pi}E_{\pi}$ is the product of coefficient field. The $\cH_{G}$-module $\cA_{G}$ orthogonally decomposes
		\[
			\cA_{G}=\cA_{\mathrm{Eis}}\op\cA_{\mathrm{cusp}},\quad \Sup(\cA_{\mathrm{Eis}})\sbs Z_{\mathrm{Eis}},\Sup(\cA_{\mathrm{cusp}})=Z_{\mathrm{cusp}}.
		\]
		\item In general, the decomposition is over $\bQ_{\ell}$:
		\red{(can one update this to $\bQ$? Namely, does the definition $\HH{c}{\blt}{(\Sht_{G,\oL{k}},\uL{\bQ})}:=\ilim_{d\in\cD}\HH{}{\blt}{(X^{r}_{\oL{k}},R\pi_{G!}^{\lqs d}\uL{\bQ})}$ work in the sense that the $\ell$-adic comparison $\HH{c}{\blt}{(\Sht_{G,\oL{k}}^{r},\uL{\bQ})}\ot\bQ_{\ell}=\HH{\text{{\'e}t},c}{\blt}{(\Sht_{G,\oL{k}}^{r},\uL{\bQ_{\ell}})}$ is true?)}
		\[
			\begin{tikzcd}
				\cH_{G,\bQ_{\ell}}\ar[d, two heads]\ar[r] & \cH_{\mathrm{Eis},\bQ_{\ell}}\tms\End{\bQ_{\ell}}{}{(V_{\bQ_{\ell}}^{r})} \\
				\oL{\cH_{\ell}}\ar[r, equal] & \cH_{\mathrm{Eis},\ell}\tms\cH_{0,\ell}^{r}\ar[u,hook]
			\end{tikzcd},
		\]
		where $Z_{0,\ell}^{r}={(\Spc{\cH_{0,\ell}^{r}})}^{\mathrm{red}}$ is a finite set of closed points. The $\cH_{G,\bQ_{\ell}}$-module $V_{\bQ_{\ell}}^{r}$ orthogonally decomposes
		\[
			V_{\bQ_{\ell}}^{r}=V_{\mathrm{Eis}}\op V_{0},\quad \Sup(V_{\mathrm{Eis}})\sbs Z_{\mathrm{Eis},\bQ_{\ell}},\Sup(V_{0})=Z_{0,\ell}^{r},
		\]
		Note it's not clear whether all $\km_{\pi}\in Z_{0,\ell}^{r}$ are automorphic. However each $\km_{\pi}\in Z_{\mathrm{cusp}}$ gives $\km_{\pi,\lda}\in Z_{0,\ell}^{r}$ for each place $\lda$ of $E_{\pi}$ above $\ell$. \red{(Does the structure of $Z_{0,\ell}^{r}$ has an refined picture now?)}
	\end{itm}
\end{thm}


\ssc{Geometry of Hitchin fibrations}

\begin{stp}
	Let $T=(\Res_{X'/X}\bG_{m})/\bG_{m}$ be the nonsplit torus of $G$, the $T$-Shtukas $\Sht_{T}^{\mu}$ over $X'$ is a smooth proper Deligne-Mumford stack of pure dimension $r$ with a finite {\'e}tale map $\pi_{T}^{\mu}:\Sht_{T}^{\mu}\ar X^{\prime r}$. The map $\oL{\tta}^{\mu}:\Sht_{T}^{\mu}\ar\Sht_{G}^{r}$ defined by pushing to $X$ via $v:X'\ar X$ factors through $\tta^{\mu}:\Sht_{T}^{r}\ar\Sht_{G}^{\prime r}$, one has the Heegner-Drinfeld cycle $\tta^{\mu}\zS[\Sht_{T}^{\mu}]\in\CH{c}{r}{(\Sht_{G}^{\prime r})}_{\bQ}$.

	\spg{The base space $\cA_{d}$} Let $\cA_{d}$ be the moduli stack classifying triples $(\Dta,a,b)$ where $\Dta\in\Pic_{X}^{d}$ and $a,b\in\HH{}{0}{(X,\Dta)}$ are not both zero, $\cA_{d}$ is indeed a scheme. Let $\oH{X}_{d}$ be the universal family of the Picard stack $\Pic_{X}^{d}$, its open locus $X^{(d)}$ has completement $\Pic_{X}^{d}$, there are maps $\oH{\mathrm{add}}:\oH{X}_{d_{1}}\tms\oH{X}_{d_{2}}\ar\oH{X}_{d_{1}+d_{2}}$. The map $\dta:\cA_{d}\ar\oH{X}_{d}$ via $(\Dta,a,b)\amt(\Dta,a-b)$ has fiber $\cA_{D}$ over $D$ such that $\cA_{D}\cong\HH{}{0}{(X,\cO_{X}(D))}$ via $(\cO_{X}(D),a,a-1)\alr a$, and one has the injective invariant map $\mathrm{inv}_{D}:\cA_{D}(k)\ar\bP^{1}(F)-\{1\}$ via $a\amt1-a\xI$.

	Let $j:X^{(d),\cc}\ahr X^{(d)}\ahr \oH{X}_{d}$ be the locus of multiplicity free divisors, $X^{(d),\cc}$ has an {\'e}tale cover $u:X^{\prime d,\cc}\ar X^{d,\cc}\ar X^{(d),\cc}$ with Galois group $\Gma_{d}:=\{\pm1\}^{d}\rtm \kS_{d}$. Let $\chi_{i}:\{1,\lds,d\}\ar\{\pm1\}$ be characters non-trivial on the first $i$ factors and $\Gma_{d}(i):=\{\pm1\}^{d}\rtm\kS_{i,d-i}$, then $\rho_{i}:=\Ind_{\Gma_{d}(i)}^{\Gma_{d}}(\chi_{i}\bt\mbf{1})$ are irreducible representations of $\Gma_{d}$, and define irreducible local systems $\cL(\rho_{i})$ on $X^{(d),\cc}$ and simple perverse sheaves $\cK_{i}:=j_{!*}(\cL(\rho_{i})[d])[-d]$ on $\oH{X}_{d}$.

	\spg{The fibration $\cN_{\uL{d}}$} Let $\Sgm_{d}$ be the quadruple of nonnegative integers $\uL{d}=(d_{ij})_{i,j\in\{1,2\}}$ with $d_{11}+d_{22}=d_{12}+d_{21}=d$. For each $\uL{d}\in\Sgm_{d}$ let $\oT{\cN}_{\uL{d}}$ be the moduli stack classifying $(\cK_{i},\cK_{j},(\vfi_{ij}))_{i,j\in\{1,2\}}$ where $\cK_{i},\cK_{i}'\in\Pic_{X}$ and $\Deg\cK_{i}'-\Deg\cK_{j}=d_{ij}$, $(\vfi_{ij}):\cK_{1}\op\cK_{2}\ar\cK_{1}'\op\cK_{2}'$ is $\cO_{X}$-linear and satisfies (1) at most one $\vfi_{ij}$ can be $0$, (2) if $d_{11}<d_{22}$, then $\vfi_{11}\neq0$, otherwise $\vfi_{22}\neq0$. If $d_{12}<d_{21}$, then $\vfi_{12}\neq0$, otherwise $\vfi_{21}\neq0$. Let $\cN_{d}$ be the quotient stack $\oT{\cN}_{\uL{d}}/\Pic_{X}$, $\cN_{d}$ is indeed a geometrically connected scheme and there is a map
	\[
		f_{\cN_{\uL{d}}}:\cN_{\uL{d}}\ar\cA_{d},\quad (\cK_{i},\cK_{j}',(\vfi_{ij}))\amt(\cK_{1}'\ot\cK_{2}'\ot\cK_{1}\xI\ot\cK_{2}\xI,\vfi_{11}\ot\vfi_{22},\vfi_{12}\ot\vfi_{21}).
	\]
	Via the canonical map $\HH{\text{{\'e}t}}{1}{(X,\uL{\bZ/2\bZ})}\cong\HH{\text{{\'e}t}}{1}{(\Pic_{X}^{n},\uL{\bZ/2\bZ})}\ar\HH{\text{{\'e}t}}{1}{(\oH{X}_{n},\uL{\bZ/2\bZ})}$, the local system $\cL:=(v\zS\uL{\bQ_{\ell}})^{\sgm=-1}$ maps to local system $\cL_{n}$ over $\oH{X}_{n}$, define the local system $\cL_{\uL{d}}:=\jma_{\uL{d}}\xS(\cL_{d_{11}}\bt\uL{\bQ_{\ell}}\bt\cL_{d_{12}}\bt\uL{\bQ_{\ell}})$ over $\cN_{\uL{d}}$.

	\spg{The fibration $\cM_{d}$} Let $\oT{\cM}_{d}$ be the moduli stack classifying $(\cL,\cL',\afa,\bta)$ where $\cL,\cL'\in\Pic_{X'}$ and $\Deg\cL'-\Deg\cL=d$, $\afa:\cL\ar\cL'$ and $\bta:\cL\ar\sgm\xS\cL'$ are $\cO_{X'}$-linear and not both zero. Let $\cM_{d}$ be the quotient stack $\oT{\cM}_{d}/\Pic_{X}$, $\cM_{d}$ is a Deligne-Mumford stack and there is a map
	\[
		f_{\cM_{d}}:\cM_{d}\ar\cA_{d},\quad (\cL,\cL',\afa,\bta)\amt(\Nrm(\cL')\ot\Nrm(\cL)\xI,\Nrm(\afa),\Nrm(\bta)).
	\]
	The Deligne-Mumford stack $\Hck_{\cM_{d}}^{\mu}$ gives a self-correspondence of $\cM_{d}$ over $\cA_{d}$. Consider $\cA_{d}^{\dmds}$ the open locus of $(\Dta,a,b)$ with $b\neq0$, its preimage $\Hck_{\cM_{d}^{\dmds}}^{1}$ defines a self-correspondence over $\Hck_{\cM_{d}}^{1}$ and thus an endomorphism $f_{\cM_{d}!}[\Hck_{\cM_{d}^{\dmds}}^{1}]:Rf_{\cM_{d}!}\uL{\bQ_{\ell}}\ar Rf_{\cM_{d}!}\uL{\bQ_{\ell}}$.
\end{stp}

\begin{thm}[{\cite[Proposition~3.1, Proposition~6.1]{YZShtukas1}}]
	The two Hitchin type fibrations of $\cA_{d}$ satisfy
	\[
		\begin{tikzcd}
			\cN_{\uL{d}}\ar[rr, "f_{\cN_{\uL{d}}}"]\ar[d, "\jma_{\uL{d}}"] && \cA_{d}\ar[d, hook] && \cM_{d}\ar[ll, "f_{\cM_{d}}"']\ar[d, "\ima_{d}"] \\
			(\oH{X}_{d_{11}}\tms\oH{X}_{d_{22}})\tms_{\Pic_{X}^{d}}(\oH{X}_{d_{12}}\tms\oH{X}_{d_{21}})\ar[rr, "\oH{\mathrm{add}}\tms\oH{\mathrm{add}}"] && \oH{X}_{d}\tms_{\Pic_{X}^{d}}\oH{X}_{d} && \oH{X}_{d}'\tms_{\Pic_{X}^{d}}\oH{X}_{d}'\ar[ll, "\oH{v}_{d}\tms\oH{v}_{d}"']
		\end{tikzcd}
	\]
	\begin{itm}
		\item $\jma_{\uL{d}},\ima_{d}$ are an open embeddings, $f_{\cN_{\uL{d}}},f_{\cM_{d}}$ are proper.
		\item If $d\gqs2g'-1$, then $\cN_{\uL{d}},\cM_{d}$ are smooth of dimension $2d-g+1$.
	\end{itm}
\end{thm}

\begin{thm}[{\cite[Proposition~3.2, Theorem~6.5]{YZShtukas1}}]
	Define $\bI_{r}(f)$ and $\bJ_{r}(f)$ as
	\[
		\bJ_{r}(f)=(\dif{}^{r}/\dif{s}^{r})\bJ(f,s)|_{s=0},\quad \bI_{r}(f):=\sA{\tta\zS^{\mu}[\Sht_{T}^{\mu}],f*\tta\zS^{\mu}[\Sht_{T}^{\mu}]}_{\Sht_{G}^{\prime r}}.
	\]
	If $f=h_{D}$ with $D$ an effective divisor of degree $d\gqs\max\{2g'-1,2g\}$, then
	\begin{align*}
	  \bJ_{r}(h_{D})&=\sum_{a\in\cA_{D}(k)}(\log{q})^{r}\sum_{\uL{d}\in\Sgm_{d}}{(2d_{12}-d)}^{r}\Trc\sR{\Frb_{a},(Rf_{\cN_{\uL{d}*}}\cL_{\uL{d}})_{\oL{a}}},\\
	  \bI_{r}(h_{D})&=\sum_{a\in\cA_{D}(k)}\Trc\sR{(f_{\cM_{d}!}[\Hck_{\cM_{d}^{\dmds}}^{1}])_{a}^{r}\cc\Frb_{a},(Rf_{\cM_{d}*}\uL{\bQ_{\ell}})_{\oL{a}}}.
	\end{align*}
\end{thm}

\begin{thm}[{\cite[Proposition~8.2, Proposition~8.3, Proposition~8.5, Theorem~8.1]{YZShtukas1}}]
	For $d\gqs\max\{2g'-1,2g\}$ and $\uL{d}\in\Sgm_{d}$, there are canonical isomorphisms
	\begin{align*}
	  Rf_{\cN_{\uL{d}*}}\cL_{\uL{d}}\cong(\cK_{d_{11}}\bt\cK_{d_{12}})|_{\cA_{d}},\quad Rf_{\cM_{d}*}\uL{\bQ_{\ell}}\cong\Op_{i,j=0}^{d}(\cK_{i}\bt\cK_{j})|_{\cA_{d}},
	\end{align*}
	and the action of $f_{\cM_{d}!}[\Hck_{\cM_{d}^{\dmds}}^{1}]$ on $Rf_{\cM_{d}*}\uL{\bQ_{\ell}}$ preserves direct summands and acts on $\cK_{i}\bt\cK_{j}$ as scalar $(d-2j)$. Thus $Rf_{\cM_{d}*}\uL{\bQ_{\ell}}\cong\Op_{\uL{d}\in\Sgm_{d}}Rf_{\cN_{\uL{d}*}}\cL_{\uL{d}}$, and one has ${(\log{q})}^{-r}\bJ_{r}(h_{D})=\bI_{r}(h_{D})$ for all $h_{D}\in\cH_{G}$ with $\Deg{D}=d$.
\end{thm}

\ssc{The comparison equality}

\begin{thm}[{\cite[Theorem~9.2]{YZShtukas1}}]
	The equality $(\log{q})^{-r}\bJ_{r}(f)=\bI_{r}(f)$ holds for all $f\in\cH_{G}$.
\end{thm}
\begin{rmk}
	This essentially comes from an algebraic lemma (c.f., \cite[Lemma~9.1]{YZShtukas1}):

	Let $E$ be any field of characteristic $0$, $I$ any nonzero ideal of $R=E[t+t\xI]$, and $R_{\gqs m}:=E[t^{n}+t^{-n}]_{n\gqs m}\sbs R$, then $R_{\gqs m}\ar R/I$ is surjective for all $m\gqs1$.
	\red{I am a bit confused, the contradiction arguments on Page 875 still hold if one replaces all ``$n\gqs m$'' by ``$n=mk, k\gqs1$''. Thus the arguments implies the composition $E[t^{m}+t^{-m}]\ar E[t+t^{-1}]\ar E[t+t^{-1}]/I$ is surjective, but this cannot be true as the first arrow is just $\bA^{1}\ar\bA^{1}$ by the Chebyshev polynomial, which is ramified, thus the composition $Z\ar\bA^{1}\ar\bA^{1}$ cannot be a closed embedding for any $Z$. Did I miss some crucial point here......?}
\end{rmk}

\begin{thm}[{\cite[Theorem~1.6]{YZShtukas1}}]
	One has a decomposition of the Heegner-Drinfeld cycle
	\[
		\cl{(\tta^{\mu}\zS[\Sht_{T}^{\mu}])}=[\Sht_{T}]_{\mathrm{Eis}}+\sum_{\km\in Z_{0,\ell}^{\prime r}}[\Sht_{T}]_{\km}\in\HH{\text{{\'e}t},c}{2r}{(\Sht_{G,\oL{k}}^{\prime r},\bQ_{\ell})(r)}.
	\]
	Consider closed points $\km_{\pi,\lda}\in Z_{0,\ell}^{\prime r}$ where $\pi$ is an everywhere unramified cuspidal automorphic representation of $G$ and $\lda$ a place of $E_{\pi}$ above $\ell$, one has an equality in $E_{\pi,\lda}$,
	\[
		\frac{\sP{\oga_{X}}}{2(\log{q})^{r}}L^{(r)}(\pi_{F'},1/2)=([\Sht_{T}]_{\km_{\pi,\lda}},[\Sht_{T}]_{\km_{\pi,\lda}}).
	\]
\end{thm}

%% \sct{Higher Gross-Zagier, \Rnum{2}}
%% Ref:~\cite{YZShtukas2}.


\printref
\end{document}

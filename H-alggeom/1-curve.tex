\documentclass[article, a4paper, twoside]{universal}

\setshowlvl{1}
\begin{document}
\confighead{}{}{}
\printhead{}{}{1}


\sct{Kodaira dimension of moduli of curves}

\ssc{Preliminaries}

\begin{thm}[{\cite[Theorem~2]{AC1987}}]
	Let $g\gqs3,n\gqs0$ be integers, then
	\begin{enr}
		\item $\Pic(\cM_{g,n})$ is freely generated by $\{\lda,\psi_{1},\ldots,\psi_{n}\}$
		\item $\Pic(\oL{\cM}_{g,n})$ is freely generated by $\{\lda,\psi_{1},\ldots,\psi_{n},\dta_{0},\dta_{a;\{i_{1},\ldots,i_{h}\}},\ldots\}$, where
		\[
			\text{$0\lqs a\lqs[g/2], 0\lqs h\lqs n, 1\lqs i_{1}<\cdots<i_{h}\lqs n$, and $h\gqs2$ if $a=0$}
		\]
	\end{enr}
\end{thm}

\ssc{General strategy}

Assume $g\gqs4$ throughout the note. To prove $\oL{\cM}_{g}$ is of general type, it suffices to prove $K_{\oL{\cM}_{g}}$ is nef and big, so it suffices to find an effective divisor $D$ such that $K_{X}-D$ is ample. Now $K_{\oL{\cM}_{g}}$ is computed in \cite{HM1982}, and the ampleness criterion is in \cite{Mumford1977Stability}, so one has the slope criterion.

\begin{thm}[{\cite[Theorem~2~bis.]{HM1982}}]
	On the smooth variety $\oL{\cM}_{g,\mathrm{sm}}$, one has
	\[
		K_{\oL{\cM}_{g}}=13\lda-2\dta_{0}-3\dta_{1}-2\dta_{2}-\cdots-2\dta_{[g/2]}.
	\]
\end{thm}

\begin{thm}[{\cite[Corollary~5.18]{Mumford1977Stability}}]
	$(12e-4)\lda-e\dta$ is ample for $e\gqs5$.
\end{thm}

\begin{cor}[Slope Criterion]
	$\oL{\cM}_{g}$ is of general type if one can find an effective divisor
	\[
		D=a\lda-b_{0}\dta_{0}-\cdots-b_{[g/2]}\dta_{[g/2]},
	\]
	with $a/b_{i}<13/2, i\neq1$ and $a/b_{1}<13/3$.
\end{cor}

\begin{rmk}
	So the strategy along this line is to
	\begin{enr}
		\item study special divisors by imposing some geometric conditions
		\item find their coefficients and show they meet the slope criterion
	\end{enr}
\end{rmk}

Denote $\kpa:=\kpa(\oL{\cM}_{g,n})$, $d:=\Dim\oL{\cM}_{n,g}$. Let \red{NK} stand for ``not known'', then we summarize as follows.

\begin{table}[H]
	\centering
	\begin{tabular}{cc}
		\begin{tabular}[t]{cccc}
			\toprule
		  	$g$ & $n$   & $\kpa$ & Reference \\
			\midrule
		  	$1$ & $\lqs10$ & $-$ & \cite{Belorousski1998} \\
				& $=11$ & $0$ & \cite{BF2006Moduli} \\
				& $\gqs12$ & $1$ & \cite{BF2006Moduli} \\
		  	\hline
			$2$ & $\lqs13$ & $-$ & \cite{AB2021} \\
				& $=14$ & $11$& \cite{BM2021Kodaira} \\
				& $\gqs15$ & $d$ & \cite{BM2021Kodaira} \\
		  	\hline
		  	$3$ & $\lqs14$ & $-$ & \cite{CF2007} \\
				& $=15$ & \red{NK} & \\
		  		& $\gqs16$ & $d$ & Barros-Mullane \\
		  	\hline
		  	$4$ & $\lqs15$ & $-$ & \cite{CF2007} \\
				& $\gqs16$ & $d$ & \cite{Farkas2009} \\
		  	\hline
		  	$5$ & $\lqs13$ & $-$ & \cite{FV2013} \\
				& $=14$ & \red{NK} & \\
				& $\gqs15$ & $d$ & \cite{Farkas2009} \\
		  	\hline
		  	$6$ & $\lqs15$ & $-$ & \cite{Logan2003} \\
				& $\gqs16$ & $d$ & \cite{Farkas2009} \\
		  	\hline
			$7$ & $\lqs13$ & $-$ & \cite{FV2013} \\
				& $=14$ & \red{NK} & \\
				& $\gqs15$ & $d$ & \cite{Farkas2009} \\
		  	\hline
		  	$8$ & $\lqs12$ & $-$ & \cite{FV2013} \\
				& $=13$ & \red{NK} & \\
				& $\gqs14$ & $d$ & \cite{Farkas2009} \\
		  	\hline
			$9$ & $\lqs10$ & $-$ & \cite{FV2013} \\
				& $[11,12]$ & \red{NK} & \\
				& $\gqs13$ & $d$ & \cite{Farkas2009} \\
		  	\hline
		  	$10$& $\lqs9$  & $-$ & \cite{FP2005Effective} \\
				& $=10$ & $0$ & \cite{FP2005Effective} \\
				& $\gqs11$ & $d$ & \cite{FP2005Effective} \\
		  	\hline
		  	$11$& $\lqs10$ & $-$ & \cite{Logan2003} \\
				& $=11$ & $19$& \cite{Logan2003} \\
				& $\gqs12$ & $d$ & \cite{Farkas2009} \\
		  	\bottomrule
		\end{tabular} &
		\begin{tabular}[t]{cccc}
		  	\toprule
		  	$g$ & $n$   & $\kpa$ & Reference \\
		  	\midrule
		  	$12$& $\lqs7$  & $-$ & \cite{AB2021} \\
				& $=8$  & $\leq39$ & \cite{AB2021} \\
				& $=9$  & \red{NK} & \\
				& $\gqs10$ & $d$ & \cite{Kadikoylu2020} \\
		  	\hline
		  	$13$& $\lqs4$  & $-$ & \cite{AB2021} \\
				& $[5,8]$ & \red{NK} & \\
				& $\gqs9$  & $d$ & \cite{FJP2021} \\
		  	\hline
			$14$& $\lqs3$  & $-$ & \cite{AB2021} \\
				& $[4,9]$ & \red{NK} & \\
				& $\gqs10$ & $d$ & \cite{AB2021} \\
		  	\hline
		  	$15$& $\lqs2$  & $-$ & \cite{AB2021} \\
				& $[3,9]$ & \red{NK} & \\
				& $\gqs10$ & $d$ & \cite{AB2021} \\
		  	\hline
		  	$16$& $=0$  & $\lqs43$ & \cite{AB2021} \\
				& $[1,7]$ & \red{NK} & \\
				& $\gqs8$ & $d$ & \cite{Kadikoylu2020} \\
		  	\hline
			$17$& $\lqs7$ & \red{NK} & \\
				& $\gqs8$ & $d$ & \cite{Kadikoylu2020} \\
		  	\hline
		  	$18$& $\lqs8$ & \red{NK} & \\
				& $\gqs9$ & $d$ & \cite{Farkas2009} \\
		  	\hline
			$19$& $\lqs6$ & \red{NK} & \\
				& $\gqs7$ & $d$ & \cite{Farkas2009} \\
		  	\hline
			$20$& $\lqs5$ & \red{NK} & \\
				& $\gqs6$ & $d$ & \cite{Farkas2009} \\
		  	\hline
		  	$21$& $\lqs3$ & \red{NK} & \\
				& $\gqs4$ & $d$ & \cite{Farkas2009} \\
		  	\hline
			$22$& $\gqs0$ & $d$ & \cite{FJP2020} \\
		  	$23$& $\gqs0$ & $d$ & \cite{FJP2020} \\
		  	\hline
		  	$\gqs24$& $\gqs0$ & $d$ & \cite{EH1987Kodaira} \\
		  	\bottomrule
		\end{tabular}
	\end{tabular}
\end{table}


\ssc{Eisenbund-Harris-Mumford}

\begin{thm}[{\cite[Theorem~A]{EH1987Kodaira}}]
	$\oL{\cM}_{g,n}$ is of general type for all $g\gqs24,n\gqs0$.
\end{thm}

\begin{rmk}
	To achieve this, the special divisors studied in question are
	\begin{enr}
		\item Brill-Noether divisors $D_{s}^{r}\sbs\oL{\cM}_{g}$
		\item Gieseker-Petri divisors $E_{s}^{r}\sbs\oL{\cM}_{g}$
		\item Weierstrass divisors $W\sbs\oL{\cM}_{2,1}$
	\end{enr}
	Key facts and arguments about them are collected in this section.
\end{rmk}

\sss{Brill-Noether divisors}

\begin{thm}[Brill-Noether Theorem]
	Let $C$ be a smooth projective curve of genus $g$, then the Brill-Noether locus is defined by
	\[
		W_{d}^{r}(C)=\{L\in\Pic(C)\mid h^{0}(L)\gqs r+1\}.
	\]
	If $C$ is general, then
	\begin{enr}[label=(\arabic*)]
		\item \cite{GH1980}: $\Dim W_{d}^{r}(C)=\rho(g,r,d)=g-(r+1)(d-g+r)$.
		\item \cite{Gieseker1982Petri}: $W_{d}^{r}(C)$ is smooth away from $W_{d}^{r+1}(C)$.
		\item \cite{FL1981}: $W_{d}^{r}(C)$ is irreducible when $\rho(g,r,d)>0$.
	\end{enr}
\end{thm}

% (A digression: As learnt from a seminar, I learnt that in the proof of \cite{FJP2020}, Farkas has to prove the divisors $\cM_{23,12}^{1},\cM_{23,17}^{2},\cM_{23,20}^{3}$ all have distinct support.)

\begin{thm}[{\cite[Theorem~1]{EH1987Kodaira}}]
	Assume $g=(r+1)(s-1)-1, s\gqs3, d=rs-1$. Then $\rho(g,r,d)=-1$, the Brill-Noether locus $D_{s}^{r}\eqv\oL{\cM}_{g,d}^{r}$ is a divisor, one has
	\[
		D_{s}^{r}=c\sR{(g+e)\lda-\frac{g+1}{6}\dta_{0}-\sum_{i=1}^{[g/2]}i(g-i)\dta_{i}}, c\in\bQ_{>0}.
	\]
\end{thm}

\begin{cor}
	If $g\gqs25$ and is odd, take $r=1$, then $s\gqs14$, $D_{(g+1)/2}^{1}$ meets the slope criterion.

	For $g=24$, $D_{6}^{4}$ meets the slope criterion.

	For $g=26$, $D_{10}^{2}$ meets the slope criterion.
\end{cor}


\sss{Gieseker-Petri divisors}
\begin{thm}[{\cite[Theorem~2]{EH1987Kodaira}}]
	Assume $g=2(s-1), s\gqs3, d=s$. Then $\rho(g,r,d)=0$, the Gieseker-Petri locus $E_{s}^{r}\eqv E_{d}^{1}$ is a divisor, one has
	\[
		E_{d}^{1}=c\sR{e\lda-\sum_{i=0}^{[g/2]}f_{i}\dta_{i}},
	\]
	where
	\begin{align*}
	  &c=\frac{2(2d-4)!}{d!(d-2)!}, e=6d^{2}+d-6,\\
	  &f_{0}=d(d-1), f_{1}=(2d-3)(3d-2), f_{2}=3(d-2)(4d-3),\\
	  &f_{i}>f_{i-1}, \fal e\lqs i\lqs [g/2].
	\end{align*}
	\red{Why $e\lqs i\lqs d-1$...?}
\end{thm}

\begin{cor}
	If $g\gqs28$ and is even, then $d\gqs15$, $E_{d}^{1}$ meets the slope criterion: one simply checks
	\[
		\frac{6d^{2}+d-6}{d(d-1)}<13/2\alR d^{2}-15d+12>0\alR d\gqs15.
	\]
\end{cor}

\begin{stp}
	Let $P_{g}$ be the moduli space of stable $g$-pointed rational curves. Then there are natural maps
	\[
		i:P_{g}\ar\oL{\cM}_{g}, j:\oL{\cM}_{2,1}\ar\oL{\cM}_{g}.
	\]
\end{stp}

\sss{Weierstrass divisors}

\begin{thm}[Theorem~2.2]
	The Weierstrass divisor $W\sbs\oL{\cM}_{2,1}$ is given by
	\[
		W=3\oga-\frac{1}{10}\dta_{0}-\frac{6}{5}\dta_{1}.
	\]
\end{thm}

\begin{thm}[Theorem~2.1, Theorem~3.1]
	Let $D$ be a divisor on $\oL{\cM}_{g}$ such that $D=a\lda-\sum_{i=0}^{[g/2]}b_{i}\dta_{i}$.
	\begin{enr}
		\item If $j\xS D=qW$, then
		\[
			q=b_{2}/3, a=5b_{1}-2b_{2}, b_{0}=\frac{b_{1}}{2}-\frac{b_{2}}{6}.
		\]
		\item If $i\xS D=0$, then
		\[
			b_{i}=\frac{i(g-i)}{g-1}b_{1}, \fal 2\lqs i\lqs [g/2].
		\]
	\end{enr}
\end{thm}


\begin{thm}[Proposition~4.1]
	If $g=(r+1)(s-1)-1, s\gqs3, d=rs-1$. Then $D_{s}^{r}$ meets neither $i(P_{g})$ nor $j(\oL{\cM}_{2,1}-W)$.
\end{thm}


\sct{Stabilization of line bundles}

Ref: \href{https://arxiv.org/pdf/2501.18825}{arXiv:2501.18825}

\begin{stp}
	Let $X$ be a smooth irreducible projective curve of genus $g$ over $\bC$, and $f:X\ar\bP^{1}$ a surjective morphism of degree $n$.
\end{stp}

\begin{thm}[Section~3, Proposition]
	In the case $g=0$, consider any integer $k=qn+i$ with $0\lqs i< n$, then
	\[
		f\zS\cO_{X}(k)=(i+1)\cO_{\bP^{1}}(q)\op(n-i-1)\cO_{\bP^{1}}(q-1).
	\]
\end{thm}


\begin{thm}[Section~4, Proposition]
	In the case $g=1$, let $\cE(r,d)$ be the set of isomorphism classes of indecomposable vector bundles over $X$ with rank $r$ and degree $d$, with $\cE_{r}$ as the unique class in $\cE(r,0)$ such that $h^{0}(X,\cE_{r})\neq0$. Given $\cE\in\cE(r,d)$ and $q\in\bZ$ such that $0\lqs d-qrn<rn$, then
	\[
		f\zS\cE=\begin{cases}
		  \cO_{\bP^{1}}(q)\op(rn-2)\cO_{\bP^{1}}(q-1)\op\cO_{\bP^{1}}(q-2) & \cE\ot f\xS\cO_{\bP^{1}}(-q)=\cE_{r} \\
		  (d-qrn)\cO_{\bP^{1}}(q)\op((q+1)rn-d)\cO_{\bP^{1}}(q-1) & \T{otherwise}
		\end{cases}
	\]
\end{thm}

\begin{thm}[Section~5, Theorem, Proposition]
	In general, one has $f\zS(K_{X}\ot\cE\xS)=K_{\bP^{1}}\ot(f\zS\cE)\xS$.

	For a line bundle $\cL$, and given $f\zS\cL=\op_{j=1}^{n}\cO_{\bP^{1}}(n_{j})$, let $s(\cL,f):=\max_{1\lqs i,j\lqs n}\sP{n_{i}-n_{j}}$ be the range, then
	\begin{itm}
		\item if $n>2g-2$, $s(\cL,f)\lqs2$,
		\item if $n>g-1$,
		\begin{itm}
			\item if $\cL$ is general, then $s(\cL,f)\lqs1$,
			\item if $d=g-1$, then $s(\cL,f)\lqs2$,
			\item if $d=g-2$, then $s(\cL,f)\lqs3$, equality holds exactly when $\cL=K_{X}\ot f\xS\cO_{\bP^{1}}(-1)$,
			\item if $d=g$, then $s(\cL,f)\lqs3$, equality holds exactly when $\cL=f\xS\cO_{\bP^{1}}(1)$,
		\end{itm}
		\item if $n$ is odd, then $s(\cL,f)\lqs (2g-5/2)/n+5/2$,
		\item if $n$ is even, then $s(\cL,f)\lqs(2g-2)/n+5/2$.
	\end{itm}
\end{thm}


\printref
\end{document}

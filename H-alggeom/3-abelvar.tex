\documentclass[article, a4paper, twoside]{universal}

\setshowlvl{1}
\begin{document}
\confighead{}{}{}
\printhead{}{}{1}


\sct{Moduli of abelian varieties}

\ssc{Top weight rational cohomology}
Ref:~\cite{BBCMMW2020}

The paper computed the top-weight rational cohomology of the moduli stack $\cA_{g}$ of principally polarized abelian varieties of dimension $g$ in a tropical way. More specifically, it computed $\Gr_{g(g+1)}^{W}\mathrm{H}_{}^{\blt}(\cA_{g};\bQ)$ for $g=5,6,7$, and gives vanishing results for $g=8,9,10$.

\sss{Preliminaries}
\begin{thm}[{\cite[Theorem~5.8]{CGP2021}}]
	Let $\cX\sbe\oL{\cX}$ be an irreducible smooth variety or Deligne-Mumford stack of dimension $d$ over $\bC$, with a normal crossing compactification $\oL{\cX}$ and boundary $\cD$. Let $\Dta$ be the dual complex of $\cD$, then
	\[
		\Gr_{2d}^{W}\mathrm{H}_{}^{k}(\cX;\bQ)\cong \oT{\mathrm{H}}_{2d-k-1}^{}(\Dta;\bQ).
	\]
\end{thm}

\begin{thm}[{\cite[Theorem~3.1, Theorem~4.4]{BBCMMW2020}}]
	For any admissible decomposition $\Sgm$, $\cA_{g}$ admits a toroidal compactification $\oL{\cA}_{g}^{\Sgm}$ with the dual complex of the boundary being $LA_{g}^{\text{trop},\Sgm}$. In particular, take $\Sgm=\Sgm_{g}^{P}$ to be the perfect cone decomposition, then the perfect chain complex $P\zB^{(g)}$ computes the cohomology,
	\[
		\Gr_{g(g+1)}^{W}\mathrm{H}_{}^{k}(\cA_{g};\bQ)\cong \oT{\mathrm{H}}_{g(g+1)-k-1}^{}(LA_{g}^{\text{trop},\Sgm};\bQ)\cong \mathrm{H}_{g(g+1)-k-1}^{}(P\zB^{(g)}).
	\]
\end{thm}

\begin{thm}[{\cite[Theorem~4.13]{BBCMMW2020}}]
	The Voronoi complex $(V\zB^{(g)},d\zB)$ consisting of those of $(P\zB^{(g)},\ptl\zB)$ having non-trivial intersection with $\Oga_{g}$ fits in a short exact sequence.
	\[
		\begin{tikzcd}
			0\ar[r] & P\zB^{(g-1)}\ar[r] & P\zB^{(g)}\ar[r] & V\zB^{(g)}\ar[r] & 0
		\end{tikzcd}
	\]
\end{thm}

% \begin{thm}[{\cite[Theorem~5.15]{BBCMMW2020}}]
% 	The inflation complex $I\zB^{(g)}$ consisting of those of $P\zB^{(g)}$ of rank $\lqs g-1$ and of rank $g$ with a coloop is acyclic.
% \end{thm}

\sss{The computation}

As summarized above, the top-weight cohomology computation boils down to working out properties of $P\zB^{(g)}$, which requires a careful study of $V\zB^{(g)}$ and $P\zB^{(g-1)}$ inductively.

\begin{itm}
	\item $g=3$, it suffices to compute $P\zB^{(3)}$, use the fact dating back to \cite{Voronoi1908} that the top dimensional perfect cone is generated by $K_{4}$, hence computation of $P\zB^{(3)}$ reduces to determine whether there exist non-alternating automorphisms of the matroid defined by $K_{4}$ with $i$ edges removed.
	\item $g=4$, it suffices to compute $V\zB^{(4)}$, which comes from an improvement of results of \cite{LS1978}, in which $\SL_{4}(\bZ)$-alternating (instead of $\GL_{4}(\bZ)$) perfect cones are studied.
	\item $g=5,6,7$, it suffices to compute $V\zB^{(g)}$, which comes from \cite[Theorem~4.3]{EVGS2013}, and carefully combine two sequences in case $g=6,7$.
	\item $g=8,9,10$, use vanishing results from \cite[Theorem~4.5]{SEVKM2019}.
\end{itm}

	Hence, the computation of the top weight cohomology of $\cA_{g}$ can be reduced to the computation of the homology of the dual complex of the boundary of its perfect toriodal compactification, whose computation can be in turn reduced to the computation of Voronoi complex, which is far from well-known in higher dimensions.

\printref
\end{document}

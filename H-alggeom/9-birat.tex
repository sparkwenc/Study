\documentclass[article, a4paper, twoside]{universal}

\setshowlvl{1}
\begin{document}
\confighead{}{}{}
\printhead{}{}{1}


\sct{Birational geometry}

\ssc{Singularities}
Ref:~\cite{KM1998Birational}.

\begin{dfn}[2.11, 2.12]
	Let $Y$ be a normal variety, say that $Y$ has
	\begin{itm}
		\item \tbf{canonical singularities} if $K_{Y}$ is $\bQ$-Cartier and $f\zS\cO_{X}(mK_{X})=\cO_{Y}(mK_{Y})$ for every resolution of singularities $f:X\ar Y$.
		\item \tbf{terminal singularities} if $K_{Y}$ is $\bQ$-Cartier and $f\zS\cO(mK_{X}-E)=\cO_{Y}(mK_{Y})$ for every resolution of singularities $f:X\ar Y$ where $E$ is the reduced exceptional divisor.
	\end{itm}
\end{dfn}

\begin{dfn}[2.13]
	Let $X$ be a normal and proper variety, say that $X$ is a \tbf{minimal model} if $X$ has terminal singularities and $K_{X}$ is nef.
\end{dfn}

\begin{dfn}[2.25, 2.28, 2.34, 2.37, 6.22]
	Let $(X,\Dta)$ be a pair where $X$ is normal and $D=\sum a_{i}D_{i}$ a sum of distinct prime divisors, assume $m(K_{X}+D)$ is Cartier for some $m>0$.


	Let $f:Y\adr X$ be a birational morphism from a normal variety $Y$ with $\Ex(f)=E$ and $E_{i}$ its irreducible exceptional divisors, then $f\zS\xI\Dta=\sum a_{i}f\zS\xI D_{i}$. There exists $a(E_{i},X,\Dta)\in\bQ$ such that $ma(E_{i},X,\Dta)\in\bZ$ and
	\[
		\cO_{Y}(m(K_{Y}+f\zS\xI\Dta))\cong f\xS\cO_{X}(m(K_{X}+\Dta))(\sum ma(E_{i},X,\Dta)E_{i}),
	\]
	$E_{i}$ is called \tbf{crepant} if $a(E_{i},X,\Dta)=0$, $f$ is called \tbf{crepant} if $a(E_{i},X,D)=0$ for all $i$.

	Define the \tbf{discrepancy} of $(X,\Dta)$ to be $\Dsc(X,\Dta):=\inf_{E}\{a(E,X,\Dta):\T{$E$ is an exceptional divisor over $X$}\}$, and the total discrepancy $\totDsc(X,\Dta):=\inf_{E}\{a(E,X,\Dta):\T{$E$ is a divisor over $X$}\}$.


	Say that $(X,\Dta)$ is
	\begin{itm}
		\item \tbf{terminal} if $\Dsc(X,\Dta)>0$,
		\item \tbf{canonical} if $\Dsc(X,\Dta)\gqs0$,
		\item \tbf{klt (Kawamata log terminal)} if $\Dsc(X,\Dta)>-1$ and $\sFl{\Dta}\lqs0$,
		\item \tbf{plt (purely log terminal)} if $\Dsc(X,\Dta)>-1$,
		\item \tbf{lc (log canonical)} if $\Dsc(X,\Dta)\gqs-1$,
	\end{itm}
	assume that $0\lqs a_{i}\lqs1$, say that $(X,\Dta)$ is \tbf{dlt (divisorial log terminal)} if there is a closed subset $Z\sbs X$ such that (1) $X\ssm Z$ is smooth, $\Dta|_{X\ssm Z}$ is a simple normal crossing divisor, (2) if $f:Y\ar X$ is birational and $E\sbs Y$ is an irreducible divisor such that $\Ctr_{X}E\sbs Z$, then $a(E,X,\Dta)>-1$.
\end{dfn}

\begin{thm}[Kodaira Vanishing, 2.70]
	Let $(X,\Dta)$ be a proper klt pair, $N$ a $\bQ$-Cartier Weil divisor on $X$ such that $N\eqv M+\Dta$ where $M$ is a nef and big $\bQ$-Cartier $\bQ$-divisor, then $\HH{}{i}{(X,\cO_{X}(-N))}=0$ for $0\lqs i <\Dim X$.
\end{thm}

\begin{thm}[Non-vanishing, 3.4]
	Let $X$ be a proper variety, $D$ a nef Cartier divisor, $G$ a $\bQ$-divisor. If $aD+G-K_{X}$ is $\bQ$-Cartier, nef and big for some $a>0$, $(X,-G)$ is klt, then for all $m\gg0$, $\HH{}{0}{(X,mD+\sCe{G})}\neq0$.
\end{thm}

\begin{thm}[Basepoint-free, 3.3]
	Let $(X,\Dta)$ be a proper klt pair with $\Dta$ effective, $D$ a nef Cartier divisor such that $aD-K_{X}-\Dta$ is nef and big for some $a>0$. Then $\sP{bD}$ is basepoint-free for all $b\gg0$.
\end{thm}

\begin{thm}[Rationality, 3.5]
	Let $(X,\Dta)$ be a proper klt pair with $\Dta$ effective and $K_{X}+\Dta$ not nef. Take $a(X)>0$ such that $a(X)\cdot(K_{X}+\Dta)$ is Cartier, let $H$ be a nef and big Cartier divisor, then $r(H):=\max\{t\in\bR:H+t(K_{X}+\Dta)\T{ is nef}\}$ is a rational number of the form $u/v$ with $0<v\lqs a(X)(\Dim X+1)$.
\end{thm}

\begin{thm}[Cone theorem, 3.7]
	Let $(X,\Dta)$ be a projective klt pair with $\Dta$ effective, then
	\begin{enr}[label=(\arabic*)]
		\item there are countably many rational curves $C_{j}\sbs X$ such that $0<-(K_{X}+\Dta)C_{j}\lqs2\Dim X$ and
		\[
			\oL{\NE}(X)=\oL{\NE}(X)_{(K_{X}+\Dta)\gqs0}+\sum\bR_{\gqs0}[C_{j}].
		\]
		\item for any $\eps>0$ and ample $\bQ$-divisor $H$,
		\[
			\oL{\NE}(X)=\oL{\NE}(X)_{(K_{X}+\Dta+\eps H)\gqs0}+\sum_{\T{finite}}\bR_{\gqs0}[C_{j}].
		\]
		\item (contraction) let $F\sbs\oL{\NE}(X)$ be a $(K_{X}+\Dta)$-negative extremal face, then there is a unique morphism $c_{F}:X\ar Z$ to a projective variety such that $c_{F,*}\cO_{X}=\cO_{Z}$, and an irreducible curve $C\sbs X$ is mapped to a point by $c_{F}$ if and only if $[C]\in F$. If $L$ is a line bundle on $X$ such that $(L\cdot C)=0$ for all $C$ with $[C]\in F$, then there exists a line bundle $L_{Z}$ on $Z$ such that $L\cong c_{F}\xS L_{Z}$.
	\end{enr}
\end{thm}

\begin{dfn}[Flip, 3.33]
	Let $X$ be a normal scheme, $D$ a $\bQ$-divisor on $X$ such that $K_{X}+D$ is $\bQ$-Cartier.

	A \tbf{$(K_{X}+D)$-flipping contraction} is a proper birational morphism $f:X\ar Y$ to a normal scheme $Y$ such that $\Cdm\Ex(f)\gqs2$ and $-(K_{X}+D)$ is $f$-ample.

	A normal scheme $X\xP$ with a proper birational morphism $f\xP:X\xP\ar Y$ is called a \tbf{$(K_{X}+D)$-flip} of $f$ if $K_{X\xP}+D\xP$ is $\bQ$-Cartier, $K_{X\xP}+D\xP$ is $f\xP$-ample and $\Cdm\Ex(f\xP)\gqs2$. The induced rational map $\phi:X\adr X\xP$ is also called a \tbf{$(K_{X}+D)$-flip}.
\end{dfn}

\begin{dfn}[Flop, 6.10]
	Let $X$ be a normal variety, a \tbf{flopping contraction} is a proper birational morphism $f:X\ar Y$ to a normal variety $Y$ such that $\Cdm\Ex(f)\gqs2$ and $K_{X}$ is numerically $f$-trivial.

	If $D$ is a $\bQ$-Cartier $\bQ$-divisor on $X$ such that $-(K_{X}+D)$ is $f$-ample, then the $(K_{X}+D)$-flip of $f$ is also called the \tbf{$D$-flop}, if $X\ar Y$ is extremal, the $D$-flop does not depend on $D$, call $\phi:X\adr X\xP$ the \tbf{flop} of $f$.
\end{dfn}

\ssc{Finite generation of canonical ring}
Ref:~\cite{BCHM2010}.

\begin{thm}[{\cite[Theorem~1.2]{BCHM2010}}]
	Let $(X,\Dta)$ be a klt pair, where $K_{X}+\Dta$ is Cartier. Let $\pi:X\ar U$ be a morphism of quasi-projective varieties. If either (1) $\Dta$ is $\pi$-big and $K_{X}+\Dta$ is $\pi$-pseudo-effective or (2) $K_{X}+\Dta$ is $\pi$-big, then
	\begin{itm}
		\item $K_{X}+\Dta$ has a log terminal model over $U$,
		\item if $K_{X}+\Dta$ is $\pi$-big then $K_{X}+\Dta$ has a log canonical model over $U$,
		\item if $K_{X}+\Dta$ is $\bQ$-Cartier, then the following $\cO_{U}$-algebra is finitely generated,
		\[
			\kR(\pi,K_{X}+\Dta)=\Op_{m\in\bN}\pi\zS\cO_{X}(\sFl{m(K_{X}+\Dta)}).
		\]
	\end{itm}
\end{thm}


\begin{thm}[$A_{n}$]
	Let $f:X\ar Z$ be a pl-flipping contraction for an $n$-dimensional purely log terminal pair $(X,\Dta)$, then the flip $f\xP:X\xP\ar Z$ exists.
\end{thm}

\begin{thm}[$B_{n}$]
	Let $\pi:X\ar U$ be a projective morphism of normal quasi-projective varieties, where $X$ is $\bQ$-factorial of dimension $n$. Let $V$ be a finite dimensional affien subspace of $\mathrm{WDiv}_{\bR}(X)$ defined over $\bQ$, $S$ the sum of finitely many prime divisors and $A$ a general ample $\bQ$-divisor over $U$. Let $(X,\Dta_{0})$ be a divisorially log terminal pair such that $S\lqs\Dta_{0}$. Fix a finite set $\kC$ of prime divisors on $X$.

	Then there are finitely many $1\lqs i\lqs k$ many birational maps $\phi_{i}:X\adr Y_{i}$ over $U$ such that if $\phi:X\adr Y$ is any $\bQ$-factorial weak log canonical model over $U$ of $K_{X}+\Dta$, where $\Dta\in\cL_{S+A}(V)$, which only contracts elements of $\kC$ and which does not contract every component of $S$, then there is an index $1\lqs i\lqs k$ such that the induced birational map $\xi:Y_{i}\adr Y$ is an isomorphism in a neighborhood of the strict transforms of $S$.
\end{thm}

\printref
\end{document}

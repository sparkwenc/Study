\documentclass[article, a4paper, twoside]{universal}

\setshowlvl{0}
\begin{document}
\confighead{}{}{}
\printhead{}{}{1}


\sct{Cyclotomic Iwasawa theory}

\begin{cmt}[0]

Ref:~\cite{Ribet1976Modular,Wiles1980,MW1984}, \cite{Wiles1990Iwasawa}.

Ref:~\cite{Washington1997}, \cite{Rubin1991}.


\begin{thm}[Herbrand--Ribet, {\cite[1.1]{Ribet1976Modular}}]
	Take a rational prime $p\gqs5$ and $K=\bQ(\zta_{p})$, let $2\lqs k\lqs p-3$ be even.
	Then $C=\Cl(\cO_{K})/\Cl(\cO_{K})^{p}$ admits an action of cyclotomic character $\chi$, one has $C=\op_{i}C^{(i)}$, and
	\[
		p\mid B_{k}\aolR C^{(1-k)}\neq0.
	\]
\end{thm}

\begin{prf}[Sketch of the proof]
	We only collect the idea of Ribet in the direction $\aoR$.

	By class field theory it suffices to prove the existence of a finite field $\bF\sps\bF_{p}$ and a Galois representation $\oL{\rho}:G_{\bQ}\ar\GL_{2}(\bF)$ such that (1) $\oL{\rho}$ is unramified away from $p$. (2) $\oL{\rho}$ is reducible with diagonal $1$ and $\chi^{k-1}$. (3) $\oL{\rho}$ is not diagonalizable. (4) $\oL{\rho}|_{G_{\bQ_{p}}}$ is diagonalizable. One prove this in two steps,

	\spg{Step 1} There is an explicit decomposition $M_{2}(\Gma_{0}(p),\eps)=S_{2}(\Gma_{0}(p),\eps)\op\bC s_{2,\eps}\op\bC G_{2,\eps}$, they are given by
	\[
		G_{2,\eps}=L(-1,\eps)/2+\sum_{n\gqs1}\sum_{d\mid n}\eps(d)dq^{n},\quad s_{2,\eps}=\sum_{n\gqs1}\sum_{d\mid n}\eps(n/d)dq^{n},
	\]
	the condition $p\mid B_{k}$ will result in a congruence between $G_{2,\eps}$ and a cuspform $f$ (\cite[3.4, 3.5]{Ribet1976Modular}).

	\spg{Step 2}
	There exists a normalized eigenform $f\in S_{2}(\Gma_{0}(p),\eps)$, where $\eps=\oga^{k-1}$, satisfying $a_{\ell}\eqv1+\ell^{k-1}\eqv1+\eps(\ell)\ell\bmod{\kp}$ for all $\ell\neq p$ and some $\kp\mid p$ in $K_{f}$ (\cite[3.7]{Ribet1976Modular}). One can prove the mod $\kp$ Galois representation attached to this $f$ is the desired representation (\cite[4.2, 4.3, 4.4]{Ribet1976Modular}).
\end{prf}


\end{cmt}

\ssc{Method of Eisenstein ideals}

Ref:~\cite{MW1984}.

\begin{stp}
    Let $p$ be an odd prime, $\chi:\Gal(\oL{\bQ}/\bQ)\atr\Gal(K_{\chi}/\bQ)\ar\cO_{\chi}\xT$ a non-trivial even primitive $p$-adic Dirichlet character of conductor not divisible by $p^{2}$ with field of definition $K_{\chi}$.
\end{stp}

\begin{stp}
	Let $\bQ\zF$ be the unique $\bZ_{p}$-extension of $\bQ$ and $\bQ_{n}$ the subfield of $\bQ\zF$ of degree $p^{n}$, let $F\sps K_{\chi}$ be an abelian extension of $\bQ$, one has $F\cap\bQ\zF=\bQ$, denote $F_{n}=F\bQ_{n}$, $F\zF=F\bQ\zF$ and $\Gal(F\zF/\bQ)\cong\Gal(F/\bQ)\tms\Gma$ the natural decomposition, fix a topological generator $\gma\in\Gma$, there is an isomorphism $\sgm_{\gma}:\Lda_{\chi}:=\cO_{\chi}\dS{\Gma}\ar\cO_{\chi}\dS{T}$ given by $\gma\amt1+T$.

	Let $L_{n}$ be the $p$-Hilbert class field of $F_{n}$, the reciprocity isomorphisms $C_{n}:=\Cl(F_{n})\cong H_{n}:=\Gal(L_{n}/F_{n})$ are compatible with the action by $\Gal(F/\bQ)$, both $C\zF:=\ilim_{n}C_{n}$ and $H\zF:=\plim_{n}H_{n}$ are $\bZ_{p}\dS{\Gma}$-modules with a commuting action by $\Gal(F/\bQ)$. $H_{\ift,\chi}:=H_{\ift}\ot_{\bZ_{p}[\Gal(F/\bQ)]}\cO_{\chi}$ is a torsion $\Lda_{\chi}$-module of finite type, it admits a morphism to $\Op_{i=1}^{r}\Lda_{\chi}/(h_{i})$ with finite kernel and cokernel, the ideal $\sgm_{\gma}(h_{1},\lds,h_{r})\sbs\cO_{\chi}\dS{T}$ admits a unique generator of the form $h_{p}(\chi,T)=\pi_{\chi}^{\mu_{\chi}}h_{\chi}(T)$ where $h_{\chi}(T)$ is a distinguished polynomial.

	Define $h_{p}(\chi,T)$, $\mu_{\chi}$ and $\lda_{\chi}:=\Deg(h_{\chi}(T))$ to be the \tbf{Iwasawa polynomial}, \tbf{$\mu$-invariant} and \tbf{$\lda$-invariant} of the $\Lda_{\chi}$-module $H_{\ift,\chi}$. The Iwasawa polynomial of $\Hom{}{}{(C_{\ift},\bQ_{p}/\bZ_{p})}_{\chi}$ is given by $h_{p}(\chi\xI,(1+T)\xI-1)$.
\end{stp}

\begin{stp}
	The \tbf{$p$-adic $L$-function $L_{p}(s,\chi)$} is the unique continuous function on $\bZ_{p}$ such that for all $k\gqs1$, let $\chi_{k}$ be the primitive character associated to $\chi\oga^{-k}$ and $f_{k}$ be the conductor of $\chi_{k}$, one has
	\[
		L_{p}(\chi,1-k)=-(1-\chi_{k}(p)p^{k-1})B_{k}(\chi_{k})/k,\quad B_{k}(\chi_{k})=(1/f_{k})\sum_{t=1, (t,f_{k})=1}^{f_{k}}B_{k}(\sA{t/N})\chi_{k}(t),
	\]
	moreover, there is a unique power series $G_{p}(\chi,T)\in\cO_{\chi}\dS{T}$ such that $G_{p}(\chi,u^{s}-1)=L_{p}(\chi,s)$ for all $s\in\bZ_{p}$. \red{(TODO: add more explicit description, ref Page 206)}
\end{stp}

\begin{thm}[{\cite[1.9~Theorem]{MW1984}}]\label{thm:main}
	As ideals in $\cO_{\chi}\dS{T}$, $(h_{p}(\oH{\chi},T))=(G_{p}(\chi,T))$.
\end{thm}

\begin{prf}[Sketch of the proof of \cref{thm:main}]
	By the analytic class number formula, it suffices to prove one-side inclusion, as for the method of Eisenstein ideals, one proves $h_{p}(\oH{\chi},T)\in(G_{p}(\chi,T))$.

	Indeed, let $\psi$ be the character $\chi\oga^{-2}$, it belongs to some pseudo-primitive component $\km$ of $R^{(1)}$. Let $\kX$ be the set of pseudo-primitive components satisfying one of the conditions (1) of level $N=p^{n}$ and basic character $\oga^{-2}$, (2) of level $N=p^{n}$ and basic character $\chi_{0}$, (3) of level $N=ap^{n}$ and $\oH{S}_{\km}'(ap)$ the unit ideal in $R_{\km}^{(1)}$.


	\begin{itm}
		\item If $\km\in\kX$, the right hand side will be a unit ideal.
		\item If $\km\nin\kX$, one proves the following inclusion, and then show the extra factor is either nonzero, or is zero in which case the desired equality is easy or known.
		\[
			(1-\psi(l)[l\xI])^{\kpa}\prod_{q_{i}\mid r, q_{i}\neq p}(1-\psi(n_{q_{i}})[q_{i}\xI])\cdot h_{p}(\oH{\chi},T)\sbs G_{p}(\chi,T).
		\]
	\end{itm}

	To prove the inclusion in the case $\km\nin\kX$, it suffices to construct an ideal $\kb_{\km}^{(n)}\sbs R_{\km}^{(n)}$ and an extension $L_{\km}^{(n)}/K_{\km}^{(n)}$ virtually unramified of type $\km$ such that
	\begin{enr}[label=(\arabic*)]
		\item if $n_{q_{i}}\eqv1\bmod{q_{i}}$ and $n_{q_{i}}\eqv q_{i}\bmod{(N/q_{i})}$, then $(1-l[l])^{\kpa}\prod_{q_{i}\neq p, q_{i}\mid r}(q_{i}[n_{q_{i}}]-1)\kb_{\km}^{(n)}\sbs\oH{S}_{\km}'(N)$.
		\item let $\Phi_{\km}^{(n)}$ be the $R_{\km}^{(\ift)}$-fitting ideal of $\Gal(L_{\km}^{(n)}/K_{\km}^{(n)})$, there exists an ideal $\kA_{\km}\sbs R_{\km}^{(\ift)}$ of finite index independent of $n$, such that $\kA_{\km}\Phi_{\km}^{(n)}\sbs\kb_{\km}^{(n)}$ in $R_{\km}^{(\ift)}$ for all $n\gqs1$.
	\end{enr}

	For any pseudo-primitive component $\km\nin\kX$, let $\psi=\psi^{p}\psi_{p}$ be its basic character, where $\psi_{p}=\oga^{k}, 0\lqs k<p-1$, say $\km$ is \tbf{in case (1)} if either $\psi^{p}(p)\neq1$ or $k\neq-1$, and is \tbf{in case (2)} otherwise.

	The construction of $\kb_{\km}^{(n)}\sbs R_{\km}^{(n)}$ and $L_{\km}^{(n)}/K_{\km}^{(n)}$ in case (1) is sketched below.
\end{prf}

% \begin{cmt}
% \begin{dfn}
% 	Let $\km$ be a component, $F_{\km}\xP$ the finite abelian extension of $\bQ$ cut out by the Dirichlet characters in $\km$ of conductor dividing $ap$, and let $F_{\km}=F_{\km}\xP(\zta_{p})$. Consider the field tower
% 	\[
% 		\begin{tikzcd}
% 			\bQ\ar[r, "\bZ_{p}", hook]\ar[d, hook] & \bQ\zF\ar[d, hook] && \\
% 			F_{\km}\ar[r, hook] & K_{\km}\ar[r, "H_{\km}", hook] & L_{\km}\ar[r, "I_{\km}", hook] & L_{\km}\xR
% 		\end{tikzcd}
% 	\]
% 	where $L_{\km}$ is the maximal unramified abelian extension of $K_{\km}$ of type $\km$, and $L_{\km}\xR$ is the maximal virtually unramified extension of $K_{\km}$ of type $\km$.
% \end{dfn}
% \end{cmt}

\begin{cmt}
\begin{stp}
	Let $a$ be an integer prime to $p$. Denote $\bZ_{p,a}:=\plim_{n}\bZ/p^{n}a\bZ$ as a compact topological group, then $\bZ_{p,a}\xT\cong\sR{\bZ/pa\bZ}\xT\tms\Oga$. Denote by $R^{(\ift)}=\plim_{n}R^{(n)}$ where $R^{(n)}:=\bZ_{p}[\sR{\bZ/ap^{n}\bZ}\xT]$, then $R^{(\ift)}\cong R^{(1)}\dS{\Oga}$.
	\begin{thm}[{\cite[Page~195]{MW1984}}]
		Let $p$ be a prime, $G=G_{p}\tms G^{p}$ be a finite abelian group of order $n$. Then $R=\bZ_{p}[G]$ is a complete semi-local ring, one has $R=\prod_{\km\in\Pi}R_{\km}$, where $\Pi$ is the \tbf{components} of $R$,
		\begin{itm}
			\item connected components of $\Spc{R}$.
			\item $\bQ_{p}$-conjugacy classes of $\oL{\bQ}_{p}\xT$-valued characters of $G^{p}$.
		\end{itm}
		Let $\oT{R}$ be the normalization of $R$, one has $\oT{R}=\prod_{\sgm\in\Sgm}R_{\sgm}$, where $\Sgm$ is the \tbf{sheets} of $R$,
		\begin{itm}
			\item irreducible components of $\Spc{R}$.
			\item $\bQ_{p}$-conjugacy classes of $\oL{\bQ}_{p}\xT$-valued characters of $G$.
		\end{itm}
		The surjection $\Sgm\atr\Pi$ gives an isomorphism $R_{\km}\ot\bQ_{p}\cong\prod_{\sgm\in\km}R_{\sgm}\ot\bQ_{p}$, each $R_{\sgm}\ot\bQ_{p}$ is a field.
	\end{thm}
\end{stp}
\end{cmt}


\begin{cmt}
	\begin{dfn}
		For a component $\km$ of $R^{(1)}$, say it is \tbf{$a$-primitive} if its reduced conductor is $a$ or $ap$; say it is \tbf{pseudo-primitive} if there is a character belonging to $\km$ with conductor $a$ or $ap$.

		If $\km$ is pseudo-primitive of reduced conductor $ap/r$, then (1) $r$ is square-free, (2) every prime $q|r$ is either $p$ or congruent to $1$ mod $p$, (3) $(r,ap/r)=1$.
	\end{dfn}

\end{cmt}

\begin{cmt}
	Let $\rho:\bZ_{p}\dS{\Gal(\oL{\bQ}/\bQ)}\ar R^{(\ift)}$ be the unique continuous ring homomorphism extending the cyclotomic character $\Gal(\oL{\bQ}/\bQ)\ar\bZ_{p,a}\xT$, $\oL{\rho}$ its conjugate, and $\rho(k)$ the Tate twists by $\Gal(\oL{\bQ}/\bQ)\ar\bZ_{p}\xT$.

	\begin{dfn}
		Let $\km$ be a component of $R^{(1)}$, and $\eta:\bZ_{p}\dS{\Gal(\oL{\bQ}/\bQ)}\ar R_{\km}^{(\ift)}$ a continuous morphism.

		Say a $R_{\km}^{(\ift)}$-module $M$ is \tbf{$\eta$-yoked} if it admits a commuting $\Gal(\oL{\bQ}/\bQ)$ action compatible with $\eta$.
	\end{dfn}

	\begin{dfn}[1.8~Definition]
		Let $K/\bQ$ be an abelian Galois extension, $L/K$ a $p$-abelian Galois extension such that $L/\bQ$ is Galois. Say $L/K$ is \tbf{of type $\km$} if $\Gal(L/K)$ is $\oL{\rho}(-1)$-yoked, and is \tbf{virtually unramified of type $\km$} if it is moreover unramified outside primes dividing $a$.
	\end{dfn}
\end{cmt}


\begin{cmt}

\begin{dfn}
	Given $k\gqs1,N\gqs0,b$ integers, consider the Stickelberger elements
	\[
		\vtt_{k}(b;N):=(N^{k-1}/k)\sum_{t=1,(t,N)=1}^{N}B_{k}(\sA{bt/N})[t]^{-1}\in\bQ[(\bZ/N\bZ)\xT].
	\]
	The \tbf{$k$-th Stickelberger ideals} are given by
	\[
		S_{k}(N):=\bZ[(\bZ/N\bZ)\xT]\cap\sum_{b\in\bZ}\vtt_{k}(b;N)\bZ[(\bZ/N\bZ)\xT],\quad S_{k}'(N):=\bZ[(\bZ/N\bZ)\xT]\cap\sum_{b\in\bZ, (b,p)=1}\vtt_{k}(b;N)\bZ[(\bZ/N\bZ)\xT].
	\]
\end{dfn}
\end{cmt}

\begin{cmt}
\begin{dfn}[3.7~Proposition~1, Remark]
	Fix a prime $p$ and an integer $a$ prime to $p$, $F$ a number field and $S=\Spc\cO_{F}[a\xI]$. If $\Gma\ar S$ is a $p$-divisible group such that $\Gma_{F}$ is the subquotient of the $p$-divisible group associated to an abelian variety over $F$ with potentially good reduction above $p$, then the ind-scheme $\mu(\Gma)$ generated by all multiplicative-type subgroup schemes in $\Gma$ is a finite flat multiplicative-type group scheme.

	In such case, the $p$-divisible group $\Gma':=\Gma/\mu(\Gma)$ is said to be the canonical \tbf{$\mu$-deprived quotient} of $\Gma$. If $\Gma=\Gma'$, say that $\Gma$ is \tbf{$\mu$-deprived}.
\end{dfn}
Let $A_{n,S}$ be the N{\'e}ron model of $A_{n}$, $\km$ an even pseudo-primitive component, $P$ a maximal ideal of $\bT_{\km}$ containing $U_{p}-1$, $A_{n,P}$ the associated $P$-divisible group, let $B_{n}:=A_{n}/\mu(A_{n,P})\ar S$, which is called the canonical quotient of $A$ which is \tbf{$\mu$-deprived at $P$}. \red{(Thus $B$ depends on a choice of $\km$, HOW???)}
\end{cmt}

\begin{cmt}

\begin{stp}
	\red{Chapter5's case(1) set}

	Let $\km$ be a pseudo-primitive character, $\psi=\psi^{p}\oga^{k}$ be the basic character associated to $\km$, where $\psi^{p}$ has conductor prime to $p$, $\oga$ the Teichm{\"u}ller character, $0\lqs k<p-1$, say that $\km$ is in case (1) if $\Gcd(k,p-1)>1$ or $\psi^{p}(p)\neq1$ or $k\neq-1$ is odd.
\end{stp}

\end{cmt}

\begin{stp}
	Fix $p$ a prime and $n\gqs1$, let $N=ap^{n}$ with $(a,p)=1$. For $k\gqs1$, let $X_{1}(ap^{k};ap^{k-1})\ar\Spc\bQ$ be the modular curve associated to $\Gma_{0}(ap^{k})\cap\Gma_{1}(ap^{k-1})$, the map $X_{1}(ap^{k})\axr{\pi_{k}}X_{1}(ap^{k};ap^{k-1})\axr{\rho_{k}}X_{1}(ap^{k-1})$ induces the map $J_{1}(ap^{k})\axl{\pi_{k}\xS}J_{1}(ap^{k};ap^{k-1})\axr{\rho_{k,*}}J_{1}(ap^{k-1})$, let $K_{1}$ be the maximal abelian subvariety of $J_{1}(ap;a)$ with multiplicative reduction at $p$, one can then inductively define
	\[
		\begin{tikzcd}
			\cds & A_{2} && A_{1} && \\
			\cds & J_{1}(ap^{2})\ar[u, "\afa_{2}", two heads] & J_{1}(ap^{2};ap)\ar[l, "\pi_{2}\xS"']\ar[r, "\rho_{2,*}"]\ar[ru, "\afa_{1}\cc\rho_{2,*}"] & J_{1}(ap)\ar[u, "\afa_{1}", two heads] & J_{1}(ap; a)\ar[l, "\pi_{1}\xS"']\ar[r, "\rho_{1,*}"] & J_{1}(a) \\
			\cds & K_{2}\ar[ru, hook]\ar[u, "\pi_{2}\xS|_{K_{2}}"] && K_{1}\ar[ru, hook]\ar[u, "\pi_{1}\xS|_{K_{1}}"] &&
		\end{tikzcd}
	\]
	where $A_{k}:=\Cok(\pi_{k}\xS|_{K_{k}}:K_{k}\ar J_{1}(ap^{k}))$ and $K_{k+1}:=\Ker(\afa_{k}\cc\rho_{k+1,*}:J_{1}(ap^{k+1},ap^{k})\ar A_{k})$. The Hecke algebra $\bT^{(n)}\sbs\End{\bQ}{}{(A_{n})}$ is generated over $\bZ$ by $T_{l}$ for $l\nmid N$, $U_{q,*}$ for $q\mid N$ and $\sA{r}$ for $r\in(\bZ/N\bZ)\xT/\{\pm1\}$, $\bT^{(n)}$ admits a $R^{(n)}$-module structure via $R^{(n)}\ar\bT^{(n)}, [r]\amt\sA{r}$, let $\bT^{(n)}_{p}=\bT^{(n)}\ot\bZ_{p}$.

\end{stp}

\begin{dfn}[{\cite[4.3~Definition, 5.1]{MW1984}}]
	Let $\km$ be a pseudo-primitive component with basic character $\chi$ having conductor $ap/r$. Define $\bT^{(n)}_{\km}:=\bT^{(n)}_{p}\ot R_{\km}^{(n)}$, $\bT_{\km}^{(n)}$ admits a \tbf{maximal Eisenstein ideal} $P_{n}$ generated by $T_{l}-1-l\sA{l}$, $U_{q}$ for $q\mid N$, $\sA{r}-[r]_{\km}$ and $\Img(\km R_{\km}^{(n)}\ar\bT_{\km}^{(n)})$, one has $R_{\km}^{(n)}/\km R_{\km}^{(n)}\cong \bT_{\km}^{(n)}/P_{n}$; let $B_{n}$ be the canonical quotient of $A_{n}$ which is $\mu$-deprived at $P_{n}$; let $Z_{n}$ be the $\Gma_{1}(N)$-orbit in $\bP^{1}(\bQ)$ of the elements $\pm(\afa,\bta)$ where $(\bta,N)=1$, define $\eps_{r}:=U_{p}^{n-1}\prod_{q\mid pr}(U_{q}-q\sA{n_{q}})$, then the \tbf{stabilized cuspidal group} is defined
	\[
		C_{\km,r}^{(n)}:=\Img(\bZ_{p}[Z_{n}]\ahr\Div(X_{1}(ap^{n}))\ot\bZ_{p}\axr{\eps_{r}}\Div^{0}(X_{1}(ap^{n}))\ot\bZ_{p}\ar J_{1}(ap^{n})_{\km}\atr A_{n,\km}\atr B_{n,\km}).
	\]
	\begin{itm}
		\item Define the \tbf{basic ideal} $\kb_{\km}^{(n)}\sbs R_{\km}^{(n)}$ to be the annihilator ideal of $C_{\km,r}^{(n)}$.
		\item Define the \tbf{stabilized Eisenstein ideal} $I_{\km}^{(n)}\sbs\bT_{\km}^{(n)}$ to be the annihilator ideal of $C_{\km,r}^{(n)}$.
	\end{itm}
	The natural map $R_{\km}^{(n)}/\kb_{\km}^{(n)}\ar\bT_{\km}^{(n)}/I_{\km}^{(n)}$ is an isomorphism. If $C_{\km,r}^{(n)}$ is not trivial, $I_{\km}^{(n)}$ is primary to $P_{n}$, indeed, if $\oH{S}_{\km}'(ap)$ is not the unit, then $C_{\km,r}^{(n)}$ is non-trivial for sufficiently large $n$.
\end{dfn}

\begin{thm}[{\cite[4.3~Theorem]{MW1984}}]
	The basic ideal $\kb_{\km}^{(n)}\sbs R_{\km}^{(n)}$ satisfies the following,
	\begin{itm}
		\item $\prod_{q\mid r, q\neq p}(q[n_{q}]_{\km}-1)\cdot \kb_{\km}^{(n)}\sbs\oH{S}_{\km}'(ap^{n})$.
		\item $(R_{\km}^{(n)}/\kb_{\km}^{(n)})[\km R_{\km}^{(n)}]$ has dimension $\lqs1$ over $R_{\km}^{(n)}/\kb_{\km}^{(n)}$, the ring $R_{\km}^{(n)}/\kb_{\km}^{(n)}$ is Gorenstein.
	\end{itm}
\end{thm}

\begin{dfn}[{\cite[5.2~Lemma~4]{MW1984}}]
	There is an exact sequence of finite flat group schemes over $S=\Spc\cO_{F}[a\xI]$, $0\ar C_{\km,r}^{(n)}\ar B_{n}[I_{\km}^{(n)}] \ar M_{\km}^{(n)}\ar 0$, where $M_{\km}^{(n)}$ is of multiplicative type.

	Let $K_{\km}^{(n)}$ be the minimal extension of $\bQ$ which contains $\bQ(\zta_{p\xF})$ and is a splitting field for all $\chi$ of conductor $ap^{t}$ belonging to $\km$, and let $L_{\km}^{(n)}$ be the minimal extension of $K_{\km}^{(n)}$ which is a splitting field of $B_{n}[I_{\km}^{(n)}]$.
\end{dfn}


\begin{thm}
	The extension $L_{\km}^{(n)}/K_{\km}^{(n)}$ satisfies the following,
	\begin{itm}
		\item (\cite[5.4~Proposition~1]{MW1984}) $L_{m}^{(n)}/K_{\km}^{(n)}$ is virtually unramified of type $\km$.
		\item (\cite[5.5~Proposition~1]{MW1984}) The Fitting ideal of the $R_{\km}^{(n)}$-module $\Gal(L_{\km}^{(n)}/K_{\km}^{(n)})$ is contained in $\kb_{\km}^{(n)}$.
	\end{itm}
\end{thm}

% \begin{thm}[1.6~Proposition~3]
	% $\afa_{\km,\chi}(\oH{S}_{\km}'(ap^{\ift}))=G_{p,2}(\chi\xI,(1+T)\xI-1)$.
% \end{thm}


\newpage
\ssc{Method of Euler systems}
Ref:~\cite{MR2004Kolyvagin}.

\begin{stp}
    Fix a prime $p$, let $R$ be a complete Noetherian local ring with maximal ideal $\km$ and finite residue field $k=R/\km$ of characteristic $p$. Let $T$ be a free $R$-module of finite rank with a continuous action of $\Gal_{\bQ}$ unramified outside a finite set of primes.
\end{stp}

\begin{dfn}[1.1.6]
    Let $\ell\neq p,\ift$ be an unramified prime of $T$, associated to $\HH{}{1}{(\bQ_{\ell},T)}$, define the \tbf{finite part} as $\HH{\T{f}}{1}{(\bQ_{\ell},T)}:=\Ker(\HH{}{1}{(\bQ_{\ell},T)}\ar\HH{}{1}{(\bQ_{\ell}^{\T{ur}},T)})$, the \tbf{singular quotient} as $\HH{\T{s}}{1}{(\bQ_{\ell},T)}:=\HH{}{1}{(\bQ_{\ell},T)}/\HH{\T{f}}{1}{(\bQ_{\ell},T)}$, and the \tbf{transverse} condition $\HH{\T{tr}}{1}{(\bQ_{\ell},T)}:=\Ker(\HH{}{1}{(\bQ_{\ell},T)}\ar\HH{}{1}{(\bQ_{\ell}(\mu_{\ell}),T)})$.

    % indeed, one has $\HH{}{1}{(\bQ_{\ell},T)}=\HH{\T{f}}{1}{(\bQ_{\ell},T)}\op\HH{\T{tr}}{1}{(\bQ_{\ell},T)}$.
\end{dfn}

\begin{thm}[1.2.1, 1.2.2, 1.2.4]
    Suppose that $(\ell-1)T=0$, then
    \begin{itm}
        \item $\HH{\T{f}}{1}{(\bQ_{\ell},T)}\cong T/(\Frb-\Id)T,\quad \HH{}{1}{(\bQ_{\ell},T)}\cong\HH{\T{f}}{1}{(\bQ_{\ell},T)}\op\HH{\T{tr}}{1}{(\bQ_{\ell},T)}$.
        \item $\HH{\T{s}}{1}{(\bQ_{\ell},T)}\cong\Hom{}{}{(I_{\bQ_{\ell}},T^{\Frb=\Id})},\quad \HH{\T{s}}{1}{(\bQ_{\ell},T)}\ot\bF_{\ell}\xT=T^{\Frb=\Id}$.
        \item there is the \tbf{finite-singular comparison map} $\phi_{\ell}^{\T{fs}}:\HH{\T{f}}{1}{(\bQ_{\ell},T)}\ar\HH{\T{s}}{1}{(\bQ_{\ell},T)}\ot\bF_{\ell}\xT$.
    \end{itm}
\end{thm}


\begin{dfn}[2.1.1, 2.3.1]
    A \tbf{Selmer structure} $\cF$ on $T$ is the data of
    \begin{itm}
        \item a finite set $\Sgm(\cF)$ of places of $\bQ$ containing $p,\ift$ and primes where $T$ is ramified.
        \item for every $\ell\in\Sgm(\cF)$, a choice of $R$-submodule $\HH{\cF}{1}{(\bQ_{\ell},T)}\sbs\HH{}{1}{(\bQ_{\ell},T)}$.
        \item for every $\ell\nin\Sgm(\cF)$, let $\HH{\cF}{1}{(\bQ_{\ell},T)}:=\HH{\T{f}}{1}{(\bQ_{\ell},T)}$.
    \end{itm}
    The \tbf{Selmer module} $\HH{\cF}{1}{(\bQ,T)}\sbs\HH{}{1}{(\bQ,T)}$ is defined by
    \[
        \HH{\cF}{1}{(\bQ,T)}:=\Ker(\HH{}{1}{(\bQ_{\Sgm(\cF)}/\bQ,T)}\ar\op_{\ell\in\Sgm(\cF)}\HH{}{1}(\bQ_{\ell},T)/\HH{\cF}{1}{(\bQ_{\ell},T)}).
    \]
\end{dfn}

\begin{rmk}
    The dual Selmer structure $\cF\xS$ on $T\xS:=\Hom{}{}{(T,\mu_{p\xF})}$ is given by $\Sgm(\cF\xS)=\Sgm(\cF)$, and $\HH{\cF\xS}{1}{(\bQ_{\ell},T)}$ is the orthogonal complement of $\HH{\cF}{1}{(\bQ_{\ell},T)}$ under the local Tate pairing
    \[
        \sA{-,-}_{\ell}:\HH{}{1}{(\bQ_{\ell},T)}\tms\HH{}{1}{(\bQ_{\ell},T\xS)}\ar\bQ_{p}/\bZ_{p}.
    \]
    The Selmer structures on $T$ admits a partial ordering, $\cF\lqs \cG$ if and only if $\HH{\cF}{1}{(\bQ_{\ell},T)}\sbs\HH{\cG}{1}{(\bQ_{\ell},T)}$ for all $\ell$.

    (2.1.8) Let $\cF$ be a Selmer structure, $a,b,c\in\bZ_{>0}$ relatively prime integers such that $c$ is not divided by any primes in $\Sgm(\cF)$, the modified Selmer structure $\cF_{a}^{b}(c)$ is given by $\Sgm(\cF_{a}^{b}(c))=\Sgm(\cF)\cup\{\ell:\ell\mid abc\}$ and
    \[
        \HH{\cF_{a}^{b}(c)}{1}{(\bQ_{\ell},T)}=\begin{cases}
          0 & \ell \mid a \\
          \HH{}{1}{(\bQ_{\ell},T)} & \ell\mid b \\
          \HH{\T{tr}}{1}{(\bQ_{\ell},T)} & \ell\mid c
        \end{cases}
    \]
    one has for every $n$, $\cF_{n}\lqs\cF\lqs\cF^{n}$ and $\cF_{n}\lqs\cF(n)\lqs\cF^{n}$.
\end{rmk}

\begin{thm}[2.1.5]
    $\HH{\cF}{1}{(\bQ,T)}$ is a finitely generated $R$-module.

    If $R$ is an integral domain, $T$ is torsion-free, and $(T/\km T)^{\Gal_{\bQ}}=0$, then $\HH{\cF}{1}{(\bQ,T)}$ is torsion-free.
\end{thm}

\begin{thm}[2.3.4]

    Let $\cF,\cG$ be Selmer structures on $T$ such that $\cF\lqs\cG$, there are exact sequences
    \[
        \begin{tikzcd}[row sep=small]
            0\ar[r] & \HH{\cF}{1}{(\bQ,T)}\ar[r] & \HH{\cG}{1}{(\bQ,T)}\ar[r, "\loc_{\cF}^{\cG}"] & \Op_{\ell}\HH{\cG}{1}{(\bQ_{\ell},T)}/\HH{\cF}{1}{(\bQ_{\ell},T)} \\
            0\ar[r] & \HH{\cG\xS}{1}{(\bQ,T\xS)}\ar[r] & \HH{\cF\xS}{1}{(\bQ,T\xS)}\ar[r, "\loc_{\cG\xS}^{\cF\xS}"] & \Op_{\ell}\HH{\cF\xS}{1}{(\bQ_{\ell},T\xS)}/\HH{\cG\xS}{1}{(\bQ_{\ell},T\xS)}
        \end{tikzcd}
    \]
    satisfying that $\Img(\loc_{\cF}^{\cG})$ and $\Img(\loc_{\cG\xS}^{\cF\xS})$ are orthogonal complements with respect to $\sum_{\ell}\sA{-,-}_{\ell}$.
\end{thm}

\begin{stp}
    Let $\cP$ be a set of primes disjoint from $\Sgm(\cF)$, and $\cN$ the set of square-free products of primes in $\cP$, for an unramified prime $\ell\neq p,\ift$, let $P_{\ell}(x)=\Det(\Id-\Frb_{\ell}x)\in R[x]$, and $I_{\ell}\sbs R$ the ideal generated by $\ell-1$ and $P_{\ell}(1)$. For $n\in\cN$ write $I_{n}:=\sum_{\ell\mid n}I_{\ell}$ and $G_{n}:=\ot_{\ell\mid n}\bF_{\ell}\xT$, indeed, $G_{n}\ot(R/I_{n})$ is free of rank one over $R/I_{n}$.
\end{stp}

\begin{dfn}[3.1.2, 3.1.3, 3.1.5, 3.1.6]
    A \tbf{Selmer sheaf} $\cH$ associated to $(T,\cF,\cP)$ is the data of
    \begin{itm}
        \item $\cH(n):=\HH{\cF(n)}{1}{(\bQ,T/I_{n}T)}\ot G_{n}$ for each $n\in\cN$.
        \item $\cH(n,n\ell):=\HH{\T{s}}{1}{(\bQ_{\ell},T/I_{n\ell}T)}\ot G_{n\ell}$ for each $n\in\cN$ and some prime $\ell$ such that $n\ell\in\cN$, and
        \[
            \begin{tikzcd}[row sep=small]
                \psi_{0}^{n,n\ell}:\cH(n)\ar[r, "\loc_{\ell}"] & \HH{\T{f}}{1}{(\bQ_{\ell},T/I_{n\ell}T)}\ot G_{n}\ar[r, "\phi_{\ell}^{\T{fs}}\ot\Id"] & \cH(n,n\ell) \\
                \psi_{1}^{n,n\ell}:\cH(n\ell)\ar[r, "\loc_{\ell}"] & \HH{\T{tr}}{1}{(\bQ_{\ell},T/I_{n\ell}T)}\ot G_{n\ell}\ar[r] & \cH(n,n\ell)
            \end{tikzcd}
        \]
    \end{itm}
    A \tbf{Kolyvagin system} $\kpa$ for $(T,\cF,\cP)$ is the data of classes $\{\kpa_{n}\in\cH(n)\}_{n\in\cN}$ such that $\psi_{1}^{n,n\ell}(\kpa_{n\ell})=\psi_{0}^{n,n\ell}(\kpa_{n})$, denote by $\KS(T,\cF,\cP)$ the $R$-module of Kolyvagin systems. The \tbf{order of vanishing} of a nonzero $\kpa$ is $\Ord(\kpa):=\min\{\nu(n):\kpa_{n}\neq0\}$, the \tbf{module of $L$-values} of $T$ is $\cL(T,\cF,\cP):=\{\kpa_{1}:\kpa\in\KS(T,\cF,\cP)\}\sbs\HH{\cF}{1}{(\bQ,T)}$.

    \red{\TODO: 5.1.1, 5.2.12, 5.2.14, 5.3.10}


    For $j\in\bZ_{>0}$, let $\cP_{j}$ be the set of primes $\ell\nin\Sgm(\cF)$ such that $T/(\km^{k}T+(\Frb_{\ell}-\Id)T)$ is free of rank $1$ over $R/\km^{k}$ and $I_{\ell}\sbs\km^{k}$, define
    \[
        \oL{\KS}(T,\cF,\cP):=\plim_{k}\ilim_{j}\KS(T/\km^{k}T,\cP\cap\cP_{j}),
    \]
    the \tbf{blind spot} of a Kolyvagin system $\kpa$ is the set of ideals $I\sbs R$ such that the image of $\kpa$ under the composition $\KS(T)\ar\KS(T/I)\ar\oL{\KS}(T/I)$ is zero.

    \red{\TODO:5.2.9}

    % $\kpa$ is called \tbf{primitive} if $R\kpa_{n}=\cH(n)$ for all $n\in\cN$.
\end{dfn}

\begin{dfn}[3.3.1]
    Suppose for each edge $e_{n,n\ell}$ there is an isomorphism $\cH(n,n\ell)\cong\cH(1,\ell)$, for each $n\in\cN$, define $\psi^{n}:\cH(n)\ar\op_{\ell\mid n}\cH(1,\ell)$ as $\psi^{n}=\op_{\ell\mid n}\psi_{1}^{n/\ell,n}$, for a Kolyvagin system $\kpa$ define the \tbf{Kolyvagin-constructed dual Selmer group} $\Sel\xS(\kpa):=\ilim_{n\in\cN}\Sel\xS(\kpa;n)$, where
    \[
        \Sel\xS(\kpa;n):=(\Op_{\ell\mid n}\cH(1,\ell))/(\sum_{d\mid n}\psi^{d}(R\kpa_{d})).
    \]
\end{dfn}

\begin{thm}[3.3.2, 3.3.3]
    Suppose $I_{\ell}=0$ for every $\ell\in\cP$, and fix a generator of $G_{\ell}$ for every $\ell\in\cP$, one has isomorphisms $\cH(n,n\ell)\cong\cH(1,\ell)\cong\HH{\T{s}}{1}{(\bQ_{\ell},T)}$. For every $\kpa\in\KS(T,\cF,\cP)$ there is an exact sequence
    \[
        0\ar\bcap_{n\in\cN}\HH{\cF_{n}\xS}{1}{(\bQ,T\xS)}\ar\HH{\cF\xS}{1}{(\bQ,T\xS)}\ar\Hom{}{}{(\Sel\xS(\kpa),\bQ_{p}/\bZ_{p})}.
    \]
\end{thm}

\begin{stp}
    Suppose now $R$ is principal and Artinian, \red{together with a triple $(T,\cF,\cP)$ satisfying (H1)-(H6)}.
\end{stp}

\begin{dfn}[4.1.2, 4.1.8, 4.10, 4.11]
    For every $n\in\cN$, define $\lda(n,T):=\Len_{R}(\cH(n))$, say that $n\in\cN$ is a \tbf{core vertex} if $\lda(n,T)=0$ or $\lda(n,T\xS)=0$. Define the \tbf{core Selmer rank} of $T$ to be $\chi(T):=\Rnk_{R}(\cH(n))$ for any core vertex $n$, which is independent of the choice of $n$.
\end{dfn}

\begin{thm}[4.2]
    The core Selmer rank controls the existence of Kolyvagin systems,
    \begin{itm}
        \item if $\chi(T)=0$, then $\KS(T)=0$.
        \item if $\chi(T)=1$, then $\KS(T)$ is free of rank one over $R$.
        \item if $\chi(T)>1$, then $\KS(T)$ contains a free $R$-module of rank $r$ for every $r$.
    \end{itm}
\end{thm}

\begin{stp}
    Suppose now $R$ is the Iwasawa algebra $\Lda:=\bZ_{p}\dS{\Gal(\bQ_{\ift}/\bQ)}$, fix a finite set of primes $\Sgm$ containing $p,\ift$ and ramified primes of $T$, \red{assume $T$ satisfies (H1)-(H4)}.
\end{stp}

\begin{thm}[5.3.2, 5.3.5, 5.3.10, 6.1.9]
    Let $\cF_{\Lda}$ be a Selmer structure defined on $T$ such that $\Sgm(\cF_{\Sgm})=\Sgm$, one has $\HH{\cF_{\Lda}}{1}{(\bQ_{\ell},T)}=\HH{}{1}{(\bQ_{\ell},T)}$ for each $\ell$, and $\HH{\cF_{\Lda}}{1}{(\bQ,T)}=\HH{}{1}{(\bQ,T)}$, indeed, $\HH{}{1}{(\bQ,T)}$ is finitely generated and torsion-free.

    For $\kpa\in\KS(T)$ or $\kpa\in\oL{\KS}(T)$, $\Chr(\Hom{}{}{(\HH{\cF_{\Lda}\xS}{1}{(\bQ,T\xS)},\bQ_{p}/\bZ_{p})})$ divides $\Chr((\HH{}{1}{(\bQ,T)}/\Lda\kpa_{1})_{\T{tor}})$. In particular, for a finite order character $\chi:\Gal_{\bQ}\ar\bZ_{p}\xT$ with $p>2$ and fixed field $L$, let $L_{n}:=\bQ_{n}L$, $\cC_{n}$ the cyclotomic units of $L_{n}$, $\cC_{\ift},\cE_{\ift}$ the inverse limits of the $p$-adic completions of $\cC_{n}$ and $\cO_{L_{n}}\xT$, then
    \[
        \Chr(\plim_{n}(\Cl(L_{n})[p\xF])^{\chi})=\Chr(\cE_{\ift}^{\chi}/\cC_{\ift}^{\chi}).
    \]
\end{thm}


\printref
\end{document}


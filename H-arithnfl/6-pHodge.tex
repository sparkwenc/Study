\documentclass[article, a4paper, twoside]{universal}

\setshowlvl{1}
\begin{document}
\confighead{}{}{}
\printhead{}{}{1}

\sct{$p$-adic geometry}

\ssc{Integral $p$-adic Hodge theory}
Ref:~\cite{BMS2018}.

\begin{stp}
	Fix a prime number $p$, let $S$ be a commutative ring which is $\vpi$-adically complete and separated for some $\vpi\in S$ dividing $p$, $\vfi:S/pS\ar S/pS$ its absolute Frobenius and the perfect $\bF_{p}$ algebra $S\xL$ its tilt. One has Fontaine's ring $\bA_{\text{inf}}(S):=W(S\xL)$ equipped with Frobenius $\vfi$ and $\tta_{r}:\bA_{\text{inf}}(S)\ar W_{r}(S)$.

\end{stp}

\begin{dfn}
	$S$ is a \tbf{perfectoid ring} if (1) $\vpi^{p}\mid p$, (2) $\vfi$ is surjective and (3) $\Ker(\tta_{1})$ is principal.
\end{dfn}

\begin{thm}[{\cite[Lemma~3.12, Lemma~3.14]{BMS2018}}]
	If $S$ is perfectoid and $\xi=(\xi_{0},\xi_{1},\lds)\in\Ker(\tta_{1})$ is any distinguished element, then for any $r\gqs1$, $\Ker(\tta_{r})$ is generated by the non-zero divisor $\xi\vfi^{-1}(\xi)\cds\vfi^{-(r-1)}(\xi)$.

	If $S\ar R$ is any map of perfectoid rings, then $\bL_{R/S}\olt_{\bZ}\bF_{p}\cong0$.
\end{thm}

\begin{thm}[{\cite[Lemma~3.23, Proposition~3.24]{BMS2018}}]
	Let $C$ be a perfectoid field of characteristic $0$ containing all $p$-power roots of unity, denote $A_{\text{inf}}:=\bA_{\text{inf}}(\cO_{C})$ and $\vep:=(1,\zta_{p},\zta_{p^{2}},\lds)\in\cO_{C}\xL$. Then $W_{r}(\cO_{C})$ are coherent rings for $r\gqs1$, $\Ker(\tta_{r}:A_{\text{inf}}\ar W_{r}(\cO_{C}))$ and $\Ker(\tta_{\ift}:A_{\text{inf}}\ar W(\cO_{C}))$ are generated by
	\[
		\xi_{r}=\frac{[\vep]-1}{\vfi^{-r}([\vep]-1)},\quad\xi_{\ift}=[\vep]-1.
	\]
	One has Fontaine's period rings,
	\[
		A_{\text{crys}}=\oH{A_{\text{inf}}\sS{\frac{\xi_{1}^{m}}{m!}}}_{m\gqs0},\quad B_{\text{crys}}\xP=A_{\text{crys}}\sS{\frac{1}{p}},\quad B_{\text{crys}}=B_{\text{crys}}\sS{\frac{1}{[\vep]-1}},\quad B_{\text{dR}}\xP=\oH{B_{\text{crys}}\xP},\quad B_{\text{dR}}=B_{\text{dR}}\xP\sS{\frac{1}{\xi_{1}}}.
	\]
\end{thm}


\begin{stp}
	Let $K$ be a complete discretely valued extension of $\bQ_{p}$ with perfect residue field $k$ and a uniformizer $\vpi$. The ring $\kS:=W(k)\dS{T}$ is equipped with Frobenius $\vfi$, denote $\oT{\tta}:\kS\atr\cO_{K}$ the natural surjection with kernel generated by the Eisenstein polynomial $E(T)$ of $\vpi$. The commutative diagram is equivariant under $G_{K_{\ift}}$,
	\[
		\begin{tikzcd}
			\kS\ar[r, "\oT{\tta}"]\ar[d] & \cO_{K}\ar[d] \\
			A_{\text{inf}}\ar[r, "\oT{\tta}"] & \cO_{C}
		\end{tikzcd}
	\]
\end{stp}

\begin{dfn}
	A \tbf{Breuil-Kisin module} $(M,\vfi_{M})$ is a finitely generated $\kS$-module $M$ equipped with an isomorphism $\vfi_{M}:M\ot_{\kS,\vfi}\kS[1/E(T)]\cong M[1/E(T)]$.

\end{dfn}

\begin{thm}[{\cite[Proposition~4.3]{BMS2018}}]
	There is a canonical exact sequence of Breuil-Kisin modules attached to any Breuil-Kisin module $(M,\vfi_{M})$,
	\[
		0\ar(M_{\text{tor}},\vfi_{M_{\text{tor}}})\ar(M,\vfi_{M})\ar(M_{\text{free}},\vfi_{M_{\text{free}}})\ar(\oL{M},\vfi_{\oL{M}})\ar0,
	\]
	where $M_{\text{tor}}\sbs M$ is the torsion submodule, killed by a power of $p$; $M_{\text{free}}$ is a finite free $\kS$-module; $\oL{M}$ is a torsion $\kS$-module, killed by a power of $(p,T)$. In particular, $M[1/p]\cong M_{\text{free}}[1/p]$ is a finite free $\kS[1/p]$-module.
\end{thm}

\begin{thm}[{\cite[Theorem~4.4, Proposition~4.34]{BMS2018}}]
	There is a natural fully faithful tensor functor $T\amt M(T)$ from $\bZ_{p}$-lattices $T$ in a crystalline $G_{K}$-representation $V$ to finite free Breuil-Kisin modules over $\kS$, characterized by a $\vfi,G_{K_{\ift}}$-equivariant isomorphism $T\ot_{\bZ_{p}}A_{\text{inf}}[1/([\vep]-1)]\cong M(T)\ot_{\kS}A_{\text{inf}}[1/([\vep]-1)]$. There is an equality $D_{\text{crys}}(V)\ot_{W(k)[1/p]}B_{\text{crys}}\xP= M(T)\ot_{\kS}B_{\text{crys}}\xP$.
\end{thm}

\begin{stp}
	Let $\oL{K}$ an algebraic closure of $K$ with fixed $p$-power roots $\vpi^{1/p^{n}}\in\oL{K}$, denote $K_{\ift}:=K(\vpi^{1/p^{\ift}})$; let $C$ be the completion of $\oL{K}$ and $A_{\text{inf}}=\bA_{\text{inf}}(\cO_{C})$ with corresponding $A_{\text{crys}},B_{\text{crys}},B_{\text{dR}}$.
\end{stp}

\begin{dfn}
	A \tbf{Breuil-Kisin-Fargues module} $(M,\vfi_{M})$ is a finitely presented $A_{\text{inf}}$ module $M$ equipped with an isomorphism $\vfi_{M}:M\ot_{A_{\text{inf}},\vfi}A_{\text{inf}}[1/\vfi(\xi)]\cong M[1/\vfi(\xi)]$ such that $M[1/p]$ is a finite projective $A_{\text{inf}}[1/p]$-module.
\end{dfn}

\begin{thm}[{\cite[Theorem~4.28]{BMS2018}}]
	The category of finite free Breuil-Kisin-Fargues modules $(M,\vfi_{M})$ is equivalent to the category of pairs $(T,\Xi)$, where $T$ is a finite free $\bZ_{p}$-module, $\Xi$ is a $B_{\text{dR}}\xP$-lattice in $T\ot_{\bZ_{p}}B_{\text{dR}}$ given by
	\[
		T={(M\ot_{A_{\text{inf}}}W(C\xL))}^{\vfi_{M}=1},\quad \Xi=M\ot_{A_{\text{inf}}}B_{\text{dR}}\xP\sbs M\ot_{A_{\text{inf}}}B_{\text{dR}}=T\ot_{\bZ_{p}}B_{\text{dR}}.
	\]
\end{thm}

\begin{thm}[{\cite[Lemma~4.30, Proposition~4.32]{BMS2018}}]
	The map $\kS\ar A_{\text{inf}}, T\amt{[\vpi\xL]}^{p}$ is flat. The map $M\amt M\ot_{\kS}A_{\text{inf}}$ defines an exact tensor functor from Breuil-Kisin modules over $\kS$ to Breuil-Kisin-Fargues modules over $A_{\text{inf}}$.
\end{thm}

\begin{thm}[{\cite[Theorem~14.6]{BMS2018}}]
	Let $K$ be a complete discretely valued non-archimedean extension of $\bQ_{p}$ with perfect residue field $k$ and $C$ a completed algebraic closure of $K$. Let $\kX$ be a proper smooth formal scheme over $\cO_{K}$ with $X$ its rigid-analytic generic fiber over $K$. For fixed $i\gqs0$, the following holds
	\begin{enr}[label=(\arabic*)]
		\item There is a comparison isomorphism compatible with the Galois and Frobenius actions, and the filtration,
		\[
			\HH{\text{{\'e}t}}{i}{(X_{C},\bZ_{p})}\ot_{\bZ_{p}}B_{\text{crys}}\cong\HH{\text{crys}}{i}{(\kX_{k}/W(k))}\ot_{W(k)}B_{\text{crys}},
		\]
		in particular, $\HH{\text{{\'e}t}}{i}{(X_{C},\bQ_{p})}$ is a crystalline Galois representation.
		\item For all $n\gqs0$, the inequality holds
		\[
			\Len_{W(k)}\sR{\HH{\text{crys}}{i}{(\kX_{k}/W(k))}_{\text{tor}}/p^{n}}\gqs\Len_{\bZ_{p}}\sR{\HH{\text{{\'e}t}}{i}{(X_{C},\bZ_{p})}_{\text{tor}}/p^{n}},
		\]
		in particular, if $\HH{\text{crys}}{i}{(\kX_{k}/W(k))}$ is $p$-torsion-free, then so is $\HH{\text{{\'e}t}}{i}{(X_{C},\bZ_{p})}$.
		\item If $\HH{\text{crys}}{i}{(\kX_{k}/W(k))}$ and $\HH{\text{crys}}{i+1}{(\kX_{k}/W(k))}$ are $p$-torsion-free, then
		\[
			\HH{\text{crys}}{i}{(\kX_{k}/W(k))}= \mathrm{BK}\sR{\HH{\text{{\'e}t}}{i}{(X_{C},\bZ_{p})}}\ot_{\kS}W(k).
		\]
	\end{enr}
\end{thm}

\ssc{Topological Hochschild homology}

Ref:~\cite{BMS2019}.

\begin{thm}[{\cite[Theorem~1.2]{BMS2019}}]
	Let $\kX/\cO_{K}$ be a proper smooth formal scheme, there is a $\kS$-linear cohomology theory $R\Gma_{\kS}(\kX)$ equipped with a $\vfi$-linear Frobenius $\vfi:R\Gma_{\kS}(\kX)\ar R\Gma_{\kS}(kX)$ such that
	\begin{enr}[label=(\arabic*)]
		\item $R\Gma_{\kS}(\kX)\ot_{\kS}A_{\T{inf}}\cong R\Gma_{A_{\T{inf}}}(\kX_{\cO_{C}})$.
		\item $R\Gma_{\kS}(\kX)\ot_{\kS}^{\bL}\cO_{K}\cong R\Gma_{\T{dR}}(\kX/\cO_{K})$.
		\item $R\Gma_{\kS}(\kX)\ot_{\kS}^{\bL}W(k)\cong R\Gma_{\T{crys}}(\kX_{k}/W(k))$.
	\end{enr}
\end{thm}



\red{TODO}










\ssc{Prismatic cohomology}

Ref:~\cite{BS2022Prisms}.

\begin{dfn}
	Fix a prime $p$, a \tbf{$\dta$-ring} is a commutative $\bZ_{(p)}$-algebra with a map $\dta:A\ar A$ such that (1) $\dta(0)=\dta(1)=0$, (2) $\dta(xy)=x^{p}\dta(y)+y^{p}\dta(x)+p\dta(x)\dta(y)$ and (3) $\dta(x+y)=\dta(x)+\dta(y)+(x^{p}+y^{p}-{(x+y)}^{p})/p$.

	A \tbf{prism} $(A,I)$ is a $\dta$-ring $A$ and an ideal $I\sbs A$ defining a Cartier divisor such that (1) $A$ is $(p,I)$-adically complete and (2) $I+\phi_{A}(I)A$ contains $p$. A prism $(A,I)$ is, \tbf{bounded} if $A/I$ has bounded $p\xF$-torsion; \tbf{perfect} if $A$ is a perfect $\dta$-ring; \tbf{orientable} if $I$ is principal; \tbf{crystalline} if $I=(p)$.
\end{dfn}

\begin{thm}[{\cite[Theorem~3.10]{BS2022Prisms}}]
	The category of perfectoid rings $R$ and the category of perfect prisms $(A,I)$ are equivalent via $R\amt(\bA_{\text{inf}}(R),\Ker(\tta))$ and $(A,I)\amt A/I$.
\end{thm}

\begin{thm}[{\cite[Proposition~3.13]{BS2022Prisms}}]
	Let $(A,I)$ be a bounded prisms, $B$ a $(p,I)$-completely flat $\dta$-$A$-algebra, $J\sbs B$ an ideal containing $IB$, assume that Zariski locally on $\Spf{B}$, $J=(I,x_{1},\lds,x_{r})$ for a sequence $x_{1},\lds,x_{r}\in B$ that is $(p,I)$-completely regular relative to $A$, then there exists the \tbf{prismatic envelope} $(B,J)\ar (B{\{J/I\}}^{\wge},IB{\{J/I\}}^{\wge})$ such that (1) $(B{\{J/I\}}^{\wge},IB{\{J/I\}}^{\wge})$ is $(p,I)$-completely flat over $(A,I)$, (2) the construction $(B,J)\amt (B{\{J/I\}}^{\wge},IB{\{J/I\}}^{\wge})$ commutes with base change along arbitrary map of bounded prisms, and is compatible with flat localization on $B$.
\end{thm}

\begin{thm}[{\cite[Theorem~1.8]{BS2022Prisms}}]
	Fix a bounded prism $(A,I)$, and a smooth $p$-adic formal $A/I$-scheme $X$, $R\Gma_{\Prism}(X/A):=R\Gma({(X/A)}_{\Prism},\cO_{\Prism})$ is a commutative algebra in $D(A)$ equipped with a $\phi_{A}$-linear endomorphism $\phi$ such that the following holds,
	\begin{enr}[label=(\arabic*)]
		\item (Crystalline comparison) If $I=(p)$, there is a canonical $\phi$-equivariant isomorphism
		\[
			R\Gma_{\text{crys}}(X/A)\cong R\Gma_{\Prism}(X/A)\oH{\ot}^{\bL}_{A,\phi_{A}}A
		\]
		of commutative algebras in $D(A)$.
		\item (Hodge-Tate comparison) If $X=\Spf{R}$, there is a canonical $R$-module isomorphism
		\[
			\Oga_{R/(A/I)}^{i}\{-i\}\cong\HH{}{i}{(R\Gma_{\Prism}(X/A)\ot_{A}^{\bL}A/I)}.
		\]
		If $X$ is proper, $R\Gma_{\Prism}(X/A)$ is a perfect complex of $A$-modules.
		\item (de Rham comparison) There is a canonical isomorphism
		\[
			R\Gma_{\text{dR}}(X/(A/I))\cong R\Gma_{\Prism}(X/A)\oH{\ot}^{\bL}_{A,\phi_{A}}A/I
		\]
		of commutative algebras in $D(A)$.
		\item ({\'E}tale comparison) Assume $A$ is perfect. Let $X_{\eta}$ be the generic fiber of $X$ over $\bQ_{p}$ as an adic space, for any $n\gqs0$ there is a canonical isomorphism
		\[
			R\Gma_{\text{{\'e}t}}(X_{\eta},\bZ/p^{n}\bZ)\cong\sR{R\Gma_{\Prism}(X/A)/p^{n}[1/I]}^{\phi=1}
		\]
		of commutative algebras in $D(\bZ/p^{n}\bZ)$.
		\item (Base change) Let $(A,I)\ar(B,J)$ be a map of bounded prisms, $Y=X\tms_{\Spf{A/I}}\Spf{B/J}$, the natural map induces an isomorphism
		\[
			R\Gma_{\Prism}(X/A)\oH{\ot}_{A}^{\bL}B\cong R\Gma_{\Prism}(Y/B).
		\]
		\item The linearization is an isomorphism after inverting $I$,
		\[
			\phi_{A}\xS R\Gma_{\Prism}(X/A)\ar R\Gma_{\Prism}(X/A).
		\]
		If $I=(d)$ is principal, there is a map $V_{i}:\HH{\Prism}{i}{(X/A)}\ar\HH{\Prism}{i}{(\phi_{A}\xS R\Gma_{\Prism}(X/A))}$ such that $V_{i}\phi=\phi V_{i}=d^{i}$.
	\end{enr}
\end{thm}




\printref
\end{document}

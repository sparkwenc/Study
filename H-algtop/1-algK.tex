\documentclass[article, a4paper, twoside]{universal}

\setshowlvl{1}
\begin{document}
\confighead{}{}{}
\printhead{}{}{1}

\sct{Preliminaries}
Ref:~\cite{Weibel2013}.

% \ssc{Projective modules and $K_{0}$}

\begin{dfn}[\Rnum{2}.4.1, \Rnum{2}.4.3.1]
    A commutative ring $K$ is a \tbf{$\lda$-ring} if there is a family of maps $\lda^{k}:K\ar K$ for $k\gqs0$ such that (1) $\lda^{0}(x)=1,\lda^{1}(x)=x$ for all $x\in K$, (2) $\lda^{k}(x+y)=\sum_{i=0}^{k}\lda^{i}(x)\lda^{k-i}(y)$ for all $x,y\in K$, equivalently, $\lda_{t}:K\ar W(K):=1+tK\dS{t},x\amt\sum\lda^{k}(x)t^{k}$ is a homomorphism of groups.

    A $\lda$-ring $K$ is called \tbf{special} if $\lda_{t}$ is a homomorphism of $\lda$-rings.
\end{dfn}
\begin{dfn}[\Rnum{2}.4.2.1, \Rnum{2}.4.2.2]
    A \tbf{positive structure} on a $\lda$-ring $K$ is the data of (1) a binomial $\lda$-subring $H$ of $K$, (2) a $\lda$-ring surjection $\eps:K\atr H$ such that $\eps|_{H}=\Id_{H}$ and (3) a subset $P\sbs K$ of positive elements, such that
    \begin{itm}
        \item $P\sps\bN$, $P$ is closed under addition, multiplication and $\lda^{k}$-operations,
        \item all elements of $\Ker(\eps)$ are of the form $p-q$, where $p,q\in P$,
        \item for any $p\in P$, one has $\eps(p)=n\in\bN$, $\lda^{i}(p)=0$ for $i>n$ and $\lda^{n}(p)\in K\xT$.
    \end{itm}
    elements $l\in P$ with $\eps(l)=1$ are called \tbf{line elements}, denote by $L$ the group of line elements in $K$.

    $K$ satisfies the \tbf{splitting principle} if for any $p\in P$, there exists an extension of $\lda$-rings with positive structure $K\sbs K'$ such that $p$ is a sum of line elements in $K'$.


    Define the \tbf{Adams operations} $\psi^{k}:K\ar K$ for $k\gqs0$ via $\psi_{t}(x)=\sum\psi^{k}(x)t^{k}=\eps(x)-t\dif{\log\lda_{-t}(x)}/\dif{t}$.
\end{dfn}


\begin{thm}[\Rnum{2}.4.2.3, \Rnum{2}.4.4]
    Let $K$ be a $\lda$-ring with a positive structure,
    \begin{itm}
        \item if $\bN$ is cofinal in $P$, then $K$ satisfies splitting principle if and only if $K$ is special.
        \item if $K$ satisfies the splitting principle, then $\psi^{k}$ are ring endomorphisms, and $\psi^{j}\cc\psi^{k}=\psi^{jk}$ for all $j,k\gqs0$.
    \end{itm}
\end{thm}


\begin{dfn}
    Let $K$ be a special $\lda$-ring with a positive structure, the \tbf{$\lda$-operation} $\gma^{k}:K\ar K$ for $k\gqs0$ is defined by $\gma_{t}(x)=\sum\gma^{k}(x)t^{k}=\lda_{t/(1-t)}(x)$, equivalently, $\gma^{k}(x)=\lda^{k}(x+k-1)$. For $x\in K$, its $\gma$-dimension is defined to be $\Dim_{\gma}(x):=\sup\{n:\gma^{n}(x-\eps(x))\neq0\}$. The $\gma$-filtration defines a decreasing filtration $F_{\gma}\xB K$, where $F_{\gma}^{0}K=K, F_{\gma}^{1}K=\Ker(\eps)$ and for $n\gqs2$, $F_{\gma}^{n}K$ is the ideal generated by elements of the form $\gma^{k_{1}}(x_{1})\cds\gma^{k_{m}}(x_{m})$ with $x_{i}\in\Ker(\eps)$ and $\sum_{k_{i}}\gqs n$.
\end{dfn}

\begin{thm}[\Rnum{2}.4.9]
    Let $k,n\gqs1$ be integers, and $x\in F_{\gma}^{n}K$, then
    \[
        \lda^{k}(x)\eqv(-1)^{k}k^{n-1}x\bmod{F_{\gma}^{n+1}K},\quad \psi^{k}(x)\eqv k^{n}x\bmod{F_{\gma}^{n+1}K}.
    \]
\end{thm}


\begin{thm}[\Rnum{2}.4.10]
    Let $K$ be a $\lda$-ring with a positive structure in which every element has finite $\gma$-dimension, then $K_{\bQ}\cong\Op_{n\gqs0}K_{\bQ}^{(n)}$, where $K_{\bQ}^{(n)}\cong(F_{\gma}^{n}K/F_{\gma}^{n+1}K)_{\bQ}$. Moreover, eigenvalues of $\psi^{k}$ on $K_{\bQ}$ are a subset of $\{1,k,k^{2},\cds\}$, and $K_{\bQ}^{(n)}$ is the eigenspace of the eigenvalue $k^{n}$ for all $k$.
\end{thm}

\begin{thm}[\Rnum{2}.8.1]
    Let $X$ be a scheme, then $\Rnk:\KK_{0}(X)\ar\HH{}{0}{(X;\bZ)}$ is a split surjection of rings, $\Det:K_{0}(X)\ar\Pic(X)$ is a surjection of abelian groups, and $\Rnk\op\Det:\KK_{0}(X)\ar\HH{}{0}{(X;\bZ)}\op\Pic(X)$ is a surjection of rings.
\end{thm}

\begin{thm}[\Rnum{2}.8.2.1]
    If $X$ is a $1$-dimensional separated regular Noetherian scheme, then
    \[
        \KK_{0}(X)=\HH{}{0}{(X;\bZ)}\op\Pic(X).
    \]
\end{thm}

\begin{thm}[\Rnum{2}.8.5]
    Let $\cE$ be a vector bundle of rank $r+1$ over a quasi-compact scheme $X$, denote by $\pi:Y:=\bP(\cE)\ar X$ the projection, then $\KK_{0}(Y)$ is a free $\KK_{0}(X)$-module with basis $\{\cO_{Y}(-k)\}_{0\lqs k\lqs r}$.
\end{thm}

\begin{thm}[\Rnum{2}.8.8, \Rnum{2}.8.8.2]
    Let $X$ be a scheme, then $\KK_{0}(X)$ is a special $\lda$-ring, with $\lda^{k}[\cF]=[\wge^{k}\cF]$, $\eps$ given by $\Rnk:\KK_{0}(X)\ar\HH{}{0}{(X;\bZ)}$ and positive elements given by classes of vector bundles.
\end{thm}


% \ssc{Higher $K$-groups}

\begin{dfn}[\Rnum{4}.1.3, \Rnum{4}.1.4.1]
    A topological space $E$ is \tbf{acyclic} if it has the homology of a point. Let $X,Y$ be pointed connected CW complexes, a map $f:X\ar Y$ is \tbf{acyclic} if its homotopy fiber $F(f)$ is acyclic, let $P$ be a perfect normal subgroup of $\pi_{1}(X)$, $f$ is called a \tbf{$+$-construction relative to $P$} if $P=\Ker(\pi_{1}(X)\ar\pi_{1}(Y))$.
\end{dfn}

\begin{thm}[Quillen, \Rnum{4}.1.5]
    Let $X$ be a pointed CW complex and $P$ a perfect normal subgroup of $\pi_{1}(X)$, then there exists a $+$-construction $f:X\ar X\xP$ relative to $P$, where $f$ and $X\xP$ are unique up to homotopy.
\end{thm}

\begin{dfn}[Quillen, \Rnum{4}.1.1]
    Let $R$ be a ring, $\GL(R)=\ilim_{n}\GL_{n}(R)$ and $E(R)=[\GL(R),\GL(R)]$ and $\bB\GL(R)$ its classifying space. Let $i:\bB\GL(R)\ar\bB\GL(R)\xP$ be the $+$-construction relative to $E(R)$, the $\KK$-groups of $R$ are defined to be $\KK_{n}(R):=\pi_{n}(\bB\GL(R)\xP)$ for $n\gqs1$. Denote by $K(R):=K_{0}(R)\tms\bB\GL(R)\xP$.
\end{dfn}

\begin{thm}[Loday, \Rnum{4}.1.10]
    Let $R,S$ be two rings, then the product map $\KK_{p}(R)\ot\KK_{q}(S)\ar\KK_{p+q}(R\ot S)$, where $p,q\gqs1$, is natural in $R$ and $S$, bilinear and associative.
\end{thm}

\begin{thm}[{\cite{Quillen1972}}, \Rnum{4}.1.12]
    The Brauer lifting associated to an embedding $\oL{\bF}_{q}\xT\ar\bC\xT$ gives a homotopy fibration $\bB\GL(\bF_{q})\xP\axr{\rho}\bB\RU\axr{\psi^{q}-1}\bB\RU$, thus the associated long exact sequence gives
    \[
        \begin{tikzcd}
            \cds\ar[r] & \pi_{2k}(\bB\RU)\ar[d, equal]\ar[r] & \pi_{2k}(\bB\RU)\ar[r]\ar[d, equal] & \pi_{2k-1}(\bB\GL(\bF_{q})\xP)\ar[r]\ar[d, equal] & \pi_{2k-1}(\bB\RU)\ar[d, equal] \ar[r] & \cds\\
            \cds\ar[r] & \bZ\ar[r, "q^{k}-1"] & \bZ\ar[r] & \bZ/(q^{k-1})\ar[r] & 0 \ar[r] & \cds
        \end{tikzcd}
    \]

    Thus one has
    \[
        \KK_{r}(\bF_{q})=\begin{cases}
          \bZ & r = 0 \\
          \bZ/(q^{k}-1) & r = 2k-1 \\
          0 & \T{otherwise}
        \end{cases}
    \]
\end{thm}

\begin{thm}[Borel, \Rnum{4}.1.17, \Rnum{4}.1.18]
    Let $A$ be a finite-dimensional semisimple $\bQ$-algebra, and $R$ any order in $A$, then $\KK_{n}(R)\ot\bQ\cong\KK_{n}(A)\ot\bQ$ for all $n\gqs2$. Moreover, if $F$ is a number field with $F\ot_{\bQ}\bR=\bR^{r_{1}}\ot\bC^{r_{2}}$, $A$ a central simple $F$-algebra, then one has $\KK_{n}(A)\ot\bQ\cong\KK_{n}(F)\ot\bQ$ for all $n\gqs2$, and
    \[
       \Rnk_{\bQ}\KK_{n}(A)\ot\bQ=\begin{cases}
         r_{2} & n\eqv 3\bmod{4} \\
         r_{1}+r_{2} & n\eqv 1\bmod{4} \\
         0 & \T{otherwise}
       \end{cases}
    \]
\end{thm}

% \ssc{$K$-theory with finite coefficient}

\begin{stp}
    Let $\ell$ be a positive integer.
\end{stp}

\begin{dfn}[\Rnum{4}.2.1, \Rnum{4}.2.4]
    Take $m\gqs2$, let $P^{m}(\bZ/\ell)$ be the Moore space of $\bZ/\ell$ for $m$.

    Let $X$ be a pointed topological space, then define the pointed set $\pi_{m}(X;\bZ/\ell):=[P^{m}(\bZ/\ell),X]$, it is a group for $m\gqs3$ and abelian for $m\gqs4$. Let $R$ be a ring, then define $\KK_{m}(R;\bZ/\ell):=\pi_{m}(\KK(R); \bZ/\ell)$.
\end{dfn}

\begin{thm}
    If $\ell=pq$ with $p,q$ are relatively prime, then $\pi_{m}(X;\bZ/\ell)\cong\pi_{m}(X;\bZ/p)\tms\pi_{m}(X;\bZ/q)$ naturally.
\end{thm}

\begin{thm}[Universal Coefficient, \Rnum{4}.2.2, \Rnum{4}.2.5]
    For $m\gqs3$, there are natural short exact sequences
    \begin{align*}
        0\ar\pi_{m}(X)\ot\bZ/\ell\ar\pi_{m}(X;\bZ/\ell)\ar\pi_{m-1}(X)[\ell]\ar0,\\
        0\ar\KK_{m}(R)\ot\bZ/\ell\ar\KK_{m}(R;\bZ/\ell)\ar\KK_{m-1}(R)[\ell]\ar 0,
    \end{align*}
    which is split (not naturally) when $\ell\not\eqv 2\bmod{4}$.
\end{thm}

\begin{thm}[Gabber's rigidity, \Rnum{4}.2.10]
    Let $(R,I)$ be a Henselian pair with $1/\ell\in R$, then for all $n\gqs1$, $\KK_{n}(R;\bZ/\ell)\cong \KK_{n}(R/I;\bZ/\ell)$.
\end{thm}

\begin{cnj}[Quillen, \Rnum{4}.6.9]
    Let $K$ be a global field, then $\KK_{n}(\cO_{K})$ is finite generated for all $n$.
\end{cnj}

\begin{thm}[\Rnum{5}.5.1, Quillen, Abelian localization]
    Let $\cB$ be a Serre subcategory of an abelian category $\cA$, then $\KK(\cB)\ar \KK(\cA)\axr{\mrm{loc}}\KK(\cA/\cB)$ is a homotopy fibration, there is a long exact sequence
    \[
        \cds\axr{\ptl}\KK_{n}(\cB)\ar\KK_{n}(\cA)\axr{\mrm{loc}}\KK_{n}(\cA/\cB)\ar\cds\ar\KK_{0}(\cA)\axr{\mrm{loc}} \KK_{0}(\cA/\cB)\ar0.
    \]
\end{thm}

\begin{thm}[\Rnum{5}.6.3]
    Let $R$ be a regular Noetherian ring, then for all $n$,
    \[
        \KK_{n}(R[t])\cong\KK_{n}(R),\quad \KK_{n}(R[t,t\xI])\cong \KK_{n}(R)\op \KK_{n-1}(R).
    \]
\end{thm}


\begin{thm}[\Rnum{5}.6.8]
    Let $R$ be a Dedekind domain such that $\Frc R$ is a global field. Then for odd $n\gqs3$, $\KK_{n}(R)\cong \KK_{n}(F)$, and for even $n\gqs2$, there is an exact sequence
    \[
        0\ar \KK_{n}(R)\ar \KK_{n}(F)\ar \Op_{\kp}\KK_{n-1}(R/\kp)\ar 0.
    \]
\end{thm}

\begin{thm}[\Rnum{5}.8.3]
    For any quasi-projective scheme $X$ and $n$, there is a canonically split exact sequence,
    \[
        0\ar \KK_{n}(X)\axr{\Dta}\KK_{n}(X\tms\Spc\bZ[t])\op \KK_{n}(X\tms\Spc\bZ[t\xI])\axr{\pm} \KK_{n}(X\tms\Spc\bZ[t,t\xI])\axr{\ptl} \KK_{n-1}(X)\ar 0.
    \]
\end{thm}

\begin{thm}[\Rnum{6}.1.3.1]
    Let $F$ be an algebraically closed field of characteristic $p>0$,  then (1) for positive even $n$, $\KK_{n}(F)$ is uniquely divisible, (2) for positive odd $n=2i-1$, $\KK_{2i-1}(F)$ is the direct sum of a uniquely divisible group and the torsion group $\bQ/\bZ[p\xI]$. If $p\nmid m$, the choice of a Bott element $\bta\in \KK_{2}(F;\bZ/m)$ determines a graded ring isomorphism $\KK\zB(F;\bZ/m)\cong(\bZ/m)[\bta]$.
\end{thm}

\begin{thm}[\Rnum{6}.1.6, \Rnum{6}.1.7.1, \Rnum{6}.1.4.1]
    Let $F$ be an algebraically closed field of characteristic $0$, then (1) for positive even $n$, $\KK_{n}(F)$ is a uniquely divisible group, (2) for positive odd $n=2i-1$, $\KK_{n}$ is the direct sum of a uniquely divisible group and the torsion group $\bQ/\bZ$, moreover, $\KK_{2i-1}(F)_{\T{tor}}\cong\mu_{F}(i)$ as $\Aut(F)$-modules. The choice of a Bott element $\bta\in \KK_{2}(F;\bZ/m)$ determines a graded ring isomorphism $\KK\zB(F;\bZ/m)\cong(\bZ/m)[\bta]$.
\end{thm}

\begin{dfn}[\Rnum{6}.2.1]
    Let $F$ be any field with Galois group $G$, the \tbf{$e$-invariant} is given by the natural map
    \[
        e:\KK_{2i-1}(F)_{\T{tor}}\ar\KK_{2i-1}(\oL{F})_{\T{tor}}^{G}\cong\mu_{F}(i)^{G}.
    \]
    Denote by $w_{i}(F)$ the order of $\mu_{F}(i)^{G}$ and $w_{i}^{(\ell)}(F)$ its $\ell$-component.
\end{dfn}

\begin{thm}[\Rnum{6}.2.2, \Rnum{6}.2.3, \Rnum{6}.2.4]
    Let $\ell$ be an odd prime, $F$ a field with $\Chr(F)\neq\ell$, $a\lqs\ift$ maximal such that $F(\zta_{\ell})$ contains a primitive $\ell^{a}$-th root of unity, and $r:=[F(\zta_{\ell}):F]$, $i=c\ell^{b}$ where $\ell\nmid c$, then
    \[
        w_{i}^{(\ell)}=\begin{cases}
            \ell^{a+b} & \zta_{\ell}\in F \\
            \ell^{a+b} & \zta_{\ell}\nin F, i\eqv 0\bmod r \\
            1 & \zta_{\ell}\nin F, i\not\eqv 0\bmod r
        \end{cases}
    \]

    Let $F$ be a field with $\Chr(F)\neq2$, $a\lqs\ift$ maximal such that $F(\sqrt{-1})$ contains a primitive $2^{a}$-th root of unity, $i=c2^{b}$ where $2\nmid c$, say $F$ is exceptional if $\Chr(F)=0$ and $\Gal(F(\zta_{2^{v}})/F)$ are not cyclic for large $v$, then
    \[
        w_{i}^{(2)}(F)=\begin{cases}
          2^{a+b} & \sqrt{-1}\in F \\
          2 & \sqrt{-1}\nin F, b=0 \\
          2^{a+b} & \sqrt{-1}\nin F, b>0, \T{$F$ is exceptional} \\
          2^{a+b-1} & \sqrt{-1}\nin F, b>0, \T{$F$ is not exceptional}
        \end{cases}
    \]

    If $i=2k$, then $w_{i}(\bQ)$ is the denominator of $B_{k}/4k$. A prime $\ell$ divides $w_{i}(\bQ)$ if and only if $(\ell-1)$ divides $i$.
\end{thm}

\begin{thm}[Harris-Segal, \Rnum{6}.2.5, \Rnum{6}.2.5.1]
    Let $F$ be a field with $\Chr(F)\neq\ell$, when $\ell=2$ assume $F$ is not exceptional. Then each $\KK_{2i-1}(F)$ admits a direct summand isomorphic to $\bZ/w_{i}^{(\ell)}(F)$, if $F=\Frc R$ for an integrally closed domain $R$, then $\KK_{2i-1}(R)$ admits a direct summand isomorphic to $\bZ/w_{i}(F)$, detected by the $e$-invariant, called the \tbf{Harris-Segal summand}, the splitting $\bZ/w_{i}\ar K_{2i-1}(R)$ is called the \tbf{($\ell$-primary) Harris-Segal map}.

    If $F$ is exceptional, there is a cyclic summand in $\KK_{2i-1}(F)$ whose order is either $w_{i}(F),2w_{i}(F)$ or $w_{i}(F)/2$.
\end{thm}


\begin{thm}[\Rnum{6}.2.6]
    Let $F$ be a real number field, then the Harris-Segal summand in $\KK_{2i-1}(F)$ and $\KK_{2i-1}(\cO_{F})$ are (1) $\bZ/w_{i}(F)$ if $i\eqv0,1\bmod{4}$, (2) $\bZ/2w_{i}(F)$ if $i\eqv2\bmod{4}$, (3) $\bZ/(w_{i}(F)/2)$ if $i\eqv3\bmod{4}$.
\end{thm}

\begin{thm}[Suslin, \Rnum{6}.3.1, \Rnum{6}.3.2]
    For all $n\gqs1$, one has (1) $\KK_{n}(\bR)$ is the direct sum of a uniquely divisible group and $\KK_{n}(\bR)_{\T{tor}}$; $\KK_{n}(\bH)$ is the direct sum of a uniquely divisible group and $\KK_{n}(\bH)_{\T{tor}}$, (2) for all integers $m$, the maps $\KK_{n}(\bR;\bZ)\ar\pi_{n}(\bB\RO;\bZ/m)$ and $\KK_{n}(\bR;\bZ)\ar\pi_{n}(\bB\Sp)$ are isomorphisms, (3) the map $\KK_{n}(\bR)_{\T{tor}}\ar\KK_{n}(\bC)_{\T{tor}}\ar\KK_{n}(\bH)_{\T{tor}}$ is given by
\begin{table}[H]
    \centering
    \begin{tabular}{c|c|c|c|c|c|c|c|c}
      \toprule
        $n\bmod 8$ &0&1&2&3&4&5&6&7 \\
      \midrule
        $\KK_{n}(\bR)_{\T{tor}}$ & 0 & $\bZ/2$  & $\bZ/2$ & $\bQ/\bZ$ & 0 & 0         & 0       & $\bQ/\bZ$ \\
        $\ad$         & 0 &  $\ahr$  & 0       &   $a\amt 2a$  & 0 & 0         & 0       &   $\cong$    \\
        $\KK_{n}(\bC)_{\T{tor}}$ & 0 & $\bQ\bZ$ & 0       & $\bQ/\bZ$ & 0 & $\bQ/\bZ$ & 0       & $\bQ/\bZ$ \\
        $\ad$         & 0 & 0        & 0       &     $\cong$    & 0 & 0         & 0       &    $a\amt 2a$   \\
        $\KK_{n}(\bH)_{\T{tor}}$ & 0 & 0        & 0       & $\bQ/\bZ$ & 0 & $\bZ/2$   & $\bZ/2$ & $\bQ/\bZ$ \\
      \bottomrule
    \end{tabular}
\end{table}
\end{thm}

\begin{thm}[Motivic-to-K spectral sequence, \Rnum{6}.4.2]
    Let $X$ be a smooth scheme over a field $k$ and $A$ an abelian group, there is a spectral sequence, $E_{2}^{p,q}:=\HH{\T{mot}}{p-q}{(X;A(-q))}\aR\KK_{-p-q}(X;A)$, natural in $X$ and $A$.
\end{thm}

\begin{thm}[{\cite{RW2000,HM2003Local}}, \Rnum{6}.7.4]
    Let $E$ be a $p$-adic field of degree $d$, then for $n\gqs2$, one has
    \[
        \KK_{n}(\cO_{E};\bZ_{p})\cong \KK_{n}(E;\bZ_{p})\cong \begin{cases}
          \bZ/w_{i}^{(p)}(E) & n = 2i \\
          \bZ_{p}^{d}\op\bZ/w_{i}^{(p)}(E) & n = 2i - 1
        \end{cases}
    \]
\end{thm}

\begin{thm}[\Rnum{6}.8.4]
    Let $F$ be a totally imaginary number field, $S$ a finite set of finite places, then
    \[
        \KK_{n}(\cO_{S})\cong \begin{cases}
          \bZ\op\Pic(\cO_{S}) & n=0 \\
          \bZ^{r_{2}+\sP{S}-1}\op\bZ/w_{1}(F) & n=1 \\
          \Op_{\ell}\HH{\T{{\'e}t}}{2}{(\cO_{S}[\ell\xI];\bZ_{\ell}(i+1))} & n=2i\gqs2 \\
          \bZ^{r_{2}}\op\bZ/w_{i}(F) & n=2i-1\gqs3
        \end{cases}
    \]
\end{thm}

\begin{thm}[{\cite{Wiles1990Iwasawa}}, \Rnum{6}.8.7, \Rnum{6}.8.8]
    Let $F$ be a totally real number field.
    \begin{itm}
        \item If $\ell$ is a odd prime, then for all $k\gqs1$, there is a rational number $u_{k}$ prime to $\ell$ such that
        \[
            \zta_{F}(1-2k)=u_{k}\frac{\sP{\HH{\T{{\'e}t}}{2}{(\cO_{F}[\ell\xI];\bZ_{\ell}(2k))}}}{\sP{\HH{\T{{\'e}t}}{1}{(\cO_{F}[\ell\xI];\bZ_{\ell}(2k))}}}.
        \]

        \item If $\Gal(F/\bQ)$ is abelian, then for all $k\gqs1$,
        \[
            \zta_{F}(1-2k)=(-1)^{kr_{1}}2^{r_{1}}\frac{\sP{\KK_{4k-2}(\cO_{F})}}{\sP{\KK_{4k-1}(\cO_{F})}}.
        \]
    \end{itm}
\end{thm}

\begin{thm}[\Rnum{6}.9.4]
    Let $F$ be a real number field, $S$ a finite set of finite places, its $r_{1}$ real embeddings define the map
    \[
        \afa_{S}^{n}(k):\HH{\T{{\'e}t}}{n}{(\cO_{S};(\bZ/2\xF)(k))}\ar\Op^{r_{1}}\HH{{\T{{\'e}t}}}{n}{(\bR;(\bZ/2\xF)(k))}\cong \begin{cases}
          (\bZ/2)^{r_{1}} & \T{$k-n$ odd} \\
          0 & \T{$k-n$ even}
        \end{cases}
    \]
    $\afa_{S}^{n}(i)$ is an isomorphism for $n\gqs3$, or $n=2,i\gqs2$.

    If $S$ contains $2$, then for $n\gqs0$,
    \[
        \KK_{n}(\cO_{S};\bZ/2\xF)\cong \begin{cases}
          \bZ/w_{4k}(F) & n\eqv0\bmod 8 \\
          \HH{\T{{\'e}t}}{1}{(\cO_{S};(\bZ/2\xF)(4k+1))} & n\eqv1\bmod 8 \\
          \bZ/2 & n\eqv2\bmod 8 \\
          \HH{\T{{\'e}t}}{1}{(\cO_{S};(\bZ/2\xF)(4k+2))} & n\eqv3\bmod 8 \\
          \bZ/2w_{4k+2}(F)\op(\bZ/2)^{r_{1}-1} & n\eqv4\bmod 8 \\
          (\bZ/2)^{r_{1}-1}\rtm\HH{\T{{\'e}t}}{1}{(\cO_{S};(\bZ/2\xF)(4k+3))} & n\eqv5\bmod 8 \\
          0 & n\eqv6\bmod 8 \\
          \Ker(\afa_{S}^{1}(4k+4)) & n\eqv7\bmod 8
        \end{cases}
    \]
\end{thm}

\begin{thm}[\Rnum{6}.9.5]
    Let $F$ be a number field, $S$ a finite set of finite places, for each odd $n\gqs3$, $\KK_{n}(\cO_{S})\cong\KK_{n}(F)$ is given by
    \begin{itm}
        \item if $F$ is totally imaginary, $\KK_{n}(F)\cong\bZ^{r_{2}}\op\bZ/w_{i}(F)$.
        \item if $F$ has $r_{1}>0$ real embeddings, let $k=(n+1)/2$,
        \[
            \KK_{n}(F)\cong \begin{cases}
              \bZ^{r_{1}+r_{2}}\op\bZ/w_{k}(F) & n\eqv1\bmod 8 \\
              \bZ^{r_{2}}\op\bZ/2w_{k}(F)\op(\bZ/2)^{r_{1}-1} & n\eqv3\bmod 8 \\
              \bZ^{r_{1}+r_{2}}\op\bZ/(w_{k}(F)/2) & n\eqv5\bmod 8 \\
              \bZ^{r_{2}}\op\bZ/w_{k}(F) & n\eqv7\bmod 8
            \end{cases}
        \]
    \end{itm}
\end{thm}

\begin{thm}[\Rnum{6}.9.6.1, \Rnum{6}.9.7]
    Let $F$ be a real number field, $S$ a finite set of finite places, define the \tbf{signature defect} $j$ to be $j:=\Dim\Cok(\afa^{1}:\HH{\T{{\'e}t}}{1}{(\cO_{S};\bZ/2)}\ar\Op^{r_{1}}\HH{\T{{\'e}t}}{1}{(\bR;\bZ/2)}=(\bZ/2)^{r_{1}})$. If $S$ contains $2$, then for $n>0$,
    \[
        \KK_{n}(\cO_{S};\bZ/2)=\begin{cases}
          \Ker(\afa^{2})\op\bZ/2 & n\eqv0\bmod 8 \\
          \HH{\T{{\'e}t}}{1}{\cO_{S};\bZ/2} & n\eqv1\bmod 8 \\
          \HH{\T{{\'e}t}}{2}{\cO_{S};\bZ/2}\rtm\bZ/2 & n\eqv2\bmod 8 \\
          (\bZ/2)^{r_{1}-1}\rtm\HH{\T{{\'e}t}}{1}{\cO_{S};\bZ/2} & n\eqv3\bmod 8 \\
          (\bZ/2)^{j}\rtm\HH{\T{{\'e}t}}{2}{\cO_{S};\bZ/2} & n\eqv4\bmod 8 \\
          (\bZ/2)^{r_{1}-1}\rtm\Ker(\afa^{1}) & n\eqv5\bmod 8 \\
          (\bZ/2)^{j}\op\Ker(\afa^{2}) & n\eqv6\bmod 8 \\
          \Ker(\afa^{1}) & n\eqv7\bmod 8 \\
        \end{cases}
    \]
\end{thm}

\begin{thm}[\Rnum{6}.9.12]
    Let $F$ be a totally real number field with $r_{1}$ real embeddings, then for all even $k>0$,
    \[
        2^{r_{1}}\frac{\sP{\KK_{2k-2}(\cO_{S})}}{\sP{\KK_{2k-1}(\cO_{S})}}=\frac{\prod_{\ell}\sP{\HH{\T{{\'e}t}}{2}{(\cO_{S}[\ell\xI];\bZ_{\ell}(k))}}}{\prod_{\ell}\sP{\HH{\T{{\'e}t}}{1}{(\cO_{S}[\ell\xI];\bZ_{\ell}(k))}}}.
    \]
\end{thm}


\begin{thm}[\Rnum{6}.10.1]
    Let $c_{k}$ be the numerator of $B_{k}/4k$, then for odd $n\gqs2$,
    \[
        \KK_{n}(\bZ)\cong\KK_{n}(\bQ) \begin{cases}
          \bZ\op\bZ/2 & n\eqv1\bmod 8 \\
          \bZ/2w_{4k+2} & n\eqv3\bmod 8 \\
          \bZ & n\eqv5\bmod 8 \\
          \bZ/w_{4k+4} & n\eqv7\bmod 8
        \end{cases}
    \]
    if $n\eqv2\bmod 8$, $\sP{\KK_{n}(\bZ)}=2c_{2k+1}$; if $n\eqv6\bmod 8$, $\sP{\KK_{n}(\bZ)}=c_{2k+1}$.
\end{thm}


\begin{thm}[\Rnum{6}.10.2]
    Assuming Vandiver's conjecture: any irregular prime $\ell$, $\Pic(\bZ[\zta_{\ell}+\zta_{\ell}\xI])$ has no $\ell$-torsion. Then for all $n\gqs2$, let $k=\sFl{1+n/4}$, one has
    \[
        \KK_{n}(\bZ)= \begin{cases}
          0 & n\eqv0\bmod 8 \\
          \bZ\op \bZ/2 & n\eqv1\bmod 8 \\
          \bZ/2c_{k} & n\eqv2\bmod 8 \\
          \bZ/2w_{2k} & n\eqv3\bmod 8 \\
          0 & n\eqv4\bmod 8 \\
          \bZ & n\eqv5\bmod 8 \\
          \bZ/c_{k} & n\eqv6\bmod 8 \\
          \bZ/w_{2k} & n\eqv7\bmod 8
        \end{cases}
    \]
\end{thm}

\sct{Computations}

Ref:~arXiv2405.04329.

\begin{stp}
    Let $p$ be a prime, $K$ a $p$-adic field of ramification degree $e$ and residue degree $f$ with uniformizer $\vpi$, denote by $R_{n}:=\cO_{K}/\vpi^{n}$. Then $\RMK_{r}(R_{n})$ can be effectively determined for all $r\gqs0$ and $n\gqs1$.

\end{stp}

 $\RMK_{r}(R_{n};\bZ[1/p])\cong \RMK_{r}(\bF_{q};\bZ[1/p])$ for $r\gqs0$.
\begin{cmt}[1]
if $n\lqs e$, $R_{n}=\bF_{q}[z]/z^{n}$, by Hesselholt-Madsen\cite{HM1997Cyclic}, \red{(to add)}.
\end{cmt}

(\cite{DGMC2013}) $\RMK_{r}(R_{n};\bZ_{p})\cong\mrm{TC}_{r}(R_{n};\bZ_{p})$ for $r\gqs0$. (\cite{BMS2019}) There is a strictly decreasing filtration $F_{\T{mot}}^{\gqs\blt}\mrm{TC}(R_{n};\bZ_{p})$ with graded piece $\Gr_{\T{mot}}^{k}\mrm{TC}(R_{n};\bZ_{p})\cong\bZ_{p}(k)(R)[2k]$ , whose spectral sequences converges to $\mrm{TC}_{r}(R_{n};\bZ_{p})$, consequently, one has
\[
    \KK_{r}(R_{n};\bZ_{p})\cong \begin{cases}
      \HH{}{0}{(\bZ_{p}(0)(R_{n}))}\cong \bZ_{p} & r = 0 \\
      \HH{}{1}{(\bZ_{p}(k)(R_{n}))} & r=2k-1, k\gqs1 \\
      \HH{}{2}{(\bZ_{p}(k)(R_{n}))} & r=2k-2, k\gqs2
    \end{cases}
\]


Moreover, the cohomology of $\bZ_{p}(k)(R_{n}),k\gqs1$ could be effectively computed, there is a cochain complex
\[
    \bZ_{p}^{f(kn-1)}\axr{\mrm{syn}^{0}}\bZ_{p}^{2f(kn-1)}\axr{\mrm{syn}^{1}}\bZ_{p}^{f(kn-1)},
\]
where $\mrm{syn}^{0}$ and $\mrm{syn}^{1}$ are effectively computable, and which is quasi-isomorphic to $\bZ_{p}(k)(R)$. If $k\gqs2$ satisfies
\[
    k-1\gqs\frac{p}{p-1}\sR{\frac{p}{p-1}(p^{\sCe{n/e}}-1)-p^{\sCe{n/e}}(\sCe{n/e}-n/e)},
\]
then $\RMK_{2k-2}(R_{n};\bZ_{p})=0$.



\printref
\end{document}


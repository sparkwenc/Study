\documentclass[article, a4paper, twoside]{universal}

\setshowlvl{1}
\begin{document}
\confighead{}{}{}
\printhead{}{}{1}

\sct{Preliminaries}
Ref:~\cite{Weibel2013}.

% \ssc{Projective modules and $K_{0}$}

\begin{dfn}[\Rnum{2}.4.1, \Rnum{2}.4.3.1]
    A commutative ring $K$ is a \tbf{$\lda$-ring} if there is a family of maps $\lda^{k}:K\ar K$ for $k\gqs0$ such that (1) $\lda^{0}(x)=1,\lda^{1}(x)=x$ for all $x\in K$, (2) $\lda^{k}(x+y)=\sum_{i=0}^{k}\lda^{i}(x)\lda^{k-i}(y)$ for all $x,y\in K$, equivalently, $\lda_{t}:K\ar W(K):=1+tK\dS{t},x\amt\sum\lda^{k}(x)t^{k}$ is a homomorphism of groups.

    A $\lda$-ring $K$ is called \tbf{special} if $\lda_{t}$ is a homomorphism of $\lda$-rings.
\end{dfn}
\begin{dfn}[\Rnum{2}.4.2.1, \Rnum{2}.4.2.2]
    A \tbf{positive structure} on a $\lda$-ring $K$ is the data of (1) a binomial $\lda$-subring $H$ of $K$, (2) a $\lda$-ring surjection $\eps:K\atr H$ such that $\eps|_{H}=\Id_{H}$ and (3) a subset $P\sbs K$ of positive elements, such that
    \begin{itm}
        \item $P\sps\bN$, $P$ is closed under addition, multiplication and $\lda^{k}$-operations,
        \item all elements of $\Ker(\eps)$ are of the form $p-q$, where $p,q\in P$,
        \item for any $p\in P$, one has $\eps(p)=n\in\bN$, $\lda^{i}(p)=0$ for $i>n$ and $\lda^{n}(p)\in K\xT$.
    \end{itm}
    elements $l\in P$ with $\eps(l)=1$ are called \tbf{line elements}, denote by $L$ the group of line elements in $K$.

    $K$ satisfies the \tbf{splitting principle} if for any $p\in P$, there exists an extension of $\lda$-rings with positive structure $K\sbs K'$ such that $p$ is a sum of line elements in $K'$.


    Define the \tbf{Adams operations} $\psi^{k}:K\ar K$ for $k\gqs0$ via $\psi_{t}(x)=\sum\psi^{k}(x)t^{k}=\eps(x)-t\dif{\log\lda_{-t}(x)}/\dif{t}$.
\end{dfn}


\begin{thm}[\Rnum{2}.4.2.3, \Rnum{2}.4.4]
    Let $K$ be a $\lda$-ring with a positive structure,
    \begin{itm}
        \item if $\bN$ is cofinal in $P$, then $K$ satisfies splitting principle if and only if $K$ is special.
        \item if $K$ satisfies the splitting principle, then $\psi^{k}$ are ring endomorphisms, and $\psi^{j}\cc\psi^{k}=\psi^{jk}$ for all $j,k\gqs0$.
    \end{itm}
\end{thm}


\begin{dfn}
    Let $K$ be a special $\lda$-ring with a positive structure, the \tbf{$\lda$-operation} $\gma^{k}:K\ar K$ for $k\gqs0$ is defined by $\gma_{t}(x)=\sum\gma^{k}(x)t^{k}=\lda_{t/(1-t)}(x)$, equivalently, $\gma^{k}(x)=\lda^{k}(x+k-1)$. For $x\in K$, its $\gma$-dimension is defined to be $\Dim_{\gma}(x):=\sup\{n:\gma^{n}(x-\eps(x))\neq0\}$. The $\gma$-filtration defines a decreasing filtration $F_{\gma}\xB K$, where $F_{\gma}^{0}K=K, F_{\gma}^{1}K=\Ker(\eps)$ and for $n\gqs2$, $F_{\gma}^{n}K$ is the ideal generated by elements of the form $\gma^{k_{1}}(x_{1})\cds\gma^{k_{m}}(x_{m})$ with $x_{i}\in\Ker(\eps)$ and $\sum_{k_{i}}\gqs n$.
\end{dfn}

\begin{thm}[\Rnum{2}.4.9]
    Let $k,n\gqs1$ be integers, and $x\in F_{\gma}^{n}K$, then
    \[
        \lda^{k}(x)\eqv(-1)^{k}k^{n-1}x\bmod{F_{\gma}^{n+1}K},\quad \psi^{k}(x)\eqv k^{n}x\bmod{F_{\gma}^{n+1}K}.
    \]
\end{thm}


\begin{thm}[\Rnum{2}.4.10]
    Let $K$ be a $\lda$-ring with a positive structure in which every element has finite $\gma$-dimension, then $K_{\bQ}\cong\Op_{n\gqs0}K_{\bQ}^{(n)}$, where $K_{\bQ}^{(n)}\cong(F_{\gma}^{n}K/F_{\gma}^{n+1}K)_{\bQ}$. Moreover, eigenvalues of $\psi^{k}$ on $K_{\bQ}$ are a subset of $\{1,k,k^{2},\cds\}$, and $K_{\bQ}^{(n)}$ is the eigenspace of the eigenvalue $k^{n}$ for all $k$.
\end{thm}

\begin{thm}[\Rnum{2}.8.1]
    Let $X$ be a scheme, then $\Rnk:K_{0}(X)\ar\HH{}{0}{(X;\bZ)}$ is a split surjection of rings, $\Det:K_{0}(X)\ar\Pic(X)$ is a surjection of abelian groups, and $\Rnk\op\Det:K_{0}(X)\ar\HH{}{0}{(X;\bZ)}\op\Pic(X)$ is a surjection of rings.
\end{thm}

\begin{thm}[\Rnum{2}.8.2.1]
    If $X$ is a $1$-dimensional separated regular Noetherian scheme, then
    \[
        K_{0}(X)=\HH{}{0}{(X;\bZ)}\op\Pic(X).
    \]
\end{thm}

\begin{thm}[\Rnum{2}.8.5]
    Let $\cE$ be a vector bundle of rank $r+1$ over a quasi-compact scheme $X$, denote by $\pi:Y:=\bP(\cE)\ar X$ the projection, then $K_{0}(Y)$ is a free $K_{0}(X)$-module with basis $\{\cO_{Y}(-k)\}_{0\lqs k\lqs r}$.
\end{thm}

\begin{thm}[\Rnum{2}.8.8, \Rnum{2}.8.8.2]
    Let $X$ be a scheme, then $K_{0}(X)$ is a special $\lda$-ring, with $\lda^{k}[\cF]=[\wge^{k}\cF]$, $\eps$ given by $\Rnk:K_{0}(X)\ar\HH{}{0}{(X;\bZ)}$ and positive elements given by classes of vector bundles.
\end{thm}


% \ssc{Higher $K$-groups}

\begin{dfn}[\Rnum{4}.1.3, \Rnum{4}.1.4.1]
    A topological space $E$ is \tbf{acyclic} if it has the homology of a point. Let $X,Y$ be pointed connected CW complexes, a map $f:X\ar Y$ is \tbf{acyclic} if its homotopy fiber $F(f)$ is acyclic, let $P$ be a perfect normal subgroup of $\pi_{1}(X)$, $f$ is called a \tbf{$+$-construction relative to $P$} if $P=\Ker(\pi_{1}(X)\ar\pi_{1}(Y))$.
\end{dfn}

\begin{thm}[Quillen, \Rnum{4}.1.5]
    Let $X$ be a pointed CW complex and $P$ a perfect normal subgroup of $\pi_{1}(X)$, then there exists a $+$-construction $f:X\ar X\xP$ relative to $P$, where $f$ and $X\xP$ are unique up to homotopy.
\end{thm}

\begin{dfn}[Quillen, \Rnum{4}.1.1]
    Let $R$ be a ring, $\GL(R)=\ilim_{n}\GL_{n}(R)$ and $E(R)=[\GL(R),\GL(R)]$ and $\bB\GL(R)$ its classifying space. Let $i:\bB\GL(R)\ar\bB\GL(R)\xP$ be the $+$-construction relative to $E(R)$, the $K$-groups of $R$ are defined to be $K_{n}(R):=\pi_{n}(\bB\GL(R)\xP)$ for $n\gqs1$. Denote by $K(R):=K_{0}(R)\tms\bB\GL(R)\xP$.
\end{dfn}

\begin{thm}[Loday, \Rnum{4}.1.10]
    Let $R,S$ be two rings, then the product map $K_{p}(R)\ot K_{q}(S)\ar K_{p+q}(R\ot S)$, where $p,q\gqs1$, is natural in $R$ and $S$, bilinear and associative.
\end{thm}

\begin{thm}[{\cite{Quillen1972}}, \Rnum{4}.1.12]
    The Brauer lifting associated to an embedding $\oL{\bF}_{q}\xT\ar\bC\xT$ gives a homotopy fibration $\bB\GL(\bF_{q})\xP\axr{\rho}\bB\RU\axr{\psi^{q}-1}\bB\RU$, thus the associated long exact sequence gives
    \[
        \begin{tikzcd}
            \cds\ar[r] & \pi_{2k}(\bB\RU)\ar[d, equal]\ar[r] & \pi_{2k}(\bB\RU)\ar[r]\ar[d, equal] & \pi_{2k-1}(\bB\GL(\bF_{q})\xP)\ar[r]\ar[d, equal] & \pi_{2k-1}(\bB\RU)\ar[d, equal] \ar[r] & \cds\\
            \cds\ar[r] & \bZ\ar[r, "q^{k}-1"] & \bZ\ar[r] & \bZ/(q^{k-1})\ar[r] & 0 \ar[r] & \cds
        \end{tikzcd}
    \]

    Thus one has
    \[
        \RMK_{r}(\bF_{q})=\begin{cases}
          \bZ & r = 0 \\
          \bZ/(q^{k}-1) & r = 2k-1 \\
          0 & \T{otherwise}
        \end{cases}
    \]
\end{thm}

\begin{thm}[Borel, \Rnum{4}.1.17, \Rnum{4}.1.18]
    Let $A$ be a finite-dimensional semisimple $\bQ$-algebra, and $R$ any order in $A$, then $K_{n}(R)\ot\bQ\cong K_{n}(A)\ot\bQ$ for all $n\gqs2$. Moreover, if $F$ is a number field with $F\ot_{\bQ}\bR=\bR^{r_{1}}\ot\bC^{r_{2}}$, $A$ a central simple $F$-algebra, then one has $K_{n}(A)\ot\bQ\cong K_{n}(F)\ot\bQ$ for all $n\gqs2$, and
    \[
       \Rnk_{\bQ}K_{n}(A)\ot\bQ=\begin{cases}
         r_{2} & n\eqv 3\bmod{4} \\
         r_{1}+r_{2} & n\eqv 1\bmod{4} \\
         0 & \T{otherwise}
       \end{cases}
    \]
\end{thm}

% \ssc{$K$-theory with finite coefficient}

\begin{stp}
    Let $\ell$ be a positive integer.
\end{stp}

\begin{dfn}[\Rnum{4}.2.1, \Rnum{4}.2.4]
    Take $m\gqs2$, let $P^{m}(\bZ/\ell)$ be the Moore space of $\bZ/\ell$ for $m$.

    Let $X$ be a pointed topological space, then define the pointed set $\pi_{m}(X;\bZ/\ell):=[P^{m}(\bZ/\ell),X]$, it is a group for $m\gqs3$ and abelian for $m\gqs4$. Let $R$ be a ring, then define $K_{m}(R;\bZ/\ell):=\pi_{m}(K(R); \bZ/\ell)$.
\end{dfn}

\begin{thm}
    If $\ell=pq$ with $p,q$ are relatively prime, then $\pi_{m}(X;\bZ/\ell)\cong\pi_{m}(X;\bZ/p)\tms\pi_{m}(X;\bZ/q)$ naturally.
\end{thm}

\begin{thm}[Universal Coefficient, \Rnum{4}.2.2, \Rnum{4}.2.5]
    For $m\gqs3$, there are natural short exact sequences
    \begin{align*}
        0\ar\pi_{m}(X)\ot\bZ/\ell\ar\pi_{m}(X;\bZ/\ell)\ar\pi_{m-1}(X)[\ell]\ar0,\\
        0\ar K_{m}(R)\ot\bZ/\ell\ar K_{m}(R;\bZ/\ell)\ar K_{m-1}(R)[\ell]\ar 0,
    \end{align*}
    which is split (not naturally) when $\ell\not\eqv 2\bmod{4}$.
\end{thm}

\begin{thm}[Gabber's rigidity, \Rnum{4}.2.10]
    Let $(R,I)$ be a Henselian pair with $1/\ell\in R$, then for all $n\gqs1$, $K_{n}(R;\bZ/\ell)\cong K_{n}(R/I;\bZ/\ell)$.
\end{thm}




\sct{Computations}

Ref:~arXiv2405.04329.

\begin{stp}
    Let $p$ be a prime, $K$ a $p$-adic field of ramification degree $e$ and residue degree $f$ with uniformizer $\vpi$, denote by $R_{n}:=\cO_{K}/\vpi^{n}$. Then $\RMK_{r}(R_{n})$ can be effectively determined for all $r\gqs0$ and $n\gqs1$.

\end{stp}

 $\RMK_{r}(R_{n};\bZ[1/p])\cong \RMK_{r}(\bF_{q};\bZ[1/p])$ for $r\gqs0$.
\begin{cmt}[1]
if $n\lqs e$, $R_{n}=\bF_{q}[z]/z^{n}$, by Hesselholt-Madsen\cite{HM1997Cyclic}, \red{(to add)}.
\end{cmt}

(\cite{DGMC2013}) $\RMK_{r}(R_{n};\bZ_{p})\cong\mrm{TC}_{r}(R_{n};\bZ_{p})$ for $r\gqs0$. (\cite{BMS2019}) There is a strictly decreasing filtration $F_{\T{mot}}^{\gqs\blt}\mrm{TC}(R_{n};\bZ_{p})$ with graded piece $\Gr_{\T{mot}}^{k}\mrm{TC}(R_{n};\bZ_{p})\cong\bZ_{p}(k)(R)[2k]$ , whose spectral sequences converges to $\mrm{TC}_{r}(R_{n};\bZ_{p})$, consequently, one has
\[
    \RMK_{r}(R_{n};\bZ_{p})\cong \begin{cases}
      \HH{}{0}{(\bZ_{p}(0)(R_{n}))}\cong \bZ_{p} & r = 0 \\
      \HH{}{1}{(\bZ_{p}(k)(R_{n}))} & r=2k-1, k\gqs1 \\
      \HH{}{2}{(\bZ_{p}(k)(R_{n}))} & r=2k-2, k\gqs2
    \end{cases}
\]


Moreover, the cohomology of $\bZ_{p}(k)(R_{n}),k\gqs1$ could be effectively computed, there is a cochain complex
\[
    \bZ_{p}^{f(kn-1)}\axr{\mrm{syn}^{0}}\bZ_{p}^{2f(kn-1)}\axr{\mrm{syn}^{1}}\bZ_{p}^{f(kn-1)},
\]
where $\mrm{syn}^{0}$ and $\mrm{syn}^{1}$ are effectively computable, and which is quasi-isomorphic to $\bZ_{p}(k)(R)$. If $k\gqs2$ satisfies
\[
    k-1\gqs\frac{p}{p-1}\sR{\frac{p}{p-1}(p^{\sCe{n/e}}-1)-p^{\sCe{n/e}}(\sCe{n/e}-n/e)},
\]
then $\RMK_{2k-2}(R_{n};\bZ_{p})=0$.



\printref
\end{document}


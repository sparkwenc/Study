\documentclass[article, a4paper, twoside]{universal}

\setshowlvl{1}
\begin{document}
\confighead{}{}{}
\printhead{}{}{1}

Ref: arXiv2405.04329.


\begin{stp}
    Let $p$ be a prime, $K$ a $p$-adic field of ramification degree $e$ and residue degree $f$.

    Consider the problem of computing $K_{r}(\cO_{K}/\vpi^{n})$ for $r\gqs0$ and $n\gqs1$.
\end{stp}



If $n\gqs1$, by Quillen\cite{Quillen1972}, one has
\[
    K_{r}(\bF_{q})=\begin{cases}
      \bZ & r = 0 \\
      \bZ/(q^{k}-1) & r = 2k-1 \\
      0 & \T{otherwise}
    \end{cases}
\]

if $n\gqs e$, $\cO_{K}/\vpi^{n}=\bF_{q}[z]/z^{n}$, by Hesselholt-Madsen\cite{HM1997Cyclic}, one has ???



The paper started with Dundas-Goodwille-McCarthy, $K_{r}(\cO_{K}/\vpi^{n};\bZ_{p})\cong\mrm{TC}_{r}(\cO_{K}/\vpi^{n};\bZ_{p})$ for $r\gqs0$.

Then, by Bhatt-Morrow-Scholze, there is a strictly decreasing filtration $F_{\T{mot}}^{\gqs\blt}\mrm{TC}(R;\bZ_{p})$ with graded piece $\Gr_{\T{mot}}^{k}\mrm{TC}(R;\bZ_{p})\cong\bZ_{p}(k)(R)[2k]$, whose spectral sequences converges to $\mrm{TC}_{r}(R;\bZ_{p})$, consequently, one has
\[
    K_{r}(R;\bZ_{p})\cong \begin{cases}
      \HH{}{0}{(\bZ_{p}(0)(R))}\cong \bZ_{p} & r = 0 \\
      \HH{}{1}{(\bZ_{p}(k)(R))} & r=2k-1, k\gqs1 \\
      \HH{}{2}{(\bZ_{p}(k)(R))} & r=2k-2, k\gqs2
    \end{cases}
\]


Moreover, the cohomology of $\bZ_{p}(k)(R)$ could be effectively computed: For $k\gqs1$, there is a cochain complex
\[
    \bZ_{p}^{f(kn-1)}\axr{\mrm{syn}^{0}}\bZ_{p}^{2f(kn-1)}\axr{\mrm{syn}^{1}}\bZ_{p}^{f(kn-1)},
\]
where $\mrm{syn}^{0}$ and $\mrm{syn}^{1}$ are effectively computable, and which is quasi-isomorphic to $\bZ_{p}(k)(R)$.



\printref
\end{document}


\documentclass[article, a4paper, twoside]{universal}

\setshowlvl{1}
\begin{document}
\confighead{}{}{}
\printhead{}{}{1}

\sct{Homological algebra}
Ref:~\cite{Weibel1994}


\begin{dfn}[5.2.1, 5.2.8]
    A homology spectral sequence $(E_{p,q}^{r}) $ in an abelian category $\cA$ is the data of
    \begin{itm}
        \item family of objects $(E_{p,q}^{r})$ of $\cA$ for all $p,q\in\bZ$ and $r\gqs a$.
        \item maps $d_{p,q}^{r}:E_{p,q}^{r}\ar E_{p-r,q+r-1}^{r}$ such that $d_{p-r,q+r-1}^{r}\cc d_{p,q}^{r}=0$.
        \item isomorphisms $E_{p,q}^{r+1}\cong Z_{p,q}^{r+1}/B_{p,q}^{r+1}$, where $Z_{p,q}^{r+1}:=\Ker(d_{p,q}^{r})$ and $B_{p,q}^{r+1}:=\Img(d_{p+r,q-r+1}^{r})$.
    \end{itm}
    Define $Z_{p,q}^{\ift}:=\bcap_{r=a}^{\ift}Z_{p,q}^{r}$ and $B_{p,q}^{\ift}:=\bcup_{r=a}^{\ift}B_{p,q}^{r}$, and $E_{p,q}^{\ift}:=Z_{p,q}^{\ift}/B_{p,q}^{\ift}$, note that
    \[
        0=B_{p,q}^{a}\sbs\cds B_{p,q}^{r}\sbs\cds\sbs B_{p,q}^{\ift}\sbs Z_{p,q}^{\ift}\sbs\cds\sbs Z_{p,q}^{r}\sbs\cds\sbs Z_{p,q}^{a}=E_{p,q}^{a}.
    \]
\end{dfn}

\begin{dfn}[5.2.5, 5.2.7, 5.2.9, 5.2.10]
    Define the homology spectral sequence $(E_{p,q}^{r})$ to be
    \begin{itm}
        \item \tbf{bounded} if there are finitely many nonzero terms of total degree $n$ in $E_{\blt,\blt}^{a}$ for each $n$, for such sequence, there is $r_{0}$ for each $p,q$ such that $E_{p,q}^{r}\cong E_{p,q}^{r+1}\cong E_{p,q}^{\ift}$ for all $r\gqs r_{0}$.
        \item \tbf{collapses} at $E^{r}$ if there is exactly one nonzero row or column in $(E_{p,q}^{r})$.
        \item \tbf{bounded below} if for each $n$, there is $s$ such that terms of total degree $n$ in $E_{p,q}^{a}$ vanishes for $p<s$.
        \item \tbf{regular} if for each $p,q$, $d_{p,q}^{r}=0$ for $r\gg 0$, equivalently $Z_{p,q}^{r}=Z_{p,q}^{\ift}$ for $r\gg 0$.
    \end{itm}
\end{dfn}

\begin{dfn}[5.2.5, 5.2.11]
    Let $(E_{p,q}^{r})$ be a homology spectral sequence, and $(H_{n})$ be a family of objects in $\cA$ with an increasing filtration $F_{\blt}$. Say the sequence $(E_{p,q}^{r})$
    \begin{itm}
        \item \tbf{weakly converges to} $H_{\blt}$ if $E_{p,q}^{\ift}\cong F_{p}H_{p+q}/F_{p-1}H_{p+q}$ for all $p,q$.
        \item \tbf{approaches} $H_{\blt}$ if it weakly converges to $H_{\blt}$ and $H_{n}=\bcup_{p} F_{p}H_{n},\bcap_{p}F_{p}H_{n}=0,\fal n$.
        \item \tbf{converges} to $H_{\blt}$ if it approaches $H_{\blt}$, is regular, and $H_{n}=\plim_{p}(H_{n}/F_{p}H_{n}),\fal n$, denote by $E_{p,q}^{a}\aR H_{p+q}$.
    \end{itm}

    In particular, if the filtration $F_{\blt}$ is finite, and $(E_{p,q}^{r})$ is bounded, then weak convergence implies convergence.
\end{dfn}

\begin{thm}[Mapping lemma]
    Let $f:(E_{p,q}^{r})\ar(G_{p,q}^{r})$ be a morphism of spectral sequences such that for some $r$, $f^{r}:E_{p,q}^{r}\cong G_{p,q}^{r}$ for all $p,q$, then $f^{s}:E_{p,q}^{s}\cong G_{p,q}^{s}$ for all $s\gqs r$ and all $p,q$, as well as $f^{\ift}:E_{p,q}^{\ift}\cong G_{p,q}^{\ift}$.
\end{thm}

\begin{dfn}
    Let $C$ be a chain complex with an increasing filtration $F_{\blt}$. Define the filtration to be
    \begin{itm}
        \item \tbf{bounded} if for each $n$, there are $s<t$ such that $F_{s}C_{n}=0$ and $F_{t}C_{n}=C_{n}$.
        \item \tbf{bounded below} if for each $n$, there is $s$ such that $F_{s}C_{n}=0$.
        \item \tbf{bounded above} if for each $n$, there is $t$ such that $F_{t}C_{n}=C_{n}$.
        \item \tbf{exhaustive} if $\bcup_{p}F_{p}C=C$.
        \item \tbf{Hausdorff} if $\bcap_{p}F_{p}C=0$.
        \item \tbf{complete} if $C=\plim_{p}(C/F_{p}C)$.
    \end{itm}
\end{dfn}

\begin{thm}[5.4.1, 5.4.5, 5.5.1, 5.5.10]
    Let $C$ be a chain complex with an increasing filtration $F_{\blt}$, it naturally determines a homology spectral sequence starting with $E_{p,q}^{0}=F_{p}C_{p+q}/F_{p-1}C_{p+q}$ and $E_{p,q}^{1}=\HH{p+q}{}{(E_{p,\blt}^{0})}$. The spectral sequence arising from $C$ and $\oH{C}$ are the same.

    \begin{itm}
        \item If the filtration is bounded, then the spectral sequence is bounded and converges to $\HH{\blt}{}{(C)}$.
        \item If the filtration is bounded below and exhaustive, then the spectral sequence is bounded below and converges to $\HH{\blt}{}{(C)}$. The convergence is natural in that if $f:C\ar D$ is a map of filtered complexes, then $f\zS:\HH{\blt}{}{(C)}\ar\HH{\blt}{}{(D)}$ is compatible with the corresponding map of spectral sequences.
        \item If the filtration is complete and exhaustive, and the spectral sequence is regular, then
        \begin{itm}
            \item the spectral sequence weakly converges to $\HH{\blt}{}{(C)}$.
            \item if the spectral sequence is bounded above, then it converges to $\HH{\blt}{}{(C)}$.
        \end{itm}
    \end{itm}
\end{thm}

\begin{thm}[Eilenberg-Moore Comparison, 5.5.11]
    Let $f:C\ar D$ be a map of complexes with complete and exhaustive filtration, if for some $r\gqs0$ $f^{r}:E_{p,q}^{r}(C)\cong E_{p,q}^{r}(D)$ for all $p,q$, then $f\zS:\HH{\blt}{}{(C)}\cong\HH{\blt}{}{(D)}$.
\end{thm}

\begin{thm}[5.6.1, 5.6.2]
    Consider a double complex $C_{\blt,\blt}$ in an abelian category $\cA$.

    The column filtration ${}^{\Rnum{1}}F\zB\Tot(C_{\blt,\blt})$ gives a spectral sequence ${}^{\Rnum{1}}E_{p,q}^{r}$ starting with
    \[
        {}^{\Rnum{1}}E_{p,q}^{0}=C_{p,q},\quad {}^{\Rnum{1}}E_{p,q}^{1}=\HH{q}{v}{(C_{p,\blt})},\quad {}^{\Rnum{1}}E_{p,q}^{2}=\HH{p}{h}{\HH{q}{v}(C_{\blt,\blt})}.
    \]

    \begin{itm}
        \item If $C_{\blt,\blt}$ is in the first quadrant, ${}^{\Rnum{1}}E_{p,q}^{2}$ converges to $\HH{p+q}{}{(\Tot(C_{\blt,\blt}))}$.
        \item If $C_{p,q}=0$ in the second quadrant, ${}^{\Rnum{1}}E_{p,q}^{2}$ converges to $\HH{p+q}{}{(\Tot^{\op}(C_{\blt,\blt}))}$.
        \item If $C_{p,q}$ in the fourth quadrant, ${}^{\Rnum{1}}E_{p,q}^{2}$ weakly converges to $\HH{p+q}{}{(\Tot^{\prod}(C_{\blt,\blt}))}$.
    \end{itm}

    The row filtration ${}^{\Rnum{2}}F\zB\Tot(C_{\blt,\blt})$ gives a spectral sequence ${}^{\Rnum{2}}E_{p,q}^{r}$ starting with
    \[
        {}^{\Rnum{2}}E_{p,q}^{0}=C_{q,p},\quad {}^{\Rnum{2}}E_{p,q}^{1}=\HH{q}{h}{(C_{\blt,p})},\quad {}^{\Rnum{2}}E_{p,q}^{2}=\HH{p}{v}{\HH{q}{h}{(C_{\blt,\blt})}}.
    \]

    \begin{itm}
        \item If $C_{\blt,\blt}$ is in the first quadrant, ${}^{\Rnum{2}}E_{p,q}^{2}$ converges to $\HH{p+q}{}{(\Tot(C_{\blt,\blt}))}$.
        \item If $C_{p,q}=0$ in the fourth quadrant, ${}^{\Rnum{2}}E_{p,q}^{2}$ converges to $\HH{p+q}{}{(\Tot^{\op}(C_{\blt,\blt}))}$.
        \item If $C_{p,q}=0$ in the second quadrant, ${}^{\Rnum{2}}E_{p,q}^{2}$ weakly converges to $\HH{p+q}{}{(\Tot^{\prod}(C_{\blt,\blt}))}$.
    \end{itm}
\end{thm}

\begin{thm}[5.6.6, 5.6.3]
    Let $f:R\ar S$ be a map of rings, for every $R$-module $A$ and $S$-module $B$.
    \begin{itm}
        \item there is a first quadrant homology spectral sequence
        \[
            E_{p,q}^{2}:=\Tor{p}{S}{(\Tor{q}{R}{(A,S)},B)}\aR\Tor{p+q}{R}{(A,B)}.
        \]
        \item there is a first quadrant cohomology spectral sequence
        \[
            E_{2}^{p,q}:=\Ext{S}{p}{(B,\Ext{R}{q}{(S,A)})}\aR\Ext{R}{p+q}{(B,A)}.
        \]
    \end{itm}
\end{thm}

\begin{dfn}[5.7.4, 5.7.9]
    Let $F:\cA\ar\cB$ be a functor of abelian categories.

    \begin{itm}
        \item Suppose $F$ is right exact and $\cA$ has enough projectives. For a chain complex $A\zB$ in $\cA$, take a Cartan-Eilenberg resolution $P\zB\ar A\zB$, define $\bL_{k}F(A\zB):=\HH{k}{}{(\Tot^{\op}F(P\zB))}$, this gives the \tbf{left hyper-derived functors} $\bL_{k}F:\mrm{Ch}(\cA)\ar\cB$.
        \item Suppose $F$ is left exact and $\cA$ has enough injectives. For a chain complex $A\zB$ in $\cA$, take a Cartan-Eilenberg resolution $A\zB\ar I\zB$, define $\bR_{k}F(A\zB):=\HH{k}{}{(\Tot^{\prod}F(I\zB))}$, this gives the \tbf{right hyper-derived functors} $\bR_{k}F:\mrm{Ch}(\cA)\ar\cB$.
    \end{itm}

\end{dfn}

\begin{thm}[5.7.6, 5.7.9]
    The spectral sequence ${}^{\Rnum{2}}E_{p,q}^{2}=(L_{p}F)(\HH{q}{}{(A)})$ converges to $\bL_{p+q}F(A)$. If $A$ is bounded below, ${}^{\Rnum{1}}E_{p,q}^{2}=\HH{p}{}{(L_{q}F(A))}$ converges to $\bL_{p+q}F(A)$.

    The spectral sequence ${}^{\Rnum{2}}E_{2}^{p,q}=(R^{p}F)(\HH{}{q}{(A)})$ weakly converges to $\bR^{p+q}F(A)$. If $A$ is bounded below, ${}^{\Rnum{1}}E_{2}^{p,q}=\HH{}{p}{(R^{q}F(A))}$ converges to $\bR^{p+q}F(A)$.
\end{thm}

\begin{thm}[Grothendieck, 5.8.3]
    Let $\cA,\cB,\cC$ be abelian categories such that $\cA,\cB$ have enough injectives. Let $F:\cB\ar\cC$ and $G:\cA\ar\cB$ be left exact functors such that $G$ sends injective objects to $F$-acyclic objects. Then there exists a convergent first quadrant cohomology spectral sequence for each $A$ in $\cA$,
    \[
        {}^{\Rnum{1}}E_{2}^{p,q}=(R^{p}F)(R^{q}G)(A)\aR R^{p+q}(FG)(A).
    \]
\end{thm}

\sct{Higher algebra}
Ref:~\cite{Lurie2009}, Lurie-HigherAlgebra (2017),

% \ssc{Fonudations}

\begin{dfn}
    Let $\Dta^{n}$ be the standard $n$-simplex, and $\Dta$ the simplex category. A \tbf{simplicial object} in a category $\cC$ is a functor $\Dta^{\T{op}}\ar\cC$, in particular, a \tbf{simplicial set} $\cK$ is a presheaf of sets on the simplex category $\Dta$, denote by $\Set_{\Dta}$ the category of simplicial sets. For any category $\cC$, its nerve $N(\cC)$ is a simplicial set.
\end{dfn}

\begin{thm}[{\cite[1.1.2.2]{Lurie2009}}]
    Let $\cK$ be a simplicial set, there exists a small category $\cC$ such that $K\cong N(\cC)$ if and only if any map of simplicial set $f_{0}:\Lda_{i}^{n}\ar\cK$ can be \tbf{uniquely} lifted to $f:\Dta^{n}\ar\cK$ for $0<i<n$.
\end{thm}

\begin{thm}[Dold-Kan, 1.2.3.5, 1.2.3.7, 1.2.3.9, 1.2.3.15]
    Let $\cA$ be an abelian category.

    Take $(A\zB,d)\in\Ch(\cA)_{\gqs0}$, for each $n\gqs0$, $\DK_{n}(A\zB):=\Op_{\afa:[n]\atr[k]}A_{k}$ gives an object $\DK(A\zB)\in\Fun(\Dta^{\T{op}},\cA)$; for any morphism $f:[m]\ar[n]$, the induced $f\xS:\DK_{n}(A\zB)=\Op_{\afa:[n]\atr[k]}A_{k}\ar\DK_{m}(A\zB)=\Op_{\bta:[m]\atr[l]}A_{l}$ is given by the matrix $\{f_{\afa,\bta}:A_{k}\ar A_{l}\}$,
    \[
        f_{\afa,\bta}=\begin{cases}
          \Id & l=k,\quad \afa\cc f=([l]\cong[k])\cc\bta \\
          d & l=k-1,\quad \afa\cc f\=([k-1]\ar[k])\cc\bta \\
          0 & \T{otherwise}
        \end{cases}
    \]

    Take $A\zB\in\Fun(\Dta^{\T{op}},\cA)$, for each $n\gqs0$, $N_{n}(A):=\Ker(A_{n}\ar\op_{1\lqs i\lqs n}A_{n-1})$ defines a chain complex $N\zB(A\zB)$.

    The \tbf{Dold-Kan construction} $\DK:\Ch(\cA)_{\gqs0}\ar\Fun(\Dta^{\T{op}},\cA)$ is an equivalence of categories, whose homotopy inverse is given by the \tbf{normalized chain complex} $N\zB:\Fun(\Dta^{\T{op}},\cA)\ar\Ch(\cA)_{\gqs0}$.
\end{thm}

\begin{dfn}[{\cite[2.4.1.1]{Lurie2009}}]
    Let $p:X\ar S$ be an inner fibration of simplicial sets, $f:x\ar y$ an edge in $X$. Say that $f$ is \tbf{$p$-Cartesian} if the induced $X_{/f}\ar X_{/y}\tms_{S_{/p(y)}}S_{/p(f)}$ is a trivial Kan fibration, and \tbf{$p$-coCartesian} if it is Cartesian with respect to $p^{\T{op}}:X^{\T{op}}\ar S^{\T{op}}$.
\end{dfn}

\begin{thm}[{\cite[2.4.1.8]{Lurie2009}}]
    Let $p:Y\ar S$ be an inner fibration of simplicial sets and $e:\Dta^{1}\ar Y$ an edge, then $e$ is $p$-coCartesian if and only if for each $n\gqs1$ and diagram
    \[
        \begin{tikzcd}
            \{0\}\tms\Dta^{1}\ar[r, "e"]\ar[d, hook] & Y\ar[d, equal]\\
            \Dta^{n}\tms\{0\}\coprod_{\ptl\Dta^{n}\tms\{0\}}\ptl\Dta^{n}\tms\Dta^{1}\ar[r, "f"]\ar[d, hook] & Y\ar[d, "p"] \\
            \Dta^{n}\tms\Dta^{1}\ar[r, "g"]\ar[ru, dashed, "h"] & S
        \end{tikzcd}
    \]
    there exists the map $h$ making the diagram commutative.
\end{thm}

\begin{dfn}[0.0.0.1, Page~12]
    An \tbf{$\ift$-category/weak Kan complex} is a simplicial set $\cC$ such that any map of simplicial sets $f_{0}:\Lda_{i}^{n}\ar\cC$ can be lifted to $f:\Dta^{n}\ar\cC$ for $0<i<n$.

    A \tbf{space/Kan complex} is a simplicial set $\cC$ such that any map of simplicial sets $f_{0}:\Lda_{i}^{n}\ar\cC$ can be lifted to $f:\Dta^{n}\ar\cC$ for $0\lqs i\lqs n$. (\cite[1.2.16.1]{Lurie2009}) Denote by $\Kan$ the category of Kan complexes and $\cS:=N(\Kan)$ the \tbf{$\ift$-category of spaces}.

    A space $X$ is \tbf{$n$-connective} ($n\gqs0$) if $X$ is non-empty and $\pi_{i}(X,x)$ is trivial for $i<n$ and every vertex $x$ of $X$, \tbf{$n$-truncated} ($n\gqs-1$) if $\pi_{i}(X,x)$ is trivial for $i>n$ and every vertex $x$ of $X$.
\end{dfn}

\begin{dfn}[6.1.6.1, 6.1.6.4]
    Let $\cC$ be an $\ift$-category and $X$ a Kan complex, denote by $\cC^{X}$ the $\ift$-category $\Fun(X,\cC)$. A map $f:X\ar Y$ of Kan complexes induces $f\xS:\cC^{Y}\ar\cC^{X}$, if $\cC$ admits small limits and colimits, $f\xS$ admits left and right adjoint denoted by $f\zS$ and $f\zE$.

    Let $\cC$ be an $\ift$-category with an intial and final object, then any map of Kan complexes $f:X\ar Y$ with $(-1)$-truncated homotopy fibers determines the \tbf{norm map} $\Nrm_{f}:f\zE\ar f\zS$.
\end{dfn}

\begin{dfn}[6.1.6.2, 6.1.6.24]
    Let $G$ be a group, view $\bB G$ as a Kan complex, a \tbf{$G$-equivariant object} of $\cC$ is an object of $\cC^{\bB G}$, if $\cC$ admits small limits and colimits, the projection $\pr:\bB G\ar\Dta^{0}$ induces
    \[
        \pr\zS:\cC^{\bB G}\ar\cC, M\amt M^{G};\quad \pr\zE:\cC^{\bB G}\ar\cC, M\amt M_{G},
    \]
    the \tbf{Tate construction} of $M$ is defined by $M\amt M^{tG}:=\Cfb(\Nrm:M_{G}\ar M^{G})$.
\end{dfn}

\begin{dfn}[1.1.1.1]
    Let $\cC$ be an $\ift$-category, a \tbf{zero object} $0$ is an object which is both initial and final. Say $\cC$ is \tbf{pointed} if it contains a zero object.
\end{dfn}

\begin{thm}[6.1.6.7]
    Let $\cC$ be an $\ift$-category with an initial and final object, then $\cC$ is pointed if and only if for every map $f:X\ar Y$ of Kan complexes with $(-1)$-truncated homotopy fibers, the norm map $\Nrm_{f}:f\zE\ar f\zS$ is an equivalence.
\end{thm}

\begin{dfn}[1.1.1.4]
    Let $\cC$ be a pointed $\ift$-category, a triangle in $\cC$ is a diagram
    \[
        \begin{tikzcd}
            X\ar[r, "f"]\ar[d] & Y\ar[d, "g"] \\
            0 \ar[r] & Z
        \end{tikzcd}
    \]
    \begin{itm}
        \item it is a \tbf{fiber sequence} if it is a pullback square, write $X=\mrm{fib}(Y\axr{g}Z)$.
        \item it is a \tbf{cofiber sequence} if it is a pushout square, write $Z=\mrm{cofib}(X\axr{f}Y)$.
    \end{itm}
\end{dfn}

\begin{dfn}[1.1.1.9]
    An $\ift$-category $\cC$ is \tbf{stable} if the following holds
    \begin{enr}[label=(\arabic*)]
        \item There exists a zero object $0\in\cC$.
        \item Every morphism in $\cC$ admits a fiber and a cofiber.
        \item A triangle in $\cC$ is a fiber sequence if and only if it is a cofiber sequence.
    \end{enr}

    \red{Moreover, 1.1.3.4, 1.4.2.21}
\end{dfn}

\begin{thm}[Dold-Kan, 1.2.4.1]
    Let $\cC$ be a stable $\ift$-category, then the $\ift$-categories $\Fun(N(\bZ_{\gqs0}),\cC)$ and $\Fun(N(\Dta)^{\T{op}},\cC)$ are canonically equivalent.
\end{thm}

\begin{dfn}[1.1.2.7]
    Let $\cC$ be a stable $\ift$-category, denote by $\Sgm:\cC\ar\cC$ the \tbf{suspension functor} and $\Oga:\cC\ar\cC$ the \tbf{loop functor}. For $n\in\bZ$, define $X\amt X[n]$ as $X[n]:=\Sgm^{n}(X)$ for $n\gqs0$ and $\Oga^{-n}(X)$ for $n\lqs0$.
\end{dfn}

\begin{thm}[1.1.2.14]
    \red{To define a distinguished triangle, consider (1.1.2.11)}

    Let $\cC$ be a stable $\ift$-category, then the homotopy category $h\cC$ is triangulated.

\end{thm}

\begin{thm}[1.1.4.1]
    Let $F:\cC\ar\cD$ be a functor between stable $\ift$-categories, the following are equivalent, (1) $F$ is left exact, (2) $F$ is right exact, (3) $F$ is exact.
\end{thm}


\begin{dfn}[1.2.1.1]
    Let $\cD$ be a triangulated category, a \tbf{$t$-structure} on $\cD$ is a pair of full subcategories $\cD_{\gqs0},\cD_{\lqs0}$ such that
    \begin{enr}[label=(\arabic*)]
        \item For $X\in\cD_{\gqs0}, Y\in\cD_{\lqs0}$, one has $\Hom{\cD}{}{(X,Y[-1])}=0$.
        \item $\cD_{\gqs0}[1]\sbe\cD_{\gqs0}$ and $\cD_{\lqs0}[-1]\sbe\cD_{\lqs0}$.
        \item For $X\in\cD$, there exists a fiber sequence $U\ar X\ar V$ where $U\in\cD_{\gqs0}$ and $V\in\cD_{\lqs0}[-1]$.
    \end{enr}
\end{dfn}


\begin{dfn}[1.2.1.4, 1.2.1.11]
    Let $\cC$ be a stable $\ift$-category, a $t$-structure on $\cC$ is a $t$-structure on $h\cC$.

    The heart $\cC^{\hrts}$ is the full subcategory $\cC_{\lqs0}\cap\cC_{\gqs0}\sbs\cC$, denote by $\pi_{0}:\cC\ar\cC^{\hrts}$ the functor $\tau_{\lqs0}\cc\tau_{\gqs0}\cong\tau_{\gqs0}\cc\tau_{\lqs0}$, and $\pi_{n}=[-n]\cc\pi_{0}:\cC\ar\cC^{\hrts}$.
\end{dfn}


\begin{thm}[1.4.4.12]
    Let $\cC$ be a presentable stable $\ift$-category, a $t$-structure on $\cC$ is \tbf{accessible} if $\cC_{\gqs0}$ is presentable.

    \red{equivalently, 1.4.4.13}
\end{thm}



\begin{dfn}[1.4.2.1]
    Let $F:\cC\ar\cD$ be a functor between $\ift$-categories,
    \begin{enr}[label=(\arabic*)]
        \item if $\cC$ admits pushouts, then say $F$ is \tbf{excisive} if $F$ preserves pushout squares.
        \item if $\cC$ admits a final object $*$, then say $F$ is \tbf{reduced} if $F(*)$ is a final object.
    \end{enr}
\end{dfn}


\begin{dfn}[1.4.2.5, 1.4.2.8, 1.4.3.1, 1.4.4.5]
    Let $\cS$ be the $\ift$-category of spaces, $\cS^{\T{fin}}$ the $\ift$-category of finite spaces, and $\cS\zS^{\T{fin}}$ the $\ift$-category of its pointed objects.

    Let $\cC$ be an $\ift$-category which admits finite limits, a \tbf{spectrum object} of $\cC$ is a reduced, excisive functor $X:\cS\zS^{\T{fin}}\ar\cC$, let $\Spr(\cC)$ be the category of spectrum objects of $\cC$.

    A \tbf{spectrum} is a spectrum object of the $\ift$-category $\cS$ of spaces. Denote by $\Spr$ the $\ift$-category of spectra, the \tbf{sphere spectrum} $\bS\in\Spr$ is the image of the final object $*\in\cS$ under $\Sgm^{\ift}:\cS\ar\Spr$.

    A spectrum is \tbf{connective} if $\pi_{n}X=0$ for $n<0$, \tbf{discrete} if $\pi_{n}X=0$ for $n\neq0$.
\end{dfn}


\begin{thm}[1.4.3.6]
    $\Spr$ is a stable $\ift$-category, moreover,
    \begin{enr}[label=(\arabic*)]
        \item Let $\Spr_{\lqs-1}$ be the full subcategory spanned by those $X$ such that $\Oga^{\ift}(X)$ is contractible, then $\Spr_{\lqs-1}$ determines an accessible $t$-structure on $\Spr$.
        \item The $t$-structure on $\Spr$ is both left and right complete, $\Spr^{\hrts}$ is canonically equivalent to $N(\Ab)$.
    \end{enr}
\end{thm}


\begin{dfn}[2.0.0.2]
    Let $\Fin\zS$ be the category of pointed finite set, whose objects are $\sA{n}:=\{*,1,\cds,n\}$ for each $n\gqs0$, and whose morphisms are $\Hom{\Fin\zS}{}{(\sA{m},\sA{n})}:=\{\afa\in\Hom{\Set}{}{(\sA{m},\sA{n})}:\afa(*)=*\}$.

    For each $1\lqs i\lqs n$, denote by $\rho^{i}:\sA{n}\ar\sA{1}$ the morphism sending $i$ to $1$ and others to $*$.
\end{dfn}

\begin{dfn}[2.1.1.8, 2.1.2.1]
    Say a morphism $f:\sA{m}\ar\sA{n}$ is

    (1) \tbf{inert} if for each $i\in\sA{n}\xC$, $\sP{f\xI\{i\}}=1$, (2) \tbf{active} if $f\xI\{*\}=\{*\}$.
\end{dfn}

\begin{dfn}[2.1.1.10, 2.1.1.12]
    An \tbf{$\ift$-operad} is a functor $p:\cO^{\ot}\ar N(\Fin\zS)$ of $\ift$-categories such that
    \begin{enr}[label=(\arabic*)]
        \item For every inert morphism $f:\sA{m}\ar\sA{n}$ in $N(\Fin\zS)$ and $C\in\cO^{\ot}_{\sA{m}}$, there is a $p$-coCartesian $\oL{f}:C\ar D$ in $\cO^{\ot}$ lifting $f$, where $D\in\cO^{\ot}_{\sA{n}}$, this determines $f\zE:\cO_{\sA{m}}^{\ot}\ar\cO_{\sA{n}}^{\ot}$.
        \item Let $C\in\cO_{\sA{m}}^{\ot},D\in\cO_{\sA{n}}^{\ot}$, and choose $p$-coCartesian morphisms $D\ar D_{i}$ lying over $\rho^{i}$, then the induced $\Map_{\cO^{\ot}}^{f}(C,D)\ar\prd_{1\lqs i\lqs n}\Map_{\cO^{\ot}}^{\rho^{i}\cc f}(C,D_{i})$ is a homotopy equivalence.
        \item For a finite collection $C_{1},\lds,C_{n}\in\cO_{\sA{1}}^{\ot}$, there exists $C\in\cO_{\sA{n}}^{\ot}$ and a collection of $p$-coCartesian morphisms $C\ar C_{i}$ lifting $\rho^{i}:\sA{n}\ar\sA{1}$.
    \end{enr}

    Denote by $\cO:=\cO^{\ot}_{\sA{1}}=p\xI\{\sA{1}\}$ the \tbf{underlying $\ift$-category} of $\cO^{\ot}$.
\end{dfn}

\begin{dfn}[2.1.2.3]
    Let $p:\cO^{\ot}\ar N(\Fin\zS)$ be an $\ift$-operad, say a morphism $f$ in $\cO^{\ot}$ is

    (1) \tbf{inert} if $p(f)$ is inert and $f$ is $p$-coCartesian, (2) \tbf{active} if $p(f)$ is active.
\end{dfn}

\begin{dfn}[2.1.2.7, 2.1.2.10]
    Let $\cO^{\ot},\cP^{\ot}$ be $\ift$-operads, an \tbf{$\ift$-operad map} is a map of simplicial sets $f:\cO^{\ot}\ar\cP^{\ot}$ such that (1) $f$ preserves inert morphisms, (2) $f$ commutes with $\cO^{\ot}\ar N(\Fin\zS)$ and $\cP^{\ot}\ar N(\Fin\zS)$. Say $f$ if a \tbf{fibration of $\ift$-operads} if $f$ is a categorical fibration.
\end{dfn}

\begin{dfn}[2.1.3.1]
    Let $\cC^{\ot}\ar\cO^{\ot}$ be a fibration of $\ift$-operads, and $\afa:\cO'^{\ot}\ar\cO^{\ot}$ an $\ift$-operad map, denote by $\Alg_{\cO'/\cO}(\cC)$ the full subcategory of $\Fun_{\cO^{\ot}}(\cO'^{\ot},\cC^{\ot})$ spanned by the maps of $\ift$-operads. If $\cO^{\ot}=N(\Fin\zS)$, $\Alg_{\cO'/\cO}(\cC)$ is abbreviated as $\Alg_{\cO'}(\cC)$ and is referred to as the \tbf{$\cO'$-algebra objects} of $\cC$.
\end{dfn}


\begin{dfn}[2.4.2.1]
    Let $\cC$ be an $\ift$-category, $\cO^{\ot}$ an $\ift$-operad. A \tbf{$\cO$-monoid} in $\cC$ is a functor $M:\cO^{\ot}\ar\cC$ such that for every object $X\in\cO_{\sA{n}}^{\ot}$ corresponding to $\{X_{i}\in\cO\}_{1\lqs u\lqs n}$, the canonical maps $M(X)\ar M(X_{i})$ make $M(X)$ as the product $\prod_{1\lqs i\lqs n} M(X_{i})$.
\end{dfn}


\begin{dfn}[5.1.0.2, 5.1.0.4]
    Define a topological category ${}^{t}\bE_{k}^{\ot}$, whose objects are $\sA{n}\in\Fin\zS$, and whose morphisms are
    \[
        \Hom{{}^{t}\bE_{k}^{\ot}}{}{(\sA{m},\sA{n})}:=\cpd_{\afa\in\Hom{\Fin\zS}{}{(\sA{m},\sA{n})}}\prd_{1\lqs j\lqs n}\Rect(B_{1}(0)^{k}\tms\afa\xI(j)\ar B_{1}(0)^{k})
    \]
    with the induced topology. Denote by $\bE_{k}^{\ot}=N({}^{t}\bE_{k}^{\ot})$ the \tbf{$\ift$-operad of little $k$-cubes}.
\end{dfn}

\begin{thm}[5.1.0.7, 5.1.1.5]
    One has the equivalences of $\ift$-operads,

    (1) $\bE_{1}^{\ot}\cong\Ass^{\ot}$, (2) $\bE_{\ift}^{\ot}:=\ilim_{j\gqs0}\bE_{j}^{\ot}\cong \Com^{\ot}:=N(\Fin\zS)$.
\end{thm}


\begin{dfn}[7.1.0.1]
    Let $0\lqs k\lqs \ift$, $\mrm{Sp}$ the $\ift$-category of spectra, and $\bE_{k}^{\ot}$ the $\ift$-operad of little $k$-cubes. An \tbf{$\bE_{k}$-ring} is an $\bE_{k}$-algebra object of $\Spr$, denote by $\Alg^{(k)}:=\Alg_{\bE_{k}}(\Spr)$ the $\ift$-category of \tbf{$\bE_{k}$-ring spectra}.
\end{dfn}



\printref
\end{document}

